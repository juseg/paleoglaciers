% PALEOGLACIAL MODELLING
% ======================

% Packages
% --------
\documentclass{beamer}
\usefonttheme[onlymath]{serif}
\usepackage[utf8]{inputenc}
\usepackage{rotating}
\usepackage{tikz}
\usepackage{bm}
\hypersetup{pdfpagemode=FullScreen}
\DeclareGraphicsExtensions{.pdf,.png,.jpg}
\graphicspath{{figures/}}
\usetheme{Stockholm}

% Footline command
% ----------------
% From http://tex.stackexchange.com/questions/5491/how-do-i-insert-text-into-the-footline-of-a-specific-slide-in-beamer

\makeatletter

% add a macro that saves its argument
\newcommand{\footlineextra}[1]{\gdef\insertfootlineextra{#1}}
\newbox\footlineextrabox

% add a beamer template that sets the saved argument in a box.
% The * means that the beamer font and color "footline extra" are automatically added. 
\defbeamertemplate*{footline extra}{default}{
    \begin{beamercolorbox}[ht=2.25ex,dp=1ex,leftskip=\Gm@lmargin]{footline extra}
    \insertfootlineextra
    %\par\vspace{2.5pt}
    \end{beamercolorbox}
}

\addtobeamertemplate{footline}{%
    % set the box with the extra footline material but make it add no vertical space
    \setbox\footlineextrabox=\vbox{\usebeamertemplate*{footline extra}}
    \vskip -\ht\footlineextrabox
    \vskip -\dp\footlineextrabox
    \box\footlineextrabox%
}
{}

% patch \begin{frame} to reset the footline extra material
\let\beamer@original@frame=\frame
\def\frame{\gdef\insertfootlineextra{}\beamer@original@frame}
\footlineextra{}
\makeatother

\setbeamercolor{footline extra}{fg=structure.fg}% for instance

% This document
% -------------

\title[Palæoglacial modelling]{Numerical modelling of past glaciers flow}
\author[Julien Seguinot]{Julien Seguinot}
\institute[INK]{}
\date{6 December 2012}

\AtBeginSubsection[]{\begin{frame}{Outline}
	\tableofcontents[currentsection,currentsubsection]
\end{frame}}

% ======================================================================
\begin{document}
% ======================================================================

\maketitle

\begin{frame}{Outline}
	\tableofcontents
\end{frame}

% ----------------------------------------------------------------------
\section{Introduction}
% ----------------------------------------------------------------------

{\usebackgroundtemplate{\includegraphics[resolution=254]{map-northamerica-beamer+is}}\frame{\footlineextra{Data: ETOPO1; Natural Earth}}}

{\usebackgroundtemplate{\includegraphics[resolution=254]{eo-baffinseaice-crop1280x960-barnesicecap}}\begin{frame}{Barnes ice cap (1)}
	\footlineextra{Source: http://earthobservatory.nasa.gov}
\end{frame}}

{\usebackgroundtemplate{\includegraphics[resolution=254]{eo-barnesicecap-crop1280x960-northslope}}\begin{frame}{Barnes ice cap (2)}
	\footlineextra{Source: http://earthobservatory.nasa.gov}
\end{frame}}

{\usebackgroundtemplate{\includegraphics[resolution=254]{eo-barnesicecap-crop1280x960-geelake}}\begin{frame}{Barnes ice cap (3)}
	\footlineextra{Source: http://earthobservatory.nasa.gov}
\end{frame}}

{\usebackgroundtemplate{\includegraphics[resolution=254]{10-07-03-16c-crop1280x960}}\frame{\frametitle{Meltwater channels in British Columbia}}}

{\usebackgroundtemplate{\includegraphics[resolution=254]{eo-usnortheast-crop1280x960-longisland}}\begin{frame}{Long Island and Cape Cod}
	\begin{block}<2->{Geomorphological evidence...}
		\begin{itemize}[<+(1)->]
			\item ... is sparse in space
			\item ... is sparse in time
			\item ... requires interpretation
			\item ... do not include \alert{physical processes}
		\end{itemize}
	\end{block}
	\footlineextra{Source: http://earthobservatory.nasa.gov}
\end{frame}}

\begin{frame}{Why model past glaciers?}
	\begin{itemize}[<+->]
		\item To \alert{complement/interpret} geomorphological evidence
		\item To \alert{test and constrain} glacier models
		\item To \alert{spin-up} an ice sheet model for a projection run
	\end{itemize}
\end{frame}

% ----------------------------------------------------------------------
\section{Structure of a numerical glacier model}
% ----------------------------------------------------------------------

% ⋅⋅⋅⋅⋅⋅⋅⋅⋅⋅⋅⋅⋅⋅⋅⋅⋅⋅⋅⋅⋅⋅⋅⋅⋅⋅⋅⋅⋅⋅⋅⋅⋅⋅⋅⋅⋅⋅⋅⋅⋅⋅⋅⋅⋅⋅⋅⋅⋅⋅⋅⋅⋅⋅⋅⋅⋅⋅⋅⋅⋅⋅⋅⋅⋅⋅⋅⋅⋅⋅
\subsection{Overview}
% ⋅⋅⋅⋅⋅⋅⋅⋅⋅⋅⋅⋅⋅⋅⋅⋅⋅⋅⋅⋅⋅⋅⋅⋅⋅⋅⋅⋅⋅⋅⋅⋅⋅⋅⋅⋅⋅⋅⋅⋅⋅⋅⋅⋅⋅⋅⋅⋅⋅⋅⋅⋅⋅⋅⋅⋅⋅⋅⋅⋅⋅⋅⋅⋅⋅⋅⋅⋅⋅⋅

\begin{frame}[fragile]{What is a numerical glacier model?}
	\begin{tikzpicture}[align=center]
		\fill[olivblad!80] (-6,-2.5) rectangle (-2.5,2.5);
		\fill[vatten!80] (-2,-3) rectangle (2,3);
		\fill[eld!80] (2.5,-2.5) rectangle (6,2.5);
		\draw[-latex] (-2.5,0) -- (-2,0);
		\draw[-latex] (2,0) -- (2.5,0);
		\node at (-4.25,0) {
			\only<-1>{INPUT}
			\only<2->{
				topography\\[1em]
				climate\\[1em]
				basal heat flux\\[1em]
				initial conditions\\[1em]
				...}
		};
		\node at (0,0) {
			\only<-3>{PROGRAM}
			\only<4->{
				thermodynamical core\\[3em]
				boundary conditions}
		};
		\node at (4.25,0) {
			\only<-2>{OUTPUT}
			\only<3->{
				glacier geometry\\[1em]
				velocities\\[1em]
				temperature\\[1em]
				stresses\\[1em]
				...}
		};
	\end{tikzpicture}
\end{frame}

\begin{frame}[label=real-to-num]{How is a numerical glacier model built?}
	\includegraphics{graph-real-to-num}
\end{frame}

% ⋅⋅⋅⋅⋅⋅⋅⋅⋅⋅⋅⋅⋅⋅⋅⋅⋅⋅⋅⋅⋅⋅⋅⋅⋅⋅⋅⋅⋅⋅⋅⋅⋅⋅⋅⋅⋅⋅⋅⋅⋅⋅⋅⋅⋅⋅⋅⋅⋅⋅⋅⋅⋅⋅⋅⋅⋅⋅⋅⋅⋅⋅⋅⋅⋅⋅⋅⋅⋅⋅
\subsection{The thermodynamical core}
% ⋅⋅⋅⋅⋅⋅⋅⋅⋅⋅⋅⋅⋅⋅⋅⋅⋅⋅⋅⋅⋅⋅⋅⋅⋅⋅⋅⋅⋅⋅⋅⋅⋅⋅⋅⋅⋅⋅⋅⋅⋅⋅⋅⋅⋅⋅⋅⋅⋅⋅⋅⋅⋅⋅⋅⋅⋅⋅⋅⋅⋅⋅⋅⋅⋅⋅⋅⋅⋅⋅

\begin{frame}{\insertsubsection}
	\begin{itemize}[<+->]
		\item Balance of stresses (Navier-Stockes equation)
		$$\vec{\mathrm{div}} \, \bm\sigma + \rho \, \vec{g} = 0 \qquad
			\nabla \cdot \bm\sigma + \rho \, \bm{g} = 0$$
		\item Incompressibility of flow (conservation of volume)
		$$\mathrm{div} \, \vec{v} = 0 \qquad
			\nabla \cdot \bm{v} = 0$$
		\item Constitutive law for ice (Glen's law)
		$$\bm{\dot\epsilon} = A_0 \, e^\frac{-Q}{RT*} \, \bm\tau_e^{n-1} \bm{\tau}$$
		\item Thermal equation (conservation of energy)
		$$\frac{\partial T}{\partial t}
				= \vec{v} \cdot \vec{\mathrm{grad}}\,T
				+ \frac{k}{\rho c} \Delta T
				+ \frac{4 \mu \dot \epsilon_e^2}{\rho c} \qquad
			\frac{\partial T}{\partial t}
				= \mathbf{v} \cdot \nabla T
				+ \frac{k}{\rho c} \Delta T
				+ \frac{4 \mu \dot \epsilon_e^2}{\rho c}$$
	\end{itemize}
	\footlineextra{Source: Larour et al. 2012}
\end{frame}

\begin{frame}{Shallow approximations of the stress balance}
	$$\vec{\mathrm{div}} \, \bm\sigma + \rho \, \vec{g} = 0
		\qquad\Rightarrow\qquad\left\{\begin{array}{l}
		\alert<3-4>{\frac{\partial\tau_{xx}}{\partial x}}
		\alert<3-4>{+\frac{\partial\tau_{xy}}{\partial y}}
		\alert<2-4>{+\frac{\partial\tau_{xz}}{\partial z}}
		\alert<2-4>{=\frac{\partial p}{\partial x}}\\
		\alert<3-4>{\frac{\partial\tau_{yx}}{\partial x}}
		\alert<3-4>{+\frac{\partial\tau_{yy}}{\partial y}}
		\alert<2-4>{+\frac{\partial\tau_{yz}}{\partial z}}
		\alert<2-4>{=\frac{\partial p}{\partial y}}\\
		\alert<4-4>{\frac{\partial\tau_{zx}}{\partial x}}
		\alert<4-4>{+\frac{\partial\tau_{zy}}{\partial y}}
		\alert<4-4>{+\frac{\partial\tau_{zz}}{\partial z}}
		\alert<2-4>{=\frac{\partial p}{\partial z} - \rho g}
		\end{array}\right.$$
	\begin{itemize}[<+(1)-| alert@+(1)>]
		\item Shallow ice approximation
		\item Shallow shelf approximation
		\item Full stokes
		\item<+(1)-> And several others...
	\end{itemize}
	\pause
	\begin{block}{Note}
		Stress balance approximations often define the scope of a glacier model.
	\end{block}
\end{frame}

% ⋅⋅⋅⋅⋅⋅⋅⋅⋅⋅⋅⋅⋅⋅⋅⋅⋅⋅⋅⋅⋅⋅⋅⋅⋅⋅⋅⋅⋅⋅⋅⋅⋅⋅⋅⋅⋅⋅⋅⋅⋅⋅⋅⋅⋅⋅⋅⋅⋅⋅⋅⋅⋅⋅⋅⋅⋅⋅⋅⋅⋅⋅⋅⋅⋅⋅⋅⋅⋅⋅
\subsection{Boundary conditions}
% ⋅⋅⋅⋅⋅⋅⋅⋅⋅⋅⋅⋅⋅⋅⋅⋅⋅⋅⋅⋅⋅⋅⋅⋅⋅⋅⋅⋅⋅⋅⋅⋅⋅⋅⋅⋅⋅⋅⋅⋅⋅⋅⋅⋅⋅⋅⋅⋅⋅⋅⋅⋅⋅⋅⋅⋅⋅⋅⋅⋅⋅⋅⋅⋅⋅⋅⋅⋅⋅⋅

\begin{frame}{\insertsubsection\ (1)}
	\includegraphics[width=\linewidth]{pism-climate-cartoon}
	\footlineextra{Source: PISM documentation (http://pism-docs.org)}
\end{frame}

\begin{frame}{\insertsubsection\ (2)}
	\begin{itemize}[<+->]
		\item At the atmosphere interface...
			\begin{itemize}
				\item mass is gained and lost (surface mass balance)
				\item air temperature controls ice temperature
			\end{itemize}
		\item At the ocean interface...
			\begin{itemize}
				\item mass is gained and lost (basal mass balance)
				\item water temperature controls ice temperature
				\item calving of icebergs can occur
			\end{itemize}
		\item Lithosphere interface...
			\begin{itemize}
				\item mass is gained and lost (basal mass balance)
				\item geothermal heat flux controls ice temperature
				\item basal topography controls flow
				\item sliding can occur
			\end{itemize}
	\end{itemize}
	\begin{block}<+->{Note}
		Boundary models are often crude simplifications of complex processes.
	\end{block}
\end{frame}

% ⋅⋅⋅⋅⋅⋅⋅⋅⋅⋅⋅⋅⋅⋅⋅⋅⋅⋅⋅⋅⋅⋅⋅⋅⋅⋅⋅⋅⋅⋅⋅⋅⋅⋅⋅⋅⋅⋅⋅⋅⋅⋅⋅⋅⋅⋅⋅⋅⋅⋅⋅⋅⋅⋅⋅⋅⋅⋅⋅⋅⋅⋅⋅⋅⋅⋅⋅⋅⋅⋅
\subsection{Numerical implementation}
% ⋅⋅⋅⋅⋅⋅⋅⋅⋅⋅⋅⋅⋅⋅⋅⋅⋅⋅⋅⋅⋅⋅⋅⋅⋅⋅⋅⋅⋅⋅⋅⋅⋅⋅⋅⋅⋅⋅⋅⋅⋅⋅⋅⋅⋅⋅⋅⋅⋅⋅⋅⋅⋅⋅⋅⋅⋅⋅⋅⋅⋅⋅⋅⋅⋅⋅⋅⋅⋅⋅

\begin{frame}{\insertsubsection}
	\begin{block}{Problem statement}
		The equations of the physical model describe \alert{continuous} fields, but computers can deal with a \alert{finite number of points} in time and space
	\end{block}
\end{frame}

\againframe{real-to-num}

\begin{frame}{Example: the equation of temperature diffusion}
	\begin{itemize}[<+->]
		\item We use the previous temperature equation...
		$$\frac{\partial T}{\partial t}
			= \mathbf{v} \cdot \vec{\mathrm{grad}}\,T
			+ \frac{k}{\rho c} \Delta T
			+ \frac{4 \mu \dot \epsilon_e^2}{\rho c}$$
		\item ... considering only diffusion...
		$$\frac{\partial T}{\partial t}
			= \frac{k}{\rho c} \Delta T$$
		\item ... in the vertical dimension only.
		$$\frac{\partial T}{\partial t}
			= \kappa\,\frac{\partial^2 T}{\partial z^2}
			\qquad \mathrm{with} \qquad
			\kappa = \frac{k}{\rho c}$$
	\end{itemize}
\end{frame}

\begin{frame}{Discretization}
	\begin{itemize}[<+->]
		\item In the numerical model, temperature is defined on a grid:
		$$T_i^n=T(z_i,t^n)$$
		\item Derivatives are approximated (\alert{finite difference method}):
		$$\frac{\partial T}{\partial t}\simeq\frac{T_i^{n+1}-T_i^n}{\delta t}
			\qquad \mathrm{and} \qquad
			\frac{\partial^2 T}{\partial z^2}\simeq\frac{T_{i+1}^n-2T_i^n+T_{i-1}^n}{\delta z^2}$$
		\item The temperature equation becomes:
		$$\frac{T_i^{n+1}-T_i^n}{\delta t}
		=\kappa\,\frac{T_{i+1}^n-2T_i^n+T_{i-1}^n}{\delta z^2}$$
		\item This method is said \alert{explicit} as one can write:
		$$T_i^{n+1} = f(T_{i+1}^n, T_i^n, T_{i-1}^n)$$
	\end{itemize}
\end{frame}

\begin{frame}{Some first-order discretization schemes}
	\begin{itemize}[<+->]
		\item Explicit scheme
		$$\frac{T_i^{n+1}-T_i^n}{\delta t}
			=\kappa\,\frac{T_{i+1}^n-2T_i^n+T_{i-1}^n}{\delta z^2}$$
		\item Implicit scheme
		$$\frac{T_i^{n+1}-T_i^n}{\delta t}
			=\kappa\,\frac{T_{i+1}^{n+1}-2T_i^{n+1}+T_{i-1}^{n+1}}{\delta z^2}$$
		\item Semi-implicit (time-centered) scheme
		$$\frac{T_i^{n+1}-T_i^n}{\delta t}
			=\frac{1}{2}\,\kappa\,\frac{T_{i+1}^n-2T_i^n+T_{i-1}^n}{\delta z^2}
			+\frac{1}{2}\,\kappa\,\frac{T_{i+1}^{n+1}-2T_i^{n+1}+T_{i-1}^{n+1}}{\delta z^2}$$
	\end{itemize}
\end{frame}

\begin{frame}{Numerical instability}
	Different schemes have different properties of \alert{convergence} and \alert{stability}.
	\begin{columns}
	\column{55mm}
		\begin{block}{Stable}
			\includegraphics[width=\linewidth]{temperature-stable}
		\end{block}
	\column{55mm}
		\begin{block}{Unstable}
			\includegraphics[width=\linewidth]{temperature-unstable}
		\end{block}
	\end{columns}
\end{frame}

\begin{frame}{Other discretization methods}
	There exist other methods than the \alert{finite difference} method:
	\begin{itemize}[<+->]
		\item Finite element method
		\item Discrete element method
		\item Spectral methods
		\item ...
	\end{itemize}
\end{frame}

% ----------------------------------------------------------------------
\section{Issues in palæoglacial modelling}
% ----------------------------------------------------------------------

% ⋅⋅⋅⋅⋅⋅⋅⋅⋅⋅⋅⋅⋅⋅⋅⋅⋅⋅⋅⋅⋅⋅⋅⋅⋅⋅⋅⋅⋅⋅⋅⋅⋅⋅⋅⋅⋅⋅⋅⋅⋅⋅⋅⋅⋅⋅⋅⋅⋅⋅⋅⋅⋅⋅⋅⋅⋅⋅⋅⋅⋅⋅⋅⋅⋅⋅⋅⋅⋅⋅
\subsection{The role of surface mass balance}
% ⋅⋅⋅⋅⋅⋅⋅⋅⋅⋅⋅⋅⋅⋅⋅⋅⋅⋅⋅⋅⋅⋅⋅⋅⋅⋅⋅⋅⋅⋅⋅⋅⋅⋅⋅⋅⋅⋅⋅⋅⋅⋅⋅⋅⋅⋅⋅⋅⋅⋅⋅⋅⋅⋅⋅⋅⋅⋅⋅⋅⋅⋅⋅⋅⋅⋅⋅⋅⋅⋅

{\usebackgroundtemplate{
	\includegraphics[resolution=254]{map-northamerica-beamer+is+md}
}\frame{\footlineextra{Data: ETOPO1; Natural Earth}}}

\begin{frame}{\insertsubsection}
	\includegraphics<+>[width=\linewidth]{fourclimates-modeloutput}
	\includegraphics<+>[width=\linewidth]{fourclimates-temperature}
	\includegraphics<+>[width=\linewidth]{fourclimates-precipitation}
\end{frame}

% ⋅⋅⋅⋅⋅⋅⋅⋅⋅⋅⋅⋅⋅⋅⋅⋅⋅⋅⋅⋅⋅⋅⋅⋅⋅⋅⋅⋅⋅⋅⋅⋅⋅⋅⋅⋅⋅⋅⋅⋅⋅⋅⋅⋅⋅⋅⋅⋅⋅⋅⋅⋅⋅⋅⋅⋅⋅⋅⋅⋅⋅⋅⋅⋅⋅⋅⋅⋅⋅⋅
\subsection{The role of resolution}
% ⋅⋅⋅⋅⋅⋅⋅⋅⋅⋅⋅⋅⋅⋅⋅⋅⋅⋅⋅⋅⋅⋅⋅⋅⋅⋅⋅⋅⋅⋅⋅⋅⋅⋅⋅⋅⋅⋅⋅⋅⋅⋅⋅⋅⋅⋅⋅⋅⋅⋅⋅⋅⋅⋅⋅⋅⋅⋅⋅⋅⋅⋅⋅⋅⋅⋅⋅⋅⋅⋅

\begin{frame}{\insertsubsection}
	\includegraphics[width=\linewidth]{golledge-etal-2012-fig06}
	\footlineextra{Source: Golledge et al. 2012}
\end{frame}

% ----------------------------------------------------------------------
\section*{Conclusions}
% ----------------------------------------------------------------------

\begin{frame}{\insertsection}
	\begin{itemize}
		\item Numerical glacier (ice sheet) models are great tools to study past glacier dynamics.
		\item But they should be used with caution and knowledge of their limitations.
	\end{itemize}
\end{frame}

% ======================================================================
\end{document}
% ======================================================================

