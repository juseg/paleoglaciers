\documentclass{beamer}
\usepackage[utf8]{inputenc}
%\usepackage[T1]{fontenc}
%\usepackage[frenchb]{babel}
%\usepackage{natbib}
\usepackage{rotating}
\usetheme{Stockholm}
\DeclareGraphicsExtensions{.pdf,.png,.jpg}
\graphicspath{{../../figures/}}
\let\Tiny=\tiny

\setbeamercolor{caption}{parent=frametitle}

\newcommand{\longcaptionbox}[1]{%
	\begin{beamercolorbox}[sep=1ex,center]{caption}%
		#1%
	\end{beamercolorbox}}

\newcommand{\captionbox}[1]{%
	\begin{beamercolorbox}[ht=2.5ex,dp=1ex,center]{caption}%
		#1%
	\end{beamercolorbox}}

\newcommand{\figurebox}[2]{%
	\includegraphics[width=\linewidth]{#1}\\%
	\captionbox{#2}}

\newcommand{\sidewaysbox}[2]{%
	\hfill%
	\begin{sideways}%
		\begin{minipage}{75mm}%
			\includegraphics[angle=-90,width=\linewidth]{#1}\newline%
			\captionbox{#2}%
		\end{minipage}%
	\end{sideways}%
	\hfill}

\title[Numerical glacier modelling]{Numerical glacier (ice sheet) modelling}
%\subtitle{}
\author[Julien Seguinot]{Julien Seguinot}
\institute[INK]{}
\date{1 Avril 2010}

\titlegraphic{\begin{columns}
	\column{40mm}
		\figurebox{10-09-04-07-w400}{A glacier}
	\column{40mm}
		\figurebox{code-sediment}{A glacier model}
	\column{40mm}
		\figurebox{10-09-21-corentin-05-w400}{A glacier modeller}
	\end{columns}}

\AtBeginSection[]{\begin{frame}{\insertsection}
	\tableofcontents[currentsection,hideothersubsections]
\end{frame}}

% BEGIN DOCUMENT
% ==============
\begin{document}

\maketitle



% Introduction
% ------------
\section*{Introduction}

% What is a glacier model?
% ~~~~~~~~~~~~~~~~~~~~~~~~
\subsection{What is a glacier model?}

\begin{frame}[label=ism-io]{A glacier model from outside}
	\only<1>{\figurebox{graph-ism-io}{A glacier model is a computer program}}
	\only<2>{\figurebox{graph-ism-io+out}{A glacier model is a computer program}}
\end{frame}

\begin{frame}{A glacier model from inside}
	\begin{columns}
	\column{60mm}
		\only<1>{\includegraphics[width=\linewidth]{code-sico-stresses}}
		\only<2>{\includegraphics[width=\linewidth]{temp-budd-1970-p2-5}}
	\column{60mm}
	\begin{itemize}[<+->]
		\item 10 to 10000s of lines of code
		\item Pages of equations
	\end{itemize}
	\end{columns}
\end{frame}

\begin{frame}{Why model glaciers?}
\begin{columns}
	\column{44mm}
		\figurebox{greve-2000-fig02-h660}{Future predictions}
	\column{46.7mm}
		\figurebox{golledge-etal-2008-fig09-h660}{Past reconstructions}
	\end{columns}
\end{frame}

%! TODO: side captions, and uncover piecewise

\begin{frame}[label=real-to-num]{How to model glaciers?}
	\figurebox{graph-real-to-num}{From the real world to the model}
\end{frame}

%! TODO: change physical world by real world

% The mathematical model
% ----------------------
\section{The mathematical model}

% Ice as a continuum
% ~~~~~~~~~~~~~~~~~~
\subsection{Ice as a continuum}

%! TODO: add a slide with numbers

% The variables
% ~~~~~~~~~~~~~
\subsection{The variables}

%! TODO: add drawing for the stresses

% The model components
% ~~~~~~~~~~~~~~~~~~~~
\subsection{The model components}

%! TODO: ism-io with components, inputs

% Approximations for ice flow
% ~~~~~~~~~~~~~~~~~~~~~~~~~~~
\subsection{Approximations for ice flow}

%! TODO: add text?



% The numerical model
% -------------------
\section{The numerical model}

% Discretization
% ~~~~~~~~~~~~~~
\subsection{Discretization}

\begin{frame}{Putting the equations on a discrete grid}
	\begin{itemize}
		\item Real cases can't be solved analytically
		\item Computers deal with a finite amount of quantities
		\item Inputs are generally defined on a grid
		\item Discretization in space
$$\frac{\partial T}{\partial x}\simeq\frac{T_{i+1}^n-T_{i}^n}{\delta x}$$
		\item Discretization in time
$$\frac{\partial T}{\partial t}\simeq\frac{T_i^{n+1}-T_i^n}{\delta t}$$
	\end{itemize}
\end{frame}

\begin{frame}{Numerical instability}
	\begin{columns}
	\column{60mm}
		\figurebox{bueler-2010-stable}{Stable}
	\column{60mm}
		\figurebox{bueler-2010-unstable}{Unstable}
	\end{columns}
\end{frame}

% Programming
% ~~~~~~~~~~~
\subsection{Programming}

\begin{frame}{Programming steps}
	\begin{itemize}
		\item Choose a programming environment
			\begin{itemize}
				\item High level: Octave, Matlab, Scilab...
				\item Low level: Fortran, C++...
			\end{itemize}
		\item Parse the code into parts (subroutines, classes, ...)
		\item Find best algorythms
		\item Debug, debug, debug...
	\end{itemize}
\end{frame}



% Verification and validation
% ---------------------------
\section{Verification and validation}

% Sources of error
% ~~~~~~~~~~~~~~~~
\subsection{Errors, verification, validation}

%! TODO: add ism-io with inputs and outputs

\begin{frame}{Sources of error}
	\begin{itemize}
		\item At the mathematical model level
			\begin{itemize}
				\item Equations do not model the right processes
				\item Processes are not modelled at all
				\item Approximations used become unvalid
				\item Inputs are poorly constrained (esp. climate)
				\item Initial conditions are not known
				\item Parameters are poorly constrained
			\end{itemize}
		\item At the numerical model level
			\begin{itemize}
				\item Stability condition is not respected
				\item Grid size is too rough
			\end{itemize}
		\item<2> An ice sheet model can model anything you want!
	\end{itemize}
\end{frame}

\againframe{real-to-num}

% Validation agains observations
% ~~~~~~~~~~~~~~~~~~~~~~~~~~~~~~
\subsection{Validation against observations}

\begin{frame}{Compare model output to surface velocities}
	\figurebox{bueler-etal-submitted-fig01}{Observed and modelled surface velocities on Greenland (Bueler et al.)}
\end{frame}

% Validation against geomorphology
% ~~~~~~~~~~~~~~~~~~~~~~~~~~~~~~~~
\subsection{Validation against geomorphology}

\begin{frame}{Compare model output to landform record}
	\begin{columns}
	\column{46.7mm}
		\figurebox{golledge-etal-2008-fig09-h660}{Golledge et al., 2008}
	\column{38.4mm}
		\figurebox{napieralski-etal-2007-fig02}{Napieralski et al., 2007}
	\end{columns}
\end{frame}

%! TODO: side captions

% Application to geomorphology
% ----------------------------
\section{Application to geomorphology}

% Landscape evolution modelling
% ~~~~~~~~~~~~~~~~~~~~~~~~~~~~~
\subsection{Lanscape evolution modelling}

\begin{frame}{Landscape evolution modelling}
	\figurebox{vanderbeek-2008-tectonique-climat-erosion}{Interactions between climate, tectonics and erosion (Van der Beek)}
\end{frame}

% Modelling glacial sedimentation
% ~~~~~~~~~~~~~~~~~~~~~~~~~~~~~~~
\subsection{Modelling glacial sedimentation}

\begin{frame}{A flowline sediment model}
	\figurebox{pollard-deconto-2007-fig03ab}{Pollard and DeConto, 2007}
\end{frame}

\begin{frame}{A map-plane sediment model}
	\begin{columns}
	\column{60mm}
		\figurebox{kleman-etal-2008-fig07a-w600}{Kleman et al., 2008}
	\column{60mm}
		\figurebox{pollard-deconto-2003-fig04-w600}{Pollard and DeConto, 2006}
	\end{columns}
\end{frame}

% Modelling glacial erosion
% ~~~~~~~~~~~~~~~~~~~~~~~~~
\subsection{Modelling glacial erosion}

\begin{frame}{Erosion of glacial through}
	\begin{columns}
	\column{22.4mm}
		\includegraphics[width=\linewidth]{harbor-1988-fig01-h660}
	\column{60mm}
		\captionbox{Harbor, 1988}
	\end{columns}
\end{frame}

\begin{frame}{Plucking}
	\begin{columns}
	\column{60mm}
		\figurebox{linked-cavities-w600}{Linked cavities}
	\column{60mm}
		\figurebox{08-06-14-31-crop600x450}{Plucking steps}
	\end{columns}
\end{frame}



% Conclusions
% -----------
%\section*{Conclusions}

\end{document}

