% PALEOGLACIAL MODELLING
% ======================

% Packages
% --------
\documentclass{beamer}
%\usepackage[frenchb]{babel}
%\usepackage[T1]{fontenc}
\usepackage[utf8]{inputenc}
%\usepackage{natbib}
\usepackage{rotating}
\DeclareGraphicsExtensions{.pdf,.png,.jpg}
\graphicspath{{figures/}}
\let\Tiny=\tiny
\usetheme{Stockholm}

% My commands
% -----------
%\renewcommand<>{\includegraphics}[1]{\includegraphics<#2>[resolution=254]{#1}}

\setbeamercolor{caption}{parent=frametitle}

\newcommand{\longcaptionbox}[1]{%
	\begin{beamercolorbox}[sep=1ex,center]{caption}%
		#1%
	\end{beamercolorbox}}

\newcommand{\captionbox}[1]{%
	\begin{beamercolorbox}[ht=2.5ex,dp=1ex,center]{caption}%
		#1%
	\end{beamercolorbox}}

\newcommand{\figurebox}[2]{%
	\centering%
	\includegraphics[width=\linewidth]{#1}\\%
	\captionbox{#2}}

\newcommand{\sidewaysbox}[2]{%
	\hfill%
	\begin{sideways}%
		\begin{minipage}{75mm}%
			\includegraphics[angle=-90,resolution=254]{#1}\newline%
			\captionbox{#2}%
		\end{minipage}%
	\end{sideways}%
	\hfill}

% Footline command
% ----------------
% From http://tex.stackexchange.com/questions/5491/how-do-i-insert-text-into-the-footline-of-a-specific-slide-in-beamer

\makeatletter

% add a macro that saves its argument
\newcommand{\footlineextra}[1]{\gdef\insertfootlineextra{#1}}
\newbox\footlineextrabox

% add a beamer template that sets the saved argument in a box.
% The * means that the beamer font and color "footline extra" are automatically added. 
\defbeamertemplate*{footline extra}{default}{
    \begin{beamercolorbox}[ht=2.25ex,dp=1ex,leftskip=\Gm@lmargin]{footline extra}
    \insertfootlineextra
    %\par\vspace{2.5pt}
    \end{beamercolorbox}
}

\addtobeamertemplate{footline}{%
    % set the box with the extra footline material but make it add no vertical space
    \setbox\footlineextrabox=\vbox{\usebeamertemplate*{footline extra}}
    \vskip -\ht\footlineextrabox
    \vskip -\dp\footlineextrabox
    \box\footlineextrabox%
}
{}

% patch \begin{frame} to reset the footline extra material
\let\beamer@original@frame=\frame
\def\frame{\gdef\insertfootlineextra{}\beamer@original@frame}
\footlineextra{}
\makeatother

\setbeamercolor{footline extra}{fg=structure.fg}% for instance

% This document
% -------------

\title[Palæoglacial modelling]{Numerical modelling of past glaciers and ice sheets}
%\subtitle{}
\author[Julien Seguinot]{Julien Seguinot}
\institute[INK]{}
\date{12 December 2012}

% TODO: title graphic?
%\titlegraphic{\begin{columns}
%	\column{40mm}
%		\figurebox{10-09-04-07-w400}{A glacier}
%	\column{40mm}
%		\figurebox{code-sediment}{A glacier model}
%	\column{40mm}
%		\figurebox{10-09-21-corentin-05-w400}{A glacier modeller}
%	\end{columns}}

\AtBeginSection[]{\begin{frame}{Objectives}
	\tableofcontents[currentsection,hideallsubsections]
\end{frame}}

% BEGIN DOCUMENT
% ==============
\begin{document}

\maketitle

\begin{frame}{Example: glaciation of the Alps}
	\includegraphics<1>{pism-alps-stepcool+lc+ssa-cool10-0000}
	\includegraphics<2>{pism-alps-stepcool+lc+ssa-cool10-0050}
	\includegraphics<3>{pism-alps-stepcool+lc+ssa-cool10-0100}\\
	\captionbox{Movie: simulation of glacial inception in the European Alps}
\end{frame}

\begin{frame}{Example: glaciation of the Alps}
	\begin{itemize}[<+->]
		\item Beautiful output
			\begin{itemize}
				\item Very detailed in time and space
				\item Many output fields (geometry, flow, rebound, and much more...)
				\item Highly dynamic
			\end{itemize}
		\item But how \alert{realistic} is that?
	\end{itemize}
\end{frame}

\begin{frame}{Objectives}
	\tableofcontents[hideallsubsections]
\end{frame}

% --------------------------------
\section{Why model past glaciers?}
% --------------------------------

% ~~~~~~~~~~~~~~~~~~~~~~~~~~~~~~~~~~~~~~~~~~~~~~~~~~~~~~~~~~~~~~~~~~~~~~~
\subsection{Example: reconstructing the glacial history of North America}
% ~~~~~~~~~~~~~~~~~~~~~~~~~~~~~~~~~~~~~~~~~~~~~~~~~~~~~~~~~~~~~~~~~~~~~~~

{\usebackgroundtemplate{\includegraphics[resolution=254]{map-northamerica-beamer}}\frame[label=map-northamerica]{
	\footlineextra{Data: ETOPO1; Natural Earth}
}}

{\usebackgroundtemplate{\includegraphics[resolution=254]{eo-barnesicecap-crop1280x960-northslope}}\frame{
	\frametitle{Barnes ice cap}
	\footlineextra{Source: http://earthobservatory.nasa.gov}
}}

{\usebackgroundtemplate{\includegraphics[resolution=254]{eo-barnesicecap-crop1280x960-geelake}}\frame{
	\frametitle{Barnes ice cap - meltwater channels}
	\footlineextra{Source: http://earthobservatory.nasa.gov}
}}

{\usebackgroundtemplate{\includegraphics[resolution=254]{eo-baffinseaice-crop1280x960-barnesicecap}}\frame{
	\frametitle{Barnes ice cap - paradoxical location}
	\footlineextra{Source: http://earthobservatory.nasa.gov}
}}

\begin{frame}{Barnes ice cap - summary}
	\begin{itemize}
		\item Little accumulation
		\item High melt
		\item Unstable location
	\end{itemize}
	$\Rightarrow$ The \alert{remnant} of a much larger ice sheet
\end{frame}

{\usebackgroundtemplate{\includegraphics[resolution=254]{map-northamerica-beamer}}\againframe{map-northamerica}}

{\usebackgroundtemplate{\includegraphics[resolution=254]{10-07-03-16c-crop1280x960}}\frame{
	\frametitle{British Columbia - meltwater channels}
}}

% TODO: more photos!

{\usebackgroundtemplate{\includegraphics[resolution=254]{map-northamerica-beamer}}\againframe{map-northamerica}}

{\usebackgroundtemplate{\includegraphics[resolution=254]{eo-usnortheast-crop1280x960-longisland}}\frame{
	\frametitle{Long Island and Cape Cod}
	\footlineextra{Source: http://earthobservatory.nasa.gov}
}}

\begin{frame}<1>[label=landforms]{The landform record}
	\begin{itemize}
		\item Markers of former ice margin locations
			\begin{itemize}
				\item Meltwater channels
				\item Terminal moraines
				\item ...
			\end{itemize}
		\pause
		\item Markers of former ice flow direction
			\begin{itemize}
				\item Glacial lineations
				\item Eskers
				\item ...
			\end{itemize}
	\end{itemize}
\end{frame}

\begin{frame}{Deglaciation maps of North America - method}
	\begin{columns}
	\column{80mm}
		\includegraphics[width=\linewidth]{dyke-prest-1987-s01-h720}
	\column{40mm}
		\centering
		\includegraphics[resolution=254]{kleman-etal-2006-fig02b-h300}\\
		\longcaptionbox{Reconstructing former positions of the ice margin}
	\end{columns}
	\footlineextra{Source Dyke and Prest, 1987; Kleman et al, 2006}
\end{frame}

\begin{frame}{Deglaciation maps of North America - result}
	\begin{columns}
	\column{75mm}
		\includegraphics<1>[width=\linewidth]{dyke-prest-1987-s02a-h720}
		\includegraphics<2>[width=\linewidth]{dyke-prest-1987-s02b-h720}
		\includegraphics<3>[width=\linewidth]{dyke-prest-1987-s02c-h720}
		\includegraphics<4>[width=\linewidth]{dyke-prest-1987-s02d-h720}
		\includegraphics<5>[width=\linewidth]{dyke-prest-1987-s03a-h720}
		\includegraphics<6>[width=\linewidth]{dyke-prest-1987-s03b-h720}
		\includegraphics<7>[width=\linewidth]{dyke-prest-1987-s03c-h720}
		\includegraphics<8>[width=\linewidth]{dyke-prest-1987-s03d-h720}
		\includegraphics<9>[width=\linewidth]{dyke-prest-1987-s04a-h720}
		\includegraphics<10>[width=\linewidth]{dyke-prest-1987-s04b-h720}
		\includegraphics<11>[width=\linewidth]{dyke-prest-1987-s04c-h720}
	\column{45mm}
		\only<1>{\longcaptionbox{18~000 years before present}}
		\only<2>{\longcaptionbox{14~000 years before present}}
		\only<3>{\longcaptionbox{13~000 years before present}}
		\only<4>{\longcaptionbox{12~000 years before present}}
		\only<5>{\longcaptionbox{11~000 years before present}}
		\only<6>{\longcaptionbox{10~000 years before present}}
		\only<7>{\longcaptionbox{9~000 years before present}}
		\only<8>{\longcaptionbox{8~400 years before present}}
		\only<9>{\longcaptionbox{8~000 years before present}}
		\only<10>{\longcaptionbox{7~000 years before present}}
		\only<11>{\longcaptionbox{5~000 years before present}}
	\end{columns}
	\footlineextra{Source: Dyke and Prest, 1987}
\end{frame}

\begin{frame}{Deglaciation maps of North America - drawbacks}
	\begin{itemize}
		\item Large uncertainties on the dates
			\begin{itemize}
				\item Deglaciation ages are \alert{minimum} ages
			\end{itemize}
		\pause
		\item Include subjective interpolation of the landform record
			\begin{itemize}
				\item In space
				\item In time
			\end{itemize}
		\pause
		\item Limited access to older landforms
			\begin{itemize}
				\item How about the build-up phase?
			\end{itemize}
	\end{itemize}
	\pause
	$\Rightarrow$ Let's have a look at the \alert{subglacial} record
\end{frame}

{\usebackgroundtemplate{\includegraphics[resolution=254]{sarah-ge-dubawnt-w1280}}\frame{
	\frametitle{Dubawnt Lake palæo-ice stream bed}
	\footlineextra{Source: Google Earth}
}}

\againframe<2>{landforms}

\begin{frame}{Interpreting glacial lineations}
	\includegraphics<1>[width=91.8mm]{kleman-etal-2006-fig10a}
	\includegraphics<2>[width=91.8mm]{kleman-etal-2006-fig10ab}
	\includegraphics<3>[width=91.8mm]{kleman-etal-2006-fig10}
	\footlineextra{Source: Kleman et al 2006}
\end{frame}

\begin{frame}{North american ice sheets - pre-LGM landforms}
	\includegraphics[resolution=254]{kleman-etal-2010-fig06}
	\footlineextra{Source: Kleman et al 2010}
\end{frame}

%\begin{frame}{How about other ice sheets?}
%	\begin{columns}
%	\column{75mm}
%		\includegraphics[width=\linewidth]{wp-grobe-2008-north-intgl+gl-h750}
%	\column{45mm}
%		\longcaptionbox{Present and maximum glaciation in the northern hemisphere.}
%	\end{columns}
%	\footlineextra{Source Grobe, 2008; after Ehlers and Gibbard, 2007.}
%\end{frame}

\begin{frame}{The landform record - summary}
	\begin{itemize}
		\item Sparse in space and time
		\item Very restricted before LGM
	\end{itemize}
	\pause
	$\Rightarrow$ \alert{Numerical modelling} can help
	\begin{itemize}
		\item Movie: the last glacial cycle in North America
	\end{itemize}
\end{frame}

% ~~~~~~~~~~~~~~~~~~~~~~~~~~~~~~~~~~~~~~~~~~~~~~~~~~~~~~~~~~~~~~~~~~~~
\subsection{Example: projecting the future of the Greenland ice sheet}
% ~~~~~~~~~~~~~~~~~~~~~~~~~~~~~~~~~~~~~~~~~~~~~~~~~~~~~~~~~~~~~~~~~~~~

\begin{frame}{Modelling the Greenland ice sheet}
	\figurebox{greve-2000-fig02-1x4-w1200}{Simulations of the Greenland ice sheet under warmer climates}
	\footlineextra{Source: Greve, 2000}
\end{frame}

\begin{frame}{Modelling the Greenland ice sheet - uncertainties}
	\figurebox{applegate-etal-disc-fig07}{Comparison of modelled ice surface to present day}
	\footlineextra{Source: Applegate et al., in review}
\end{frame}

\begin{frame}{Modelling the Greenland ice sheet - uncertainties}
	\figurebox{applegate-etal-disc-fig02b}{Future projections of ice volume change}
	\footlineextra{Source: Applegate et al., in review}
\end{frame}

% ~~~~~~~~~~~~~~~~~~
\subsection{Summary}
% ~~~~~~~~~~~~~~~~~~

\begin{frame}{\insertsection}
	\includegraphics<1>[width=120mm]{graph-ism-geom-01-ice-sheet-models}
	\includegraphics<2>[width=120mm]{graph-ism-geom-02-future-ice-sheets}
	\includegraphics<3>[width=120mm]{graph-ism-geom-03-physics}
	\includegraphics<4>[width=120mm]{graph-ism-geom-04-inputs}
	\includegraphics<5>[width=120mm]{graph-ism-geom-05-numerics}
	\includegraphics<6>[width=120mm]{graph-ism-geom-06-past-ice-sheets}
	\includegraphics<7>[width=120mm]{graph-ism-geom-07-geom-rec}
	\includegraphics<8>[width=120mm]{graph-ism-geom-08-geom-inputs}
	\includegraphics<9>[width=120mm]{graph-ism-geom-09-validation}
\end{frame}

% TODO: use North America figures instead of fennoscandia

\begin{frame}{\insertsection}
	\begin{itemize}
		\item Understanding past glacier fluctuations
		\item Constraining ice sheet models
	\end{itemize}
\end{frame}

% -----------------------------------------------------------
\section{What are the uncertainties of palæo-glacial models?}
% -----------------------------------------------------------

% ~~~~~~~~~~~~~~~~~~~
\subsection{Overview}
% ~~~~~~~~~~~~~~~~~~~

\begin{frame}[label=ism-io]{What is a glacier model?}
	\only<1>{\figurebox{graph-ism-io}{A glacier model is a computer program}}
	\only<2>{\figurebox{graph-ism-io+out}{A glacier model is a computer program}}
	\only<3>{\figurebox{graph-ism-io+in+out}{A glacier model is a computer program}}
\end{frame}

\begin{frame}[label=real-to-num]{What is a glacier model?}
	\figurebox{graph-real-to-num}{A glacier model is an approximation of reality}
\end{frame}

% ~~~~~~~~~~~~~~~~~~~
\subsection{Dynamics}
% ~~~~~~~~~~~~~~~~~~~

\begin{frame}{A hierarchy of stress balances}
	\begin{itemize}
		\item Full-stokes models
			\begin{itemize}
				\item Usually inappropriate for palæo-glacial modelling
			\end{itemize}
		\item Shallow models
			\begin{itemize}
				\item Zeroth order (SIA, SSA)
				\item First order
				\item 'Hybrid'
			\end{itemize}
	\end{itemize}
\end{frame}

% TODO: add figures, equations

\begin{frame}{Shallow-ice vs. 'hybrid' model}
	\includegraphics[height=64mm]{bueler-etal-submitted-fig01}\\
	\captionbox{Observed and modelled surface velocities on Greenland}
	\footlineextra{Source: Bueler et al., submitted}
\end{frame}

% TODO: add section about mass balance

% ~~~~~~~~~~~~~~~~~~~
\subsection{Numerics}
% ~~~~~~~~~~~~~~~~~~~

\begin{frame}{Discretization}
	\begin{itemize}
		\item Real cases can't be solved analytically
		\item Computers deal with a finite amount of quantities
		\item Inputs are generally defined on a grid
		\item Discretization in space
$$\frac{\partial T}{\partial x}\simeq\frac{T_{i+1}^n-T_{i}^n}{\delta x}$$
		\item Discretization in time
$$\frac{\partial T}{\partial t}\simeq\frac{T_i^{n+1}-T_i^n}{\delta t}$$
	\end{itemize}
\end{frame}

\begin{frame}{Discretization - numerical instability}
	\begin{columns}
	\column{60mm}
		\figurebox{bueler-2010-stable}{Stable}
	\column{60mm}
		\figurebox{bueler-2010-unstable}{Unstable}
	\end{columns}
\end{frame}

%\begin{frame}{Programming steps}
%	\begin{itemize}
%		\item Choose a programming environment
%			\begin{itemize}
%				\item High level: Octave, Matlab, Scilab...
%				\item Low level: Fortran, C++...
%			\end{itemize}
%		\item Parse the code into parts (subroutines, classes, ...)
%		\item Find best algorythms
%		\item Debug, debug, debug...
%	\end{itemize}
%\end{frame}

% TODO: bugs

% TODO: open-source models

% ~~~~~~~~~~~~~~~~~~
\subsection{Summary}
% ~~~~~~~~~~~~~~~~~~

\begin{frame}{Uncertainties in glacier models}
	\begin{itemize}[<+->]
		\item At the numerical level
			\begin{itemize}
				\item Discretization errors (small)
				\item Bugs
			\end{itemize}
		\item At the mathematical level
			\begin{itemize}
				\item Simplist physics (dynamics, mass balance, hydrology)
				\item Unconstrained parameters
				\item \alert{Unknown palæo-climate}
			\end{itemize}
	\end{itemize}
\end{frame}

\againframe{real-to-num}

% TODO: detail validation and verification

% -----------------------------------------------
\section{Application - the Cordilleran ice sheet}
% -----------------------------------------------

{\usebackgroundtemplate{\includegraphics[resolution=254]{map-cordillera-beamer}}\frame[label=map-cordillera]{
	\footlineextra{Data: ETOPO1; Natural Earth}
}}

\begin{frame}{Geomorphology}
	\begin{columns}
	\column{37mm}
		\includegraphics[width=\linewidth]{kleman-etal-2010-fig02-h720}
	\column{53mm}
		\longcaptionbox{Glacial lineation map}
	\end{columns}
	\footlineextra{Source: Kleman et al, 2006}
\end{frame}

\begin{frame}{Climate}
	\figurebox{cordillera-narr-10km-artm}{Mean monthly temperature}
\end{frame}

\begin{frame}{Climate}
	\figurebox{cordillera-narr-10km-precip}{Monthly precipitation}
\end{frame}

% TODO: wet and dry photos

\begin{frame}{Model output}
	\includegraphics[height=64mm]{pism-cordillera-stepcool+calv+lc+ssa+till-50ka-cool10-0000}\\
	\captionbox{Movie: simulation of glacial inception in the Cordillera}
\end{frame}

% TODO: show several frames

% -------------------
\section*{Conclusion}
% -------------------

\begin{frame}{Conclusions}
	\begin{itemize}
		\item Glaciers and ice sheets are very dynamic
		\item Their landform record is sparse in time and space
		\item Glacier and ice sheet models are uncertain
		\item But they can be constrained by geomorphology (cf. papers!)
	\end{itemize}
\end{frame}

\end{document}

