% Copyright (c) 2011--2019, Julien Seguinot <seguinot@vaw.baug.ethz.ch>
% Creative Commons Attribution-ShareAlike 4.0 International License
% (CC BY-SA 4.0, http://creativecommons.org/licenses/by-sa/4.0/)

% Beamer photogenic theme
\documentclass[aspectratio=1610]{beamer}
\usetheme{photogenic}
\usepackage{fontawesome}

% document properties
\title{Alpine palaeoglaciology}
\author{Julien Seguinot}
\institute{ETH Zürich}
\date{December 16, 2019}

% graphics
\graphicspath{{../figures/}}


% ======================================================================
\begin{document}
% ======================================================================

    % FIXME: add links to sources

    \centering  % center all frames

    \begin{backgroundframe}[b]{art/hodel-1927}{0.0}{}
      \flushleft\alphabox{
        \textbf{\inserttitle}\\
        \insertauthor, \insertinstitute, \insertdate\\
        \href{mailto: seguinot@vaw.baug.ethz.ch}
             {{\faEnvelope} seguinot@vaw.baug.ethz.ch}
        \href{http://people.ee.ethz.ch/~juliens}
             {{\faHome} http://people.ee.ethz.ch/$\sim$juliens}}
      \footlineextra{Artwork: \href{http://gletschergarten.ch}{Hodel, 1927}.}
    \end{backgroundframe}

    \begin{frame}{Past and present glaciers on Earth}
      % FIXME increase font size in plots
      % FIXME include ice shelves on worldmap
      \includegraphics<1>[width=\linewidth]{plot/worldmap-01}%
      \includegraphics<2>[width=\linewidth]{plot/worldmap-02}%
      \begin{columns}
        \only<1>{
          \column{0.5\textwidth}\centering
            Antarctica: 58.3 m s.l.e.\\ (Fretwell et al., 2013)
          \column{0.5\textwidth}\centering
            Greenland: 7.3 m s.l.e.\\ (Bamber et al., 2013)}
        \only<2>{
          \column{\textwidth}\centering
            Additional 120 to 135 m s.l.e. (Clark and Mix, 2002) \\
            \emph{How to reconstruct paleoglaciers in space and time?}}
      \end{columns}
      \footlineextra{Data:
        \href{http://www.naturalearthdata.com}{Natural Earth},
        \href{http://booksite.elsevier.com/9780444534477/}{Ehlers et al., 2011}.}
    \end{frame}


% ----------------------------------------------------------------------
\section{History of the glacial theory}
% ----------------------------------------------------------------------

% -- -- -- -- -- -- -- -- -- -- -- -- -- -- -- -- -- -- -- -- -- -- -- -
\subsection{Glacial erratics}
% The origin of the glacial theory in the Alps
% -- -- -- -- -- -- -- -- -- -- -- -- -- -- -- -- -- -- -- -- -- -- -- -

    \begin{sectionframe}{art/hodel-1927}{0.75}{Outline}
      \tableofcontents
      \footlineextra{Artwork: \href{http://gletschergarten.ch}{Hodel, 1927}.}
    \end{sectionframe}

    \begin{sectionframe}{art/hodel-1927}{0.75}{\insertsectionhead}
      \emph{Glacial erratics and glacial cycles.}
      \footlineextra{Artwork: \href{http://gletschergarten.ch}{Hodel, 1927}.}
    \end{sectionframe}

    \begin{backgroundframe}[t]{art/triner-1865}{0.0}{}
      \flushleft\alphabox{A glacial erratic near Zürich.}
      \footlineextra{Artwork:
        \href{https://www.ag.ch/staatsarchiv/suche/detail.aspx?ID=84843}
             {Triner, 1865}.}
    \end{backgroundframe}

    \begin{backgroundframe}[t]{art/cole-1829}{0.0}{}
      \flushleft\alphabox{The Subsiding of the Waters of the Deluge (1829)}
      \footlineextra{Artwork:
        \href{https://americanart.si.edu/artwork/subsiding-waters-deluge-5080}
             {Cole, 1829}.}
    \end{backgroundframe}

    \begin{backgroundframe}[b]{web/decharpentier-1841}{0.0}{}
      \flushleft\alphabox{{Mapping the erratic deposit (1841)}}
      \footlineextra{Source:  % FIXME use high-resolution version
        \href{http://shipseducation.net/glaciers/Charpentier.htm}
             {de Charpentier, 1841}.}
    \end{backgroundframe}

    \begin{frame}{The glacial theory was first controversial}
      \begin{columns}
        \column{60mm}
          \begin{itemize}[<+->]
            \item Glacial landforms had been explained by a great flood.
            \bigskip
            \item Greenland and Antarctica had not yet been explored.
            \bigskip
            \item One or several glaciations?
          \end{itemize}
        \column<2->{60mm}
          \includegraphics[width=60mm]{art/nansen-1888}
          \emph{The first crossing of Greenland} (Nansen, 1888).
      \end{columns}
      \footlineextra{Photo:
        \href{https://commons.wikimedia.org/wiki/File:Nansen\%27s_Greenland_expedition_crossing.jpg}
             {Nansen, 1890}.}
    \end{frame}

    \begin{frame}{Two glaciations in North America (1882)}
      \includegraphics[width=\linewidth]{web/chamberlin-1882}
      \footlineextra{Source:  % FIXME use high-resolution version
        \href{https://etc.usf.edu/maps/pages/10600/10623/10623.htm}
             {Chamberlin, 1882}.}
    \end{frame}

    \begin{frame}{Four glaciations in the Alps (1909)}
      \begin{columns}
        \column{45mm}
          \begin{itemize}
            \item<+-> In 1909, at least four glaciations
                      were identified in the Alps.
              \alert{four} glaciations
              \begin{itemize}
                \item Würm
                \item Riss
                \item Mindel
                \item Günz
              \end{itemize}
            \item<+-> In 2011, ``at least \alert{eight}, but probably more
                      lowland glaciations during the Quaternary.''
            \item<+-> \emph{How many glaciations were there in total?}
          \end{itemize}
        \column{75mm}
          \includegraphics<1->[height=80mm]{web/penck-bruckner-1909}
      \end{columns}
      \footlineextra{Source:
        \href{https://www.glaciers-climat.com/cg/le-quaternaire-dans-les-alpes/}
             {Penck and Brückner, 1909}; Preusser et al., 2011.}  %FIXME url
    \end{frame}


% -- -- -- -- -- -- -- -- -- -- -- -- -- -- -- -- -- -- -- -- -- -- -- -
\subsection{Glacial cycles}
% How many glaciations have there been in the past?
% -- -- -- -- -- -- -- -- -- -- -- -- -- -- -- -- -- -- -- -- -- -- -- -

    \begin{frame}{Ice sheets change the isotopic content of the ocean}
      \begin{columns}
        \column{60mm}
          \includegraphics<1->[width=\linewidth]{web/cartoon-umich-d18o-a}  % TODO
        \column{60mm}
          \includegraphics<2->[width=\linewidth]{web/cartoon-umich-d18o-b}  % TODO
      \end{columns}
      \footlineextra{Cartoon:
        \href{https://globalchange.umich.edu/globalchange1/current/labs/Lab10_Vostok/Vostok.htm}
             {University of Michigan}.}
    \end{frame}

    \begin{frame}[t]{The evolution of total ice volume on land}
      \begin{columns}
        \column{30mm}
          \vspace{9mm}
          \uncover<3->{\includegraphics[width=\linewidth]{web/csiro-2000-bubbles}}\\%
          \vspace{9mm}
          \uncover<1->{\includegraphics[width=\linewidth]{photo-morgane-foram-720p}}%  % TODO
          \vspace{9mm}
        \column{108mm}
          \includegraphics<1>[width=\linewidth]{plot/isotopes-01}%
          \includegraphics<2>[width=\linewidth]{plot/isotopes-02}%
          \includegraphics<3>[width=\linewidth]{plot/isotopes-03}%
      \end{columns}
      \footlineextra{Photo: Brosse, 2015;
        Data: Lisiecki and Raymo, 2005; Jouzel et al., 2007}  % FIXME url
    \end{frame}

    \begin{frame}{Mechanisms of glacial cycles}
      \includegraphics[height=80mm]{bib/hodell-2016-01}
      \footlineextra{Figure:
        \href{https://dx.doi.org/10.1126/science.aal4111}
             {Hodell, 2016.}}
    \end{frame}


% ----------------------------------------------------------------------
\section{Glacial landforms in the Alps}
% ----------------------------------------------------------------------

    \begin{sectionframe}{art/hodel-1927}{0.75}{\insertsectionhead}
      \emph{The traces left by glacial erosion and sedimentation.}
      \footlineextra{Artwork: \href{http://gletschergarten.ch}{Hodel, 1927}.}
    \end{sectionframe}

    \begin{backgroundframe}[b]{julien-xt10-170922-184315-184318-dev-5x8}{0.0}{}  % TODO
      \flushleft\alphabox{Gorner Glacier, 2017}
      \footlineextra{Photo: Seguinot, 2017.}
    \end{backgroundframe}

    \begin{backgroundframe}[b]{julien-xt1-190601-180527-dev}{0.0}{}  % TODO
      \flushleft\alphabox{Lauterbrunnen, 2019}
      \footlineextra{Photo: Seguinot, 2019.}
    \end{backgroundframe}


% -- -- -- -- -- -- -- -- -- -- -- -- -- -- -- -- -- -- -- -- -- -- -- -
\subsection{Rhône Glacier}
% Recent moraines in front 
% -- -- -- -- -- -- -- -- -- -- -- -- -- -- -- -- -- -- -- -- -- -- -- -

    \begin{backgroundframe}[t]{julien-xt1-150910-163122-dev}{0.0}{}
      \flushright\alphabox{Recent glacial landforms near the Rhône Glacier}
      \footlineextra{Photo: Seguinot, 2016.}
    \end{backgroundframe}

    \begin{backgroundframe}{julien-xt10-161027-152047-152107-dev-2x3}{0.75}
                           {The Rhône Glacier in 2016}
      \vspace{12mm}
      \includegraphics[width=\linewidth]{julien-xt10-161027-152047-152107-dev-1x3}
      \footlineextra{Photo: Seguinot, 2016.}
    \end{backgroundframe}

    \begin{backgroundframe}[b]{julien-xt10-161027-152047-152107-dev-2x3}{0.0}{}
      \flushleft\alphabox{The Rhône Glacier foreland}
      \footlineextra{Photo: Seguinot, 2016.}
    \end{backgroundframe}

    \begin{backgroundframe}[b]{julien-xt1-161027-154120-dev}{0.0}{}
      \flushleft\alphabox{A terminal moraine}
      \footlineextra{Photo: Seguinot, 2016.}
    \end{backgroundframe}

    \begin{backgroundframe}[b]{art/hogard-1848}{0.0}{}
      \flushleft\alphabox{Rhône Glacier in 1848}
      \footlineextra{Artwork:
        \href{http://doi.org/10.3932/ethz-a-000016730}{Hogard, 1848.}}
    \end{backgroundframe}


% -- -- -- -- -- -- -- -- -- -- -- -- -- -- -- -- -- -- -- -- -- -- -- -
\subsection{Zürich stadial}
% The Zürich Lake moraine and the Sihl River.
% -- -- -- -- -- -- -- -- -- -- -- -- -- -- -- -- -- -- -- -- -- -- -- -

    \begin{backgroundframe}[t]{art/heer-1865}{0.0}{}
      \flushleft\alphabox{Zürich during the deglaciation}
      \footlineextra{Artwork: Heer, 1865}  % FIXME url
    \end{backgroundframe}

    \begin{frame}{Zürich after the glacial retreat}
      \begin{columns}
        \column{60mm}
          \includegraphics[width=\linewidth]{web/jaggin-2008-01}  % TODO
        \column{60mm}
          \includegraphics[width=\linewidth]{web/jaggin-2008-05}  % TODO
      \end{columns}
      \bigskip
      The Sihl River breached the moraine and built a delta in the lake.
      \footlineextra{Figures:
        \href{https://www.stadt-zuerich.ch/hbd/de/index/staedtebau/archaeo_denkmal/archaeo/themen/seespiegel.html}
             {Grafiken Amt für Städtebau / Archäologie / Urs Jäggin, 2008}.}
    \end{frame}


% -- -- -- -- -- -- -- -- -- -- -- -- -- -- -- -- -- -- -- -- -- -- -- -
\subsection{Ice sheet reconstructions}
% Integrating geologic evidence to the ice-sheet scale.
% -- -- -- -- -- -- -- -- -- -- -- -- -- -- -- -- -- -- -- -- -- -- -- -

    \begin{backgroundframe}[b]{web/coutterand-2015}{0.0}{}
      \flushright\alphabox{The Alps during the Last Glacial Maximum}
      \footlineextra{Figure:
        \href{https://www.glaciers-climat.com/cg/le-quaternaire-dans-les-alpes/}
             {Coutterand, 2017}.}
    \end{backgroundframe}

    \begin{backgroundframe}[b]{photo-droker-mtlogan-js}{0.0}{}
      \flushleft\alphabox{Kluane ice field}
      \footlineextra{Photo: after
        \href{https://www.flickr.com/photos/29750062@N06/28938896344/}
             {R. Droker, 2013}.}
    \end{backgroundframe}

    \begin{frame}{Timing of the Last Glacial Maximum}
      \includegraphics[height=80mm]{bib/wirsig-etal-2016-05}
      \footlineextra{Source:
        \href{https://doi.org/10.1016/j.quascirev.2016.05.001}
             {Wirsig et al., 2016.}}
    \end{frame}


% ----------------------------------------------------------------------
\section{Paleo-glacier modelling}
% ----------------------------------------------------------------------

    \begin{sectionframe}{art/hodel-1927}{0.75}{Palaeo-glacier modelling}
      \emph{Reconstructing palaeo-glaciers using ice physics.}
      \footlineextra{Artwork: \href{http://gletschergarten.ch}{Hodel, 1927}.}
    \end{sectionframe}


% -- -- -- -- -- -- -- -- -- -- -- -- -- -- -- -- -- -- -- -- -- -- -- -
\subsection{Ice thermodynamics}
% The physical core of an ice sheet model.
% -- -- -- -- -- -- -- -- -- -- -- -- -- -- -- -- -- -- -- -- -- -- -- -

    % LATER consistent variables
    \begin{frame}<1-5>{Field equations}
      \begin{itemize}[<+->]
      \item Conservation of volume (incompressibility of flow)
        \begin{equation*}
          \div{
             \mathnode<1>{vv}{\vv}
            } = 0 
        \end{equation*}
      \item Balance of stresses (Stokes equation)
        \begin{equation*}
          \tdiv{
             \mathnode<2>{cst}{\CST}
           } + 
             \mathnode<2>{rhog}{\rho\,\vect{g}}
           = \vect{0}
        \end{equation*}
      \item Constitutive law for ice (Glen's law)
        \begin{equation*}
            \mathnode<3>{srt}{\SRT}
          = A_0\,e^\frac{-Q}{RT_{pa}}\,\tau_e^{n-1}\,
            \mathnode<3>{dst}{\DST}
          \end{equation*}
      \item Conservation of energy (heat equation)
        \begin{equation*}
            \pdv{
              \mathnode<4>{temp}{T}
            }{t}+\vect{v}\cdot\grad{}\,T =
            \mathnode<4>{diff}{\frac{k}{\rho c} \Delta T}
             +
            \mathnode<4>{hsrc}{\frac{\tr(\DST\SRT)}{\rho c}}
        \end{equation*}
      \end{itemize}
      \begin{tikzpicture}[>=latex, overlay, remember picture, myred, font=\scriptsize]
        \path<1> (vv) -- +(3, -0.25) node (vvt) {ice velocity vector};
        \path<2> (rhog) -- +(3, +0.25) node (cstt) {stress tensor};
        \path<2> (rhog) -- +(3, -0.25) node (rhogt) {gravitational force};
        \path<3> (dst) -- +(2, +0.25) node (srtt) {strain-rate tensor};
        \path<3> (dst) -- +(2, -0.25) node (dstt) {deviatoric stress tensor};
        \path<4> (temp) -- +(-2, +0.25) node (tempt) {ice temperature};
        \path<4> (hsrc) -- +(2, +0.25) node (difft) {diffusion term};
        \path<4> (hsrc) -- +(2, -0.25) node (hsrct) {source term};
        \path[draw, ->]<1> (vv) .. controls +(0,-0.5) and +(-1,0) .. (vvt.west);
        \path[draw, ->]<2> (cst) .. controls +(0,+0.5) and +(-1,0.5) .. (cstt.west);
        \path[draw, ->]<2> (rhog) .. controls +(0,-0.5) and +(-1,-0.5) .. (rhogt.west);
        \path[draw, ->]<3> (srt) .. controls +(0,+0.5) and +(-1,0.5) .. (srtt.west);
        \path[draw, ->]<3> (dst) .. controls +(0,-0.5) and +(-0.5,-0.5) .. (dstt.west);
        \path[draw, ->]<4> (temp) .. controls +(0,0.5) and +(1,0) .. (tempt.east);
        \path[draw, ->]<4> (diff) .. controls +(0,0.5) and +(-1,1) .. (difft.west);
        \path[draw, ->]<4> (hsrc) .. controls +(0,-0.5) and +(-1,-1) .. (hsrct.west);
      \end{tikzpicture}
    \end{frame}

%    % LATER fix highlighting
%    \begin{frame}{Shallow approximations of the stress balance}
%      $$\vec{\mathrm{div}} \, \bm\sigma + \rho \, \vec{g} = 0
%        \qquad\Rightarrow\qquad\left\{\begin{array}{l}
%        \alert<3-4>{\frac{\partial\tau_{xx}}{\partial x}}
%        \alert<3-4>{+\frac{\partial\tau_{xy}}{\partial y}}
%        \alert<2-4>{+\frac{\partial\tau_{xz}}{\partial z}}
%        \alert<2-4>{=\frac{\partial p}{\partial x}}\\
%        \alert<3-4>{\frac{\partial\tau_{yx}}{\partial x}}
%        \alert<3-4>{+\frac{\partial\tau_{yy}}{\partial y}}
%        \alert<2-4>{+\frac{\partial\tau_{yz}}{\partial z}}
%        \alert<2-4>{=\frac{\partial p}{\partial y}}\\
%        \alert<4-4>{\frac{\partial\tau_{zx}}{\partial x}}
%        \alert<4-4>{+\frac{\partial\tau_{zy}}{\partial y}}
%        \alert<4-4>{+\frac{\partial\tau_{zz}}{\partial z}}
%        \alert<2-4>{=\frac{\partial p}{\partial z} - \rho g}
%        \end{array}\right.$$
%      \begin{itemize}[<+(1)-| alert@+(1)>]
%        \item Shallow ice approximation
%        \item Shallow shelf approximation
%        \item Full stokes
%        \item<+(1)-> And several others...
%      \end{itemize}
%      \pause
%      Stress balance approximations often define the scope of a glacier model.
%    \end{frame}

    \begin{frame}{Shallow approximations}
      % TikZ styles
\tikzstyle{vsia}=[myblue, thick]
\tikzstyle{vssa}=[myred, thick]

\begin{tikzpicture}[>=latex]

% white background
\fill[white] (-1.5,-0.5) rectangle +(11.0,6.5);
%\draw [help lines, lightgray] (0,0) grid (8.0,5.5);

% transition lines and calving front coordinates
\coordinate (tr1) at (2.75, 0);  % transition one
\coordinate (tr2) at (5.25, 0);  % transition two
\coordinate (cf) at (7.75, 2);
\draw[lightgray, name path=tr1] (tr1 |- 0,0) -- +(0,5.5) ;
\draw[lightgray, name path=tr2] (tr2 |- 0,0) -- +(0,5.5) ;

% location of the velocity profiles
\path [name path=x1] (1.0,0) -- +(0,5) ;
\path [name path=x2] (3.0,0) -- +(0,5) ;
\path [name path=x3] (5.5,0) -- +(0,5) ;

% bedrock topography
\draw [name path=bed]
    (0,2) .. controls +(-5:4) and +(180:2) .. (8,1);
\fill [pattern=north east lines]
    (0,2) .. controls +(-5:4) and +(180:2) .. (8,1)
          -- +(0,-0.25)
          .. controls +(180:2) and +(-5:4) .. (0,1.75);

% locate the grounding line
\path [name intersections={of=bed and tr2, by=gl}] ;

% surface topograpy
\draw [name path=surf]
    (0,4.5) .. controls +(-10:4) and +(180:3) .. ($(cf)+(0,0.1)$)
          -- (cf) ;

% ice shelf base
\draw [name path=base]
    (gl) .. controls +(15:0.5) and +(180:1) .. ($(cf)-(0,0.4)$)
         -- (cf);

% sea level5
\draw [name path=sl] (cf) -- (cf -| 8,0) ;

% first velocity profile
\only<2->{
  \path [name intersections={of=surf and x1, by=s1}] ;
  \path [name intersections={of=bed and x1, by=b1}] ;
  \draw[vsia] (b1) -- (s1);
  \draw[vsia, name path=vsia]
    (b1) .. controls ++(0.75,0.5) .. ($(s1)+(0.75,0)$);
  \path[name path=vgrid] (s1) foreach \x in {1,...,5} { -- +(2,0) ++(0,-0.5) };
  \path[name intersections={of=x1 and vgrid, name=a, total=\t}];
  \path[name intersections={of=vsia and vgrid, name=b, total=\t}];
  \draw[vsia, ->] (a-1) -- (b-1);
  \draw[vsia, ->] (a-2) -- (b-2);
  \draw[vsia, ->] (a-3) -- (b-3) node [midway, above] {$\vsia$};
  \draw[vsia, ->] (a-4) -- (b-4);
  \draw[vsia, ->] (a-5) -- (b-5);
}

% second velocity profile
\only<4->{
  \path [name intersections={of=surf and x2, by=s2}] ;
  \path [name intersections={of=bed and x2, by=b2}] ;
  \draw[vssa] (b2) -- (s2);
  \draw[vssa, name path=vssa]
    (b2) -- ($(b2)+(1,0)$) -- ($(s2)+(1,0)$);
  \draw[vsia, name path=vsia]
    ($(b2)+(1,0)$) .. controls ++(1,0.5) .. ($(s2)+(2,0)$);
  \path[name path=vgrid] (s2) foreach \x in {1,...,5} { -- +(2,0) ++(0,-0.5) };
  \path[name intersections={of=x2 and vgrid, name=a, total=\t}];
  \path[name intersections={of=vssa and vgrid, name=b, total=\t}];
  \path[name intersections={of=vsia and vgrid, name=c, total=\t}];
  \draw[vssa, ->] (a-1) -- (b-1);
  \draw[vssa, ->] (a-2) -- (b-2);
  \draw[vssa, ->] (a-3) -- (b-3) node [midway, above] {$\vssa$};
  \draw[vssa, ->] (a-4) -- (b-4);
  \draw[vsia, ->] (b-1) -- (c-1);
  \draw[vsia, ->] (b-2) -- (c-2);
  \draw[vsia, ->] (b-3) -- (c-3) node [midway, above] {$\vsia$};
  \draw[vsia, ->] (b-4) -- (c-4);
}

% third velocity profile
\only<3->{
  \path [name intersections={of=surf and x3, by=s3}] ;
  \path [name intersections={of=base and x3, by=b3}] ;
  \draw[vssa] (b3) -- (s3);
  \draw[vssa, name path=vssa]
    (b3) -- ($(b3)+(2,0)$) -- ($(s3)+(2,0)$);
  \path[name path=vgrid] (s3) foreach \x in {1,...,3} { -- +(2,0) ++(0,-0.5) };
  \path[name intersections={of=x3 and vgrid, name=a, total=\t}];
  \path[name intersections={of=vssa and vgrid, name=b, total=\t}];
  \draw[vssa, ->] (a-1) -- (b-1);
  \draw[vssa, ->] (a-2) -- (b-2);
  \draw[vssa, ->] (a-3) -- (b-3) node [midway, above] {$\vssa$};
}

% add transition lines and annotations
\coordinate (m1) at ($(0,0)!0.5!(tr1)$) ;
\coordinate (m2) at ($(tr1)!0.5!(gl)$) ;
\coordinate (m3) at ($(gl)!0.5!(8,0)$) ;
\coordinate (top1) at ($(0,5.25)$) ;
\coordinate (top2) at ($(0,4.75)$) ;
\coordinate (bot) at ($(0,0.25)$) ;

\node<2-> at (m1|-top1) {ice sheet};
\node<4-> at (m2|-top1) {ice stream};
\node<3-> at (m3|-top1) {ice shelf};
\node<2-> [vsia] at (m1|-top2) {Shallow Ice Approximation};
\node<3-> [vssa] at (m3|-top2) {Shallow Shelf Approximation};
\node<2-> at (m1|-bot) {$\vsia\gg\vssa$};
\node<4-> at (m2|-bot) {$\vsia\sim\vssa$};
\node<3-> at (m3|-bot) {$\vsia\ll\vssa$};

\end{tikzpicture}

      \footlineextra{After: Winkelmann et al., 2011.}  % FIXME url
    \end{frame}


% -- -- -- -- -- -- -- -- -- -- -- -- -- -- -- -- -- -- -- -- -- -- -- -
\subsection{Boundary conditions}
% The atmosphere, the bedrock, and the ocean.
% -- -- -- -- -- -- -- -- -- -- -- -- -- -- -- -- -- -- -- -- -- -- -- -

    % FIXME maybe improve the text
    \begin{frame}[label=model-interfaces]{Parallel ice sheet model}
      % TikZ styles
\alt<2>{\tikzstyle{bedflux}=[myred, thick]}{\tikzstyle{bedflux}=[]}
\alt<3>{\tikzstyle{atmflux}=[myblue, thick]}{\tikzstyle{atmflux}=[]}

\begin{tikzpicture}[>=latex]

% white background
\fill[white] (0,0) rectangle +(8,4);
%\draw [help lines, lightgray] (0,0) grid (8.0,4.0);

% grounding line and inland margin coordinates
\coordinate (cf) at (0.5, 1.75) ;
\path[name path=xgl] (2.5,0) -- +(0,4) ;
\path[name path=xim] (7,0) -- +(0,4) ;
\path[name path=xfl] (4.5,0) -- +(0,4) ;

% bedrock topography
\draw [bedflux, name path=bed]
    (0,0.25) .. controls +(0:3) and +(-5:-2.5) ..  (5,1.75)
	  .. controls +(-5:1) and +(10:-1) .. (8,2);
\fill [bedflux, pattern=north east lines]
    (0,0.25) .. controls +(0:3) and +(-5:-2.5) ..  (5,1.75)
	  .. controls +(-5:1) and +(10:-1) .. (8,2)
	  -- +(0,-0.25)
          .. controls +(10:-1) and +(-5:1) .. (5,1.5)
          .. controls +(-5:-2.5) and +(0:3) .. (0,0);

% locate the grounding line
\path [name intersections={of=bed and xgl, by=gl}] ;
\path [name intersections={of=bed and xim, by=im}] ;

% surface topograpy
\draw [atmflux, name path=surf]
    (cf) -- +(0,0.05)
         .. controls +(0:2.75) and +(0:-2) .. (4.5, 3.5)
         .. controls +(0:1) and +(-60:-1.5) .. (im);
\draw [atmflux, dashed]
    (cf) .. controls +(0:2.75) and +(0:-2) .. (4.5, 3.3)
         .. controls +(0:1) and +(-60:-1.5) .. ($(im)-(0:0.1)$);

% ice shelf base
\draw [name path=base]
    (cf) -- +(0,-0.2)
         .. controls +(0:1.5) and +(135:0.5) .. (gl) ;
\draw [dashed]
    ($(cf)+(0,-0.15)$)
         .. controls +(0:1.5) and +(135:0.5) .. ($(gl)+(45:0.1)$) ;

% sea level
\draw [name path=sl] (cf -| 0,0) -- (cf) ;

% text labels
\node[atmflux] (atm) at (1.5, 3) {atmosphere};
\node (ice) at (4.5, 2.25) {ice sheet};
\node (ocn) at (1, 1) {ocean};
\node[bedflux] (bed) at (4.5, 0.5) {bedrock};
\node (pism) at (6.5, 1) {PISM};
\draw[->] (pism) -- (ice);
\draw[->] (pism) -- (bed);

% fluxes
\coordinate [name intersections={of=bed and xfl, by=bfl}] ;
\coordinate [name intersections={of=surf and xfl, by=afl}] ;
\draw<2>[->, bedflux] (bfl) ++(-0.1,0.25) -- ++(0,-0.5) ;
\draw<2>[->, bedflux] (bfl) ++(0.1,-0.25) -- ++(0,+0.5) ;
\draw<3>[->, atmflux] (afl) ++(-0.1,0.25) -- ++(0,-0.5) ;
\draw<3>[->, atmflux] (afl) ++(0.1,-0.25) -- ++(0,+0.5) ;

\end{tikzpicture}
\\
      \bigskip
         \begin{itemize}
           \only<1>{
             \item Shallow Shelf Approximation on pseudo-pastic till
             \item Polythermal Shallow Ice Approximation}
           \only<2>{
             \item Viscous-modulated elastic lithosphere
             \item Bedrock temperature model to 3\,km depth}
           \only<3>{
             \item Snow precipitation before 0--2 $^\circ$C
             \item Weekly resolved positive Degree Day melt model}
          \end{itemize}
      \footlineextra{After: \href{http://pism_docs.org}{PISM docs, 2014}.}
    \end{frame}

%    \begin{frame}{Surface accumulation (snowfall)}
%      \begin{tikzpicture}[>=latex]

% white background
\fill[white] (0,0) rectangle +(8,4);
%\draw[help lines, lightgray] (0,0) grid +(8,4);

\coordinate (o) at (3,1);
\coordinate (tsnow) at ($(o)+(0,2)$);
\coordinate (train) at ($(o)+(2,0)$);

\draw[->] (0.25,|-o) -- +(7.0,0) node[above] {$T$ (\degC)};
\draw[->] (o|-,0.25) -- +(0,3.5) node[right] {snow fraction};
\draw[myblue, thick] (0.25,|-tsnow) -- (tsnow) -- (train) -- (7,|-train) ;

\node [below left] at (o) {0};
\node [below left] at (tsnow) {1};
\node [below] at (train) {2};

\end{tikzpicture}
\\
%      \begin{itemize}
%        \item equal to precipitation when temperature is below 0\degC
%        \item decreases to zero linearly with temperature between 0 and 2\degC
%      \end{itemize}
%    \end{frame}
%
%    \begin{frame}{Surface ablation (melt)}
%      \begin{columns}
%        \column{60mm}
%         \begin{itemize}
%           \item<1-> proportional to the number of positive degree-days
%             $$ \mathrm{PDD} = \int_{t_1}^{t_2} T^{+} \dd{t} $$
%           \item<3-> includes daily temperature variability
%             $$ T^{+} = \int_{0}^{\infty} T_{ac}
%                         \, e^{-\frac{(T-T_{ac})^2}{2\sigma^2}} \dd{T} $$
%          \end{itemize}
%        \column{60mm}
%          \only<1-3>{\alt<3>{\tikzstyle{sd}=[draw, thick, myred]}{\tikzstyle{sd}=[]}

\begin{tikzpicture}[>=latex]

% white background
\fill[white] (0,0) rectangle +(6,6.5);
%\draw[help lines, lightgray] (0,0) grid +(8,4);

% melt plot
\only<1->{
  \coordinate (o) at (3,4.75);
  \coordinate (p1) at ($(o)-(2.75,0)$);
  \coordinate (p2) at ($(o)+(2.75,1.25)$);
  \draw[->] (p1) -- +(5.5,0) node[above left] {$T$ (\degC)};
  \draw[->] (o) -- +(0,1.5) node[right] {$M$};
  \node[below] at (o) {0};
  \draw (p1) -- (o) -- (p2) ;
  \path[sd, dashed] (p1)
    .. controls ($(p1)!0.75!(o)$) and ($(p2)!0.75!(o)$) .. (p2) ;
}

% seasonal cycle plot
\only<2->{
  \coordinate (o) at (0.25,0.75);
  \coordinate (p1) at ($(o)+(0,1.5)$);
  \coordinate (p2) at ($(o)+(2.5,3)$);
  \coordinate (p3) at ($(o)+(5,1.5)$);
  \draw[->] (o) -- +(5.5,0) node[above left] {$t$ (yr)};
  \draw[->] (o) -- +(0,3.5) node[right] {$T$};
  \node [below] at (o) {0};
  \node [below] at (p3|-o) {1};
  \draw (p1)  .. controls +(1,0) and +(-1,0) .. (p2)
              .. controls +(1,0) and +(-1,0) .. (p3) ;
  \path[sd, dashed] ($(p1)+(0,-1)$)
              .. controls +(1,0) and +(-1,0) .. ($(p2)+(0,-0.25)$)
              .. controls +(1,0) and +(-1,0) .. ($(p3)+(0,-1)$) ;
  \path[sd, dashed] ($(p1)+(0,1)$)
              .. controls +(1,0) and +(-1,0) .. ($(p2)+(0,0.25)$)
              .. controls +(1,0) and +(-1,0) .. ($(p3)+(0,1)$) ;
  \path<3>[<->, sd] (p3) -- ($(p3)+(0,1)$) node [midway, right] {$\sigma$};
}

\end{tikzpicture}

}
%      \end{columns}
%    \end{frame}


% ----------------------------------------------------------------------
\subsection{Example application}
% Modelling the Alpine paleo-glaciers.
% ----------------------------------------------------------------------

%    \begin{sectionframe}{art/hodel-1927}{0.75}{Open questions}
%      \begin{enumerate}
%        \item what climate \& glacial history lead to the maximum ice \alert{extent},
%        \item to which extent ice \alert{flow} was controlled by subglacial topography,
%        \item what caused differences in \alert{timing} of the Last Glacial Maximum,
%        \item how far above the trimline was the ice \alert{surface} located, and
%      \end{enumerate}
%      \bigskip\bigskip\bigskip\pause
%      Tool: \alert{Parallel} Ice Sheet Model (PISM)\\
%      \bigskip
%      % {\small(3D energy balance, polythermal SIA, pseudo-plastic till SSA,\\
%      %         PDD mass balance, viscous-modulated bedrock deformation)}\\
%      \bigskip\pause
%      Method: High-resolution simulation of the \alert{last glacial cycle}\\
%      \bigskip
%      {\small(120--0\,ka, 1x1\,km x 20\,m, 576 processors, 33 days)}\\
%    \end{sectionframe}

%    \begin{frame}{Present-based spatial inputs}
%      \includegraphics[height=80mm]{alpcyc_hr_inputs}
%      \footlineextra{Data: WorldClim; ERA-Interim; Goutorbe et al., 2011.}  % FIXME url
%    \end{frame}
%
%    \begin{frame}{Simulation of the last glacial cycle (120--0\,kyr)}
%      \begin{columns}
%        \column{80mm}
%          \begin{itemize}
%%            \item<+-> Spatial \alert{climate patterns} from today
%%              \begin{itemize}
%%                \item monthly temperature from WorldClim
%%                \item monthly precipitation from WorldClim
%%                \item monthly temp. stdev. from ERA-Interim
%%              \end{itemize}
%%            \bigskip
%            \item<+-> Palaeo-climate proxy records
%              \begin{itemize}
%                \item Continuous 120--0\,ka
%                \item High resolution $\approx$1\,ka
%                \item \sout<2->{Not far from the Alps}
%              \end{itemize}
%            \bigskip
%            \item<+-> Time-dependent \alert{temperature} change from
%              \begin{itemize}
%                \item \textbbf{GRIP} Greenland ice \chem{\delta^{18}O}
%                \item \textrbf{EPICA} Antarctic ice \chem{\delta^{18}O}
%                \item \textgbf{MD01-2444} sediment \chem{U^{K'}_{37}}
%                \item linearly scaled to LGM ice extent
%              \end{itemize}
%            \bigskip
%            \item<+-> Two \alert{precipitation} parametrizations
%              \begin{itemize}
%                \item Constant in time
%                \item Decreased with temperature
%              \end{itemize}
%          \end{itemize}
%        \column{40mm}
%          \uncover<2->{\includegraphics[height=80mm]{alpcyc_lr_records}}
%      \end{columns}
%    \end{frame}
%
%    \begin{frame}{Temperature forcing \only<2->{and modelled ice volume}}
%      \begin{tikzpicture}
%        \node[inner sep=0] (fig)
%          {\includegraphics<-2|handout:0>[width=\linewidth]{alpcyc_lr_timeseries_nopp}%
%           \includegraphics<3->[width=\linewidth]{alpcyc_lr_timeseries}};
%        \figvmask<-1|handout:0>{0.54}
%        \figvmask<2-|handout:0>{0.0}
%      \end{tikzpicture}\\
%      \uncover<2->{
%        \only<-2>{Ice volume fluctuations are \alert{rapid},
%                 but smaller with \textrbf{EPICA} forcing.}
%        \only<3->{These fluctuations are \alert{smoothed},
%                 in \alert{reduced} precipitation runs.}}
%      \footlineextra{Data: Dansgaard et al., 1993; Jouzel et al., 2007;
%                           Martrat et al., 2007}  % FIXME url
%    \end{frame}
%
%    \begin{frame}{Glaciated areas during MIS 2 \only<2->{and 4}}
%      \begin{tikzpicture}
%        \node[inner sep=0] (fig)
%          {\includegraphics<-2|handout:0>[width=\linewidth]{alpcyc_lr_footprints_nopp}%
%           \includegraphics<3->[width=\linewidth]{alpcyc_lr_footprints}};
%        \figvmask<-1|handout:0>{0.47}
%        \figvmask<2-|handout:0>{0.0}
%      \end{tikzpicture}\\
%      \uncover<2->{
%        \only<-2>{Using constant precipitation, only \textrbf{EPICA}
%                  yield realistic \alert{MIS 4}.}
%        \only<3->{\textbbf{GRIP} and \textgbf{MD01-2444} give
%                  better results with \alert{reduced} precipitation but still too big.}}
%    \end{frame}

    \begin{frame}{Modelled glacier dynamics of the last glacial cycle}
      \includegraphics[height=75mm]{vimeo-alps-4k-zo-ja}\\
%      最終氷期におけるアルプスの氷河の前進と後退
      \footlineextra{Video:
        \href{https://vimeo.com/300927091}{Seguinot, 2019}.}  % EN: 294517816
    \end{frame}

%    \begin{frame}{Ice extent and flow pattern at 24.6 ka}
%      \includegraphics[height=80mm]{alpcyc_hr_maxextvel}\\
%      Ice \alert{flow} mostly along topography but not always.
%    \end{frame}
%
%    \begin{frame}{Comparison with observed trimline elevations}
%      \begin{tikzpicture}
%        \node[inner sep=0] (fig)
%          {\includegraphics[width=\linewidth]{alpcyc_hr_trimlines}};
%        \fighmask<-1|handout:0>{0.50}
%        \fighmask<2-|handout:0>{0.0}
%      \end{tikzpicture}\\
%      \uncover<2->{Modelled ice \alert{surface} is 861 m above observed trimlines.}
%      \footlineextra{Data: Kelly et al., 2004.}  % FIXME url
%    \end{frame}
%
%    \begin{frame}{Timing of the Last Glacial Maximum}
%      \includegraphics[height=80mm]{alpcyc_hr_maxthkage}\\
%      Last Glacial Maximum \alert{timing} varies across the range.
%    \end{frame}
%
%    \begin{frame}{Multiple advances and retreats onto the foreland}
%      \includegraphics[height=80mm]{alpcyc_hr_profiles_four}\\
%      Individual glacier \alert{dynamics} are controlled by catchment hypsometries.
%    \end{frame}

    \begin{sectionframe}{art/hodel-1927}{0.75}{Conclusions}
      \begin{itemize}
        \item Glacial geology allows reconstruction of palaeo-glaciers\\
          \emph{\small but the record is sparse in time and space.}
        \pause\bigskip
        \item Numerical modelling allows to fill the gaps\\
          \emph{\small but uncertainties are large.}
      \end{itemize}
      \footlineextra{Artwork: \href{http://gletschergarten.ch}{Hodel, 1927}.}
    \end{sectionframe}

    \begin{backgroundframe}[b]{art/fearnley-1838}{0.0}{}
      \flushleft\alphabox{Lower Grindelwald Glacier in 1838}
      \footlineextra{Artwork:
        \href{https://commons.wikimedia.org/wiki/File:Thomas_Fearnley_-_Grindelwaldgletscher_-_Google_Art_Project.jpg}
             {Fearnley, 1838}.}
    \end{backgroundframe}


% ======================================================================
\end{document}
% ======================================================================
