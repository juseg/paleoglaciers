% GLACIER MODELLING
% =================

% Beamer presentation header
% ======================================================================

\documentclass[aspectratio=1610]{beamer}

% ----------------------------------------------------------------------
% Packages
% ----------------------------------------------------------------------

% encoding
\usepackage[utf8]{inputenc}
\usepackage[english]{babel}
\usepackage{fontawesome}

% math
\usepackage{bm}
\usepackage{physics}
%\usepackage{amsmath}

% graphics
\graphicspath{{figures/}}

% tikz
\usepackage{tikz}
\usetikzlibrary{intersections}
\usetikzlibrary{calc}
\usetikzlibrary{patterns}
%\everymath{\displaystyle}

% ----------------------------------------------------------------------
% Beamer themes
% ----------------------------------------------------------------------

\useinnertheme{rectangles}
\useoutertheme{default}
\usecolortheme{dove}
\setbeamertemplate{navigation symbols}{}
\setbeamertemplate{footline}[frame number]

% ----------------------------------------------------------------------
% Colors
% ----------------------------------------------------------------------

% Color brewer paired CMYK
\definecolor{lightblue}  {rgb}{166, 206, 227}  % qual_Paired_12_01
\definecolor{darkblue}   {rgb}{ 31, 120, 180}  % qual_Paired_12_02
\definecolor{lightgreen} {rgb}{178, 223, 138}  % qual_Paired_12_03
\definecolor{darkgreen}  {rgb}{ 51, 160,  44}  % qual_Paired_12_04
\definecolor{lightred}   {rgb}{251, 154, 153}  % qual_Paired_12_05
\definecolor{darkred}    {rgb}{227,  26,  28}  % qual_Paired_12_06
\definecolor{lightorange}{rgb}{253, 191, 111}  % qual_Paired_12_07
\definecolor{darkorange} {rgb}{255, 127,   0}  % qual_Paired_12_08
\definecolor{lightpurple}{rgb}{202, 178, 214}  % qual_Paired_12_09
\definecolor{darkpurple} {rgb}{106,  61, 154}  % qual_Paired_12_10
\definecolor{lightbrown} {rgb}{255, 255, 153}  % qual_Paired_12_11
\definecolor{darkbrown}  {rgb}{177,  89,  40}  % qual_Paired_12_12

% Color shortcuts 
\definecolor{myblue}     {RGB}{ 31, 120, 180}  % qual_Paired_12_02
\definecolor{mygreen}    {RGB}{ 51, 160,  44}  % qual_Paired_12_04
\definecolor{myred}      {RGB}{227,  26,  28}  % qual_Paired_12_06

% Beamer colors
%\setbeamercolor{alerted text}{fg=myblue} 
\setbeamercolor{titlelike}{parent=palette primary}
\setbeamercolor{frametitle}{parent=palette secondary}
\setbeamercolor{block}{parent=palette secondary}
\newcommand{\textrbf}[1]{\textcolor{myred}{\textbf{#1}}}
\newcommand{\textgbf}[1]{\textcolor{mygreen}{\textbf{#1}}}
\newcommand{\textbbf}[1]{\textcolor{myblue}{\textbf{#1}}}

% Transparent blocks
\addtobeamertemplate{block begin}{\pgfsetfillopacity{0.75}}{\pgfsetfillopacity{1}}
\addtobeamertemplate{block alerted begin}{\pgfsetfillopacity{0.75}}{\pgfsetfillopacity{1}}
\addtobeamertemplate{block example begin}{\pgfsetfillopacity{0.75}}{\pgfsetfillopacity{1}}

% ----------------------------------------------------------------------
% Math shortcuts
% ----------------------------------------------------------------------

% Vectors and tensors
\newcommand{\vect}[1]{\va*{#1}} % bold arrow vectors
\newcommand{\tens}[1]{\vb*{#1}} % bold tensors

% Differential operators
\renewcommand{\div}[1]{\mathrm{div}\,#1}            % divergence
\renewcommand{\grad}[1]{\vect{\mathrm{grad}}\,#1}   % gradient
\newcommand{\tdiv}[1]{\vect{\mathrm{div}}\,#1}      % tensor divergence
\newcommand{\tgrad}[1]{\tens{\mathrm{grad}}\,#1}    % tensor gradient
\newcommand{\matdv}[1]{\pdv{#1}{t}+\vect{v}\cdot\grad{}\,#1}  % material dv.

% Common notations
\newcommand{\doteps}[0]{\dot{\epsilon}} % epsilon dot
\newcommand{\IDT}[0]{\tens{\delta}}     % Identity tensor
\newcommand{\CST}[0]{\tens{\sigma}}     % Cauchy stress tensor
\newcommand{\DST}[0]{\tens{\tau}}       % deviatoric stress tensor
\newcommand{\SRT}[0]{\tens{\doteps}}    % strain-rate tensor
\newcommand{\vv}[0]{\vect{v}}           % velocity vector
\newcommand{\vsia}[0]{\vv_{\mathrm{SIA}}}   % SIA velocity
\newcommand{\vssa}[0]{\vv_{\mathrm{SSA}}}   % SSA velocity
\newcommand{\PDD}[0]{\mathrm{PDD}}
\newcommand{\sPDD}[0]{\sigma_{\mathrm{PDD}}}

% Common units
\newcommand{\e}[1]{\ensuremath{\times 10^{#1}}}
\newcommand{\chem}[1]{\ensuremath{\mathsf{#1}}}
\newcommand{\unit}[1]{\ensuremath{\mathsf{#1}}}
\newcommand{\degree}[0]{\ensuremath{^{\circ}}}
\newcommand{\degC}[0]{\unit{{\degree}C}}

% TikZ invisible math node
\newcommand<>{\mathnode}[2]{%
  \alt#3{\tikz[baseline, remember picture]%
         \node[anchor=base, inner sep=0pt, myred](#1){$#2$};}%
        {#2}}

% TikZ left-aligned semi-transparent text box
\newcommand{\alphabox}[1]{%
  \flushleft\tikz\node[fill=white, fill opacity=0.75, text opacity=1,%
                       align=left, inner sep=1em] {#1};}

% strike-out
\usepackage[normalem]{ulem}
\renewcommand<>{\sout}[1]{\only#2{\beameroriginal{\sout}}{#1}}


% ----------------------------------------------------------------------
% Special frames
% ----------------------------------------------------------------------

\newlength{\imgwidth}

\newenvironment{backgroundframe}[4][c]% options, image, alpha, title
  {% check width of scale image
   \settowidth{\imgwidth}{\includegraphics[height=\paperheight]{#2}}%
   % add image as background canvas
   \setbeamertemplate{background canvas}{%
    \tikz[overlay, remember picture]%
      \node[opacity=1.0] at (current page.center) {%
        \ifdim\imgwidth<\paperwidth
          \includegraphics[width=\paperwidth]{#2}
        \else
          \includegraphics[height=\paperheight]{#2}
        \fi
      };}%
   % add semi-transparent rectangle overlay
   \setbeamertemplate{background}{%
    \tikz\node[fill=white, inner sep=0, opacity=#3,%
               text width=\paperwidth, text height=\paperheight]{};}%
   \begin{frame}[#1]{#4}}
  {\end{frame}}


% section frame with custom image
%\newenvironment{sectionframe}[4][]% options, image, alpha, title
%  {\begin{backgroundframe}{#2}{#3}{}\centering{\huge #4}\\\bigskip}
%  {\end{backgroundframe}}
\newenvironment{sectionframe}[2][]% options, title
  {\begin{backgroundframe}{art_hodel_1927_rigi}{0.75}{}\centering{\huge #2}\\\bigskip}
  {\footlineextra{Background: Hodel, 1927.}\end{backgroundframe}}

% coverfig source: http://tex.stackexchange.com/a/101073
\newcommand<>{\coverfig}[1]{
  \coordinate (zone) at ($(fig.south)!#1!(fig.north)$);
  \fill#2[white, opacity=1.0] (fig.south west) rectangle(fig.east|-zone);
}

%% covergig modified
%\newcommand<>{\coverfig}[2]{\begin{tikzpicture}
%  \node[inner sep=0] (fig) {#1};
%  \coordinate (zone) at ($(fig.south)!#2!(fig.north)$);
%  \fill<1>[black, opacity=0.75] (fig.south west) rectangle (fig.east|-zone);
%\end{tikzpicture}}

% ----------------------------------------------------------------------
% Footline macro
% From http://tex.stackexchange.com/questions/5491/
%      how-do-i-insert-text-into-the-footline-of-a-specific-slide-in-beamer
% ----------------------------------------------------------------------

\makeatletter

% add a macro that saves its argument
\newcommand{\footlineextra}[1]{\gdef\insertfootlineextra{#1}}
%\newbox\footlineextrabox

% add a beamer template that sets the saved argument in a box.
% The * means that the beamer font and color "footline extra" are automatically added. 
\defbeamertemplate*{footline extra}{default}{
    %\begin{beamercolorbox}[ht=2.25ex,dp=1ex,leftskip=\Gm@lmargin]{footline extra}
    \hspace{\Gm@lmargin}\insertfootlineextra
    %\par\vspace{2.5pt}
    %\end{beamercolorbox}
}

\addtobeamertemplate{footline}{%
    % set the box with the extra footline material but make it add no vertical space
    %\setbox\footlineextrabox=\vbox{\usebeamertemplate*{footline extra}}
    %\vskip -\ht\footlineextrabox
    %\vskip -\dp\footlineextrabox
    %\box\footlineextrabox%
    \usebeamertemplate*{footline extra}
}

% alternatively override the footline template
%\setbeamertemplate{footline}{\insertfootlineextra\hfill\insertframenumber\,/\,\inserttotalframenumber}}

% patch \begin{frame} to reset the footline extra material
\let\beamer@original@frame=\frame
\def\frame{\gdef\insertfootlineextra{}\beamer@original@frame}
\footlineextra{}
\makeatother

\setbeamercolor{footline extra}{fg=structure.fg}% for instance



\title[Glacier modelling]{Glacier modelling}
\author[seguinot@vaw.baug.ethz.ch]{Julien Seguinot}
\institute[ETHZ]{ETH Zürich, Switzerland}
\date{April 7, 2016}

% ======================================================================
\begin{document}
% ======================================================================

    \begin{backgroundframe}{photo_gries_icefall}{0.75}{}
      \maketitle
    \end{backgroundframe}

    \begin{backgroundframe}{photo_gries_icefall}{0.75}{Outline}
      \tableofcontents[hideallsubsections]
    \end{backgroundframe}

% ----------------------------------------------------------------------
\section{Introduction}
% ----------------------------------------------------------------------

% ======================================================================
% Introduction to glaciers and ice sheets
% ======================================================================

    \begin{sectionframe}{photo_gries_icefall}{0.75}{Introduction}
      \emph{Glaciers and ice sheets}
    \end{sectionframe}

    \begin{frame}{Past and present glaciers}
      \centering
      \includegraphics[height=60mm]{plot_worldmap}
      \footlineextra{Data: Ehlers and Gibbard, 2004.}
    \end{frame}

% ----------------------------------------------------------------------
\subsection{Alpine glaciers}
% ----------------------------------------------------------------------

% Overview of the Alps

    \begin{backgroundframe}[t]{wikipedia_bluemarble_europe}{0.0}{}
      \vspace{1em}\hfill
      \begin{beamercolorbox}[sep=1em,wd=40mm]{titlelike}
        Europe in summer
      \end{beamercolorbox}
      \footlineextra{Source: NASA Blue Marble}
    \end{backgroundframe}

    \begin{backgroundframe}{photo_zurich_alps}{0.0}{}
      \vspace{6cm}
      \begin{beamercolorbox}[sep=1em,wd=45mm]{titlelike}
        Zürich, 17 Sep. 2015
      \end{beamercolorbox}
    \end{backgroundframe}

% Aletsch Glacier

    \begin{backgroundframe}{nasa_aletsch_iss}{0.0}{}
      \hfill
      \begin{beamercolorbox}[sep=1em,wd=45mm]{titlelike}
        Aletsch Glacier, 2006
      \end{beamercolorbox}
      \vspace{6cm}
      \footlineextra{Photo: NASA, 2006.}
    \end{backgroundframe}

    \begin{frame}{}
      \centering
      Aletsch Glacier timelapse\\
      \bigskip
      \url{https://www.youtube.com/watch?v=v9BLoSSFT4c}\\
      \url{https://www.youtube.com/watch?v=WBub90vo4Qc}
    \end{frame}

    \begin{frame}{Glaciers flow under their own weight}
      \centering
      \includegraphics[height=60mm]{usgs_cartoon_glacier}
    \end{frame}

% Rhone glacier

    \begin{backgroundframe}{wikipedia_rhone_aerial}{0.0}{}
      \vspace{6cm}\hfill
      \begin{beamercolorbox}[sep=1em,wd=45mm]{titlelike}
        Rhone Glacier, 2011
      \end{beamercolorbox}
      \footlineextra{Photo: Pegasus2, 2011.}
    \end{backgroundframe}

    \begin{backgroundframe}{photo_rhone_terminus}{0.0}{}
      \vspace{6cm}\hfill
      \begin{beamercolorbox}[sep=1em,wd=45mm]{titlelike}
        Proglacial lake, 2015
      \end{beamercolorbox}
    \end{backgroundframe}

    \begin{backgroundframe}{glaciersonline_rhone_2008}{0.0}{}
      \vspace{6cm}\hfill
      \begin{beamercolorbox}[sep=1em,wd=45mm]{titlelike}
        Rhone Glacier, 2008
      \end{beamercolorbox}
      \footlineextra{Source: Glaciers online.}
    \end{backgroundframe}

    \begin{backgroundframe}{glaciersonline_rhone_1900}{0.0}{}
      \vspace{6cm}\hfill
      \begin{beamercolorbox}[sep=1em,wd=45mm]{titlelike}
        Rhone Glacier, 1900
      \end{beamercolorbox}
      \footlineextra{Source: Glaciers online.}
    \end{backgroundframe}

% Swiss glacier plots

    \begin{frame}{Swiss glacier length changes}
      \centering
      \includegraphics[height=60mm]{glamos_lenght_change_cum}
      \footlineextra{Source: GLAMOS.}
    \end{frame}

    \begin{frame}{Proportion of retreating glaciers in Switzerland}
      \centering
      \includegraphics[height=60mm]{glamos_lenght_change_stat}
      \footlineextra{Source: GLAMOS.}
    \end{frame}

% ----------------------------------------------------------------------
\subsection{Polar ice sheets}
% ----------------------------------------------------------------------

    \begin{frame}{The two polar ice sheets}
      \centering
      \includegraphics[width=\textwidth]{nasa_bluemarble_icesheets}\\
      Greenland + Antarctica = 65.6 m potential sea level rise
      \footlineextra{Source: NASA Blue Marble; Bamber et al., 2013; Fretwell et al., 2013.}
    \end{frame}

    \begin{frame}{Ice sheets flow under their own weight}
      \centering
      \includegraphics[height=60mm]{wikipedia_cartoon_icesheet}
      \footlineextra{Source: NASA.}
    \end{frame}

    \begin{frame}{Past and present glaciers}
      \centering
      \includegraphics[height=60mm]{plot_worldmap}
      \footlineextra{Data: Ehlers and Gibbard, 2004.}
    \end{frame}

    \begin{frame}{Antarctic and Greenland surface mass balance}
      \centering
      \includegraphics[height=60mm]{vandenbroeke_etal_2011_fig01}
      \footlineextra{Source: van den Broeke et al., 2011.}
    \end{frame}

    \begin{frame}{Antarctic and Greenland surface velocities}
      \centering
      \includegraphics[height=60mm]{rignot_etal_2011_fig01}
      \hspace{1cm}
      \includegraphics[height=60mm]{rignot_mouginot_2012_fig02}
      \footlineextra{Source: Rignot et al., 2011; Rignot and Mouginot, 2012.}
    \end{frame}


% ----------------------------------------------------------------------
\section{Glacier monitoring}
% ----------------------------------------------------------------------

% ======================================================================
% Glacier monitoring in Greenland
% ======================================================================

    \begin{sectionframe}{photo_gries_icefall}{0.75}{Glacier monitoring}
      \emph{Bowdoin calving glacier, Northwest Greenland}
    \end{sectionframe}

    \begin{frame}{Past and present glaciers}
      \centering
      \includegraphics[height=60mm]{plot_worldmap}
      \footlineextra{Data: Ehlers and Gibbard, 2004.}
    \end{frame}

% ----------------------------------------------------------------------
\subsection{Context}
% ----------------------------------------------------------------------

    \begin{frame}{The Greenland ice sheet is loosing mass}
      \begin{itemize}
        \item Surface mass balance
        \begin{itemize}
          \item Melt
          \item Sublimation
        \end{itemize}
        \pause\bigskip
        \item Dynamic thinning\\
        \small{(acceleration, thinning and retreat of marine-terminating glaciers)}
        \begin{itemize}
          \item Meltwater-induced acceleration
          \item Iceberg calving
        \end{itemize}
      \end{itemize}
    \end{frame}

    \begin{frame}{Greenland outlet glacier hydrology}
      \centering
      \includegraphics[width=\textwidth]{chu_etal_2014_fig01}
      \footlineextra{Source: Chu et al., 2014.}
    \end{frame}

    \begin{frame}{Bowdoin glacier and borehole locations}
      \centering
      \includegraphics[height=70mm]{map_grl}
      \footlineextra{Data: GIMP, MapQuest Open Aerial.}
    \end{frame}

    \begin{frame}{Greenland mass lost from gravity anomalies}
      \centering
      \includegraphics[width=100mm]<1>{grace_01}
      \includegraphics[width=100mm]<2>{grace_02}
      \includegraphics[width=100mm]<3>{grace_03}\\
      \uncover<2->{Mass loss spread to the north-west.}
      \footlineextra{Source: Sutterley et al. (2014)\only<3->{, Sugiyama et al. (2015)}.}
    \end{frame}

% ----------------------------------------------------------------------
\subsection{Setting}
% ----------------------------------------------------------------------

    \begin{backgroundframe}{photo_bowdoin_aerial}{0.0}{}
    \end{backgroundframe}

    \begin{backgroundframe}{photo_bowdoin_camp}{0.0}{}
    \end{backgroundframe}

    \begin{backgroundframe}{photo_bowdoin_surface}{0.0}{}
    \end{backgroundframe}

    \begin{backgroundframe}{photo_bowdoin_night}{0.0}{}
      \vspace{6cm}\hfill
      \begin{beamercolorbox}[sep=1em,wd=30mm]{titlelike}
        27 Dec. 2015
      \end{beamercolorbox}
    \end{backgroundframe}

% ----------------------------------------------------------------------
\subsection{Results}
% ----------------------------------------------------------------------

    \begin{frame}{Temperature profiles from bed to surface}
      \centering
      \includegraphics[height=70mm]{pf_temp}
    \end{frame}

    \begin{frame}{Ice deformation over 8 months.}
      \centering
      \includegraphics[height=70mm]{pf_tilt}
    \end{frame}

    \begin{frame}{}
      \centering
      Bowdoin tilt animation\\
      \bigskip
      \url{https://polybox.ethz.ch/index.php/s/OwVk9gOpYmtu44a/}
    \end{frame}

    \begin{frame}{Surface velocity from remote sensing}
      \begin{columns}
        \column{60mm}
          \centering
          \includegraphics[width=\textwidth]{satvel_landsat}\\
          Landsat
        \column{60mm}
          \centering
          \includegraphics[width=\textwidth]{satvel_sentinel}\\
          Sentinel-1
      \end{columns}
      \footlineextra{Data: T. Abe, D. Sakakibara\only<3->{, S. Leinss}.}
    \end{frame}

    \begin{frame}{Ice deformation vs. surface velocity}
      \centering
      \includegraphics[height=70mm]{ts_satvel}
      \footlineextra{Data: S. Sugiyama, T. Abe, S. Leinss.}
    \end{frame}

    \begin{frame}{}
      \centering
      Bowdoin icequakes timelapse\\
      \bigskip
      \url{https://www.youtube.com/watch?v=U3F6kv3To3Y}\\
      \footlineextra{Author: Podolskiy et al., 2016.}
    \end{frame}

    \begin{frame}{Conclusions}
      \begin{itemize}
        \item Strong spatial variations in temperature
        \bigskip
        \item Basal sliding accounts for 90 \% of surface motion
        \bigskip
        \item Early summer speed-up at the onset of surface melt
        \bigskip
        \item High seismicity near the calving front
      \end{itemize}
      \pause\bigskip
      $\implies$ Field data is important to calibrate glacier models.
    \end{frame}


% ----------------------------------------------------------------------
\section{Glacier modelling}
% ----------------------------------------------------------------------

% ======================================================================
% Glacier modelling principles
% ======================================================================

    \begin{sectionframe}{photo_gries_icefall}{0.75}{Glacier modelling}
      \emph{Using approximated ice flow physics and numerical methods}
    \end{sectionframe}

% ----------------------------------------------------------------------
\subsection{Overview}
% ----------------------------------------------------------------------

    \begin{frame}{}
      \centering
      ``The future of glaciers''\\
      \bigskip
      \url{https://www.youtube.com/watch?v=eJNIr_0zOyk}
      \footlineextra{Author: Jouvet et al., 2013.}
    \end{frame}

    \begin{frame}{From the natural world to the model}
      \includegraphics{graph_real_to_num}
    \end{frame}

    \begin{frame}{What is a glacier model?}
      \begin{tikzpicture}[align=center]
        \fill[mygreen!75] (-6,-2.5) rectangle (-2.5,2.5);
        \fill[myblue!75] (-2,-3) rectangle (2,3);
        \fill[myred!75] (2.5,-2.5) rectangle (6,2.5);
        \draw[-latex] (-2.5,0) -- (-2,0);
        \draw[-latex] (2,0) -- (2.5,0);
        \node at (-4.25,0) {
          \only<-1>{INPUT}
          \only<2->{topography\\[1em]
                    climate\\[1em]
                    basal heat flux\\[1em]
                    initial conditions\\[1em]
                    ...}
        };
        \node at (0,0) {
          \only<-3>{MODEL}
          \only<4->{thermodynamical core\\[3em]
                    boundary conditions}
        };
        \node at (4.25,0) {
          \only<-2>{OUTPUT}
          \only<3->{glacier geometry\\[1em]
                    velocities\\[1em]
                    temperature\\[1em]
                    stresses\\[1em]
                    ...}
        };
      \end{tikzpicture}
    \end{frame}

% ----------------------------------------------------------------------
\subsection{Ice thermodynamics}
% ----------------------------------------------------------------------

\begin{frame}{Field equations}
  \begin{itemize}[<+->]
  \item Conservation of volume (incompressibility of flow)
    \begin{equation*}
      \div{
         \mathnode<1>{vv}{\vv}
        } = 0
    \end{equation*}
  \item Balance of stresses (Stokes equation)
    \begin{equation*}
      \tdiv{
         \mathnode<2>{cst}{\CST}
       } + 
         \mathnode<2>{rhog}{\rho\,\vect{g}}
       = \vect{0}
    \end{equation*}
  \item Constitutive law for ice (Glen's law)
    \begin{equation*}
        \mathnode<3>{srt}{\SRT}
      = A_0\,e^\frac{-Q}{RT_{pa}}\,\tau_e^{n-1}\,
        \mathnode<3>{dst}{\DST}
      \end{equation*}
  \item Conservation of energy (heat equation)
    \begin{equation*}
        \pdv{
          \mathnode<4>{temp}{T}
        }{t}+\vect{v}\cdot\grad{}\,T =
        \mathnode<4>{diff}{\frac{k}{\rho c} \Delta T}
         +
        \mathnode<4>{hsrc}{\frac{\tr(\DST\SRT)}{\rho c}}
    \end{equation*}
  \end{itemize}
  \begin{tikzpicture}[>=latex, overlay, remember picture, myred, font=\scriptsize]
    \path<1> (vv) -- +(3, -0.25) node (vvt) {ice velocity vector};
    \path<2> (rhog) -- +(3, +0.25) node (cstt) {stress tensor};
    \path<2> (rhog) -- +(3, -0.25) node (rhogt) {gravitational force};
    \path<3> (dst) -- +(2, +0.25) node (srtt) {strain-rate tensor};
    \path<3> (dst) -- +(2, -0.25) node (dstt) {deviatoric stress tensor};
    \path<4> (temp) -- +(-2, +0.25) node (tempt) {ice temperature};
    \path<4> (hsrc) -- +(2, +0.25) node (difft) {diffusion term};
    \path<4> (hsrc) -- +(2, -0.25) node (hsrct) {source term};
    \path[draw, ->]<1> (vv) .. controls +(0,-0.5) and +(-1,0) .. (vvt.west);
    \path[draw, ->]<2> (cst) .. controls +(0,+0.5) and +(-1,0.5) .. (cstt.west);
    \path[draw, ->]<2> (rhog) .. controls +(0,-0.5) and +(-1,-0.5) .. (rhogt.west);
    \path[draw, ->]<3> (srt) .. controls +(0,+0.5) and +(-1,0.5) .. (srtt.west);
    \path[draw, ->]<3> (dst) .. controls +(0,-0.5) and +(-0.5,-0.5) .. (dstt.west);
    \path[draw, ->]<4> (temp) .. controls +(0,0.5) and +(1,0) .. (tempt.east);
    \path[draw, ->]<4> (diff) .. controls +(0,0.5) and +(-1,1) .. (difft.west);
    \path[draw, ->]<4> (hsrc) .. controls +(0,-0.5) and +(-1,-1) .. (hsrct.west);
  \end{tikzpicture}
  \only<5>{}
\end{frame}

%    \begin{frame}{Shallow approximations of the stress balance}
%      $$\vec{\mathrm{div}} \, \bm\sigma + \rho \, \vec{g} = 0
%        \qquad\Rightarrow\qquad\left\{\begin{array}{l}
%        \alert<3-4>{\frac{\partial\tau_{xx}}{\partial x}}
%        \alert<3-4>{+\frac{\partial\tau_{xy}}{\partial y}}
%        \alert<2-4>{+\frac{\partial\tau_{xz}}{\partial z}}
%        \alert<2-4>{=\frac{\partial p}{\partial x}}\\
%        \alert<3-4>{\frac{\partial\tau_{yx}}{\partial x}}
%        \alert<3-4>{+\frac{\partial\tau_{yy}}{\partial y}}
%        \alert<2-4>{+\frac{\partial\tau_{yz}}{\partial z}}
%        \alert<2-4>{=\frac{\partial p}{\partial y}}\\
%        \alert<4-4>{\frac{\partial\tau_{zx}}{\partial x}}
%        \alert<4-4>{+\frac{\partial\tau_{zy}}{\partial y}}
%        \alert<4-4>{+\frac{\partial\tau_{zz}}{\partial z}}
%        \alert<2-4>{=\frac{\partial p}{\partial z} - \rho g}
%        \end{array}\right.$$
%      \begin{itemize}[<+(1)-| alert@+(1)>]
%        \item Shallow ice approximation
%        \item Shallow shelf approximation
%        \item Full stokes
%        \item<+(1)-> And several others...
%      \end{itemize}
%      \pause
%      Stress balance approximations often define the scope of a glacier model.
%    \end{frame}

    \begin{frame}{Shallow approximations}
      \centering
      % TikZ styles
\tikzstyle{vsia}=[myblue, thick]
\tikzstyle{vssa}=[myred, thick]

\begin{tikzpicture}[>=latex]

% white background
\fill[white] (-1.5,-0.5) rectangle +(11.0,6.5);
%\draw [help lines, lightgray] (0,0) grid (8.0,5.5);

% transition lines and calving front coordinates
\coordinate (tr1) at (2.75, 0);  % transition one
\coordinate (tr2) at (5.25, 0);  % transition two
\coordinate (cf) at (7.75, 2);
\draw[lightgray, name path=tr1] (tr1 |- 0,0) -- +(0,5.5) ;
\draw[lightgray, name path=tr2] (tr2 |- 0,0) -- +(0,5.5) ;

% location of the velocity profiles
\path [name path=x1] (1.0,0) -- +(0,5) ;
\path [name path=x2] (3.0,0) -- +(0,5) ;
\path [name path=x3] (5.5,0) -- +(0,5) ;

% bedrock topography
\draw [name path=bed]
    (0,2) .. controls +(-5:4) and +(180:2) .. (8,1);
\fill [pattern=north east lines]
    (0,2) .. controls +(-5:4) and +(180:2) .. (8,1)
          -- +(0,-0.25)
          .. controls +(180:2) and +(-5:4) .. (0,1.75);

% locate the grounding line
\path [name intersections={of=bed and tr2, by=gl}] ;

% surface topograpy
\draw [name path=surf]
    (0,4.5) .. controls +(-10:4) and +(180:3) .. ($(cf)+(0,0.1)$)
          -- (cf) ;

% ice shelf base
\draw [name path=base]
    (gl) .. controls +(15:0.5) and +(180:1) .. ($(cf)-(0,0.4)$)
         -- (cf);

% sea level5
\draw [name path=sl] (cf) -- (cf -| 8,0) ;

% first velocity profile
\only<2->{
  \path [name intersections={of=surf and x1, by=s1}] ;
  \path [name intersections={of=bed and x1, by=b1}] ;
  \draw[vsia] (b1) -- (s1);
  \draw[vsia, name path=vsia]
    (b1) .. controls ++(0.75,0.5) .. ($(s1)+(0.75,0)$);
  \path[name path=vgrid] (s1) foreach \x in {1,...,5} { -- +(2,0) ++(0,-0.5) };
  \path[name intersections={of=x1 and vgrid, name=a, total=\t}];
  \path[name intersections={of=vsia and vgrid, name=b, total=\t}];
  \draw[vsia, ->] (a-1) -- (b-1);
  \draw[vsia, ->] (a-2) -- (b-2);
  \draw[vsia, ->] (a-3) -- (b-3) node [midway, above] {$\vsia$};
  \draw[vsia, ->] (a-4) -- (b-4);
  \draw[vsia, ->] (a-5) -- (b-5);
}

% second velocity profile
\only<4->{
  \path [name intersections={of=surf and x2, by=s2}] ;
  \path [name intersections={of=bed and x2, by=b2}] ;
  \draw[vssa] (b2) -- (s2);
  \draw[vssa, name path=vssa]
    (b2) -- ($(b2)+(1,0)$) -- ($(s2)+(1,0)$);
  \draw[vsia, name path=vsia]
    ($(b2)+(1,0)$) .. controls ++(1,0.5) .. ($(s2)+(2,0)$);
  \path[name path=vgrid] (s2) foreach \x in {1,...,5} { -- +(2,0) ++(0,-0.5) };
  \path[name intersections={of=x2 and vgrid, name=a, total=\t}];
  \path[name intersections={of=vssa and vgrid, name=b, total=\t}];
  \path[name intersections={of=vsia and vgrid, name=c, total=\t}];
  \draw[vssa, ->] (a-1) -- (b-1);
  \draw[vssa, ->] (a-2) -- (b-2);
  \draw[vssa, ->] (a-3) -- (b-3) node [midway, above] {$\vssa$};
  \draw[vssa, ->] (a-4) -- (b-4);
  \draw[vsia, ->] (b-1) -- (c-1);
  \draw[vsia, ->] (b-2) -- (c-2);
  \draw[vsia, ->] (b-3) -- (c-3) node [midway, above] {$\vsia$};
  \draw[vsia, ->] (b-4) -- (c-4);
}

% third velocity profile
\only<3->{
  \path [name intersections={of=surf and x3, by=s3}] ;
  \path [name intersections={of=base and x3, by=b3}] ;
  \draw[vssa] (b3) -- (s3);
  \draw[vssa, name path=vssa]
    (b3) -- ($(b3)+(2,0)$) -- ($(s3)+(2,0)$);
  \path[name path=vgrid] (s3) foreach \x in {1,...,3} { -- +(2,0) ++(0,-0.5) };
  \path[name intersections={of=x3 and vgrid, name=a, total=\t}];
  \path[name intersections={of=vssa and vgrid, name=b, total=\t}];
  \draw[vssa, ->] (a-1) -- (b-1);
  \draw[vssa, ->] (a-2) -- (b-2);
  \draw[vssa, ->] (a-3) -- (b-3) node [midway, above] {$\vssa$};
}

% add transition lines and annotations
\coordinate (m1) at ($(0,0)!0.5!(tr1)$) ;
\coordinate (m2) at ($(tr1)!0.5!(gl)$) ;
\coordinate (m3) at ($(gl)!0.5!(8,0)$) ;
\coordinate (top1) at ($(0,5.25)$) ;
\coordinate (top2) at ($(0,4.75)$) ;
\coordinate (bot) at ($(0,0.25)$) ;

\node<2-> at (m1|-top1) {ice sheet};
\node<4-> at (m2|-top1) {ice stream};
\node<3-> at (m3|-top1) {ice shelf};
\node<2-> [vsia] at (m1|-top2) {Shallow Ice Approximation};
\node<3-> [vssa] at (m3|-top2) {Shallow Shelf Approximation};
\node<2-> at (m1|-bot) {$\vsia\gg\vssa$};
\node<4-> at (m2|-bot) {$\vsia\sim\vssa$};
\node<3-> at (m3|-bot) {$\vsia\ll\vssa$};

\end{tikzpicture}

      \footlineextra{After: Winkelmann et al., 2011}
    \end{frame}

% ----------------------------------------------------------------------
\subsection{Boundary conditions}
% ----------------------------------------------------------------------

    \begin{frame}[label=model-interfaces]{Boundary interfaces}
      \centering
      % TikZ styles
\alt<2>{\tikzstyle{bedflux}=[myred, thick]}{\tikzstyle{bedflux}=[]}
\alt<3>{\tikzstyle{atmflux}=[myblue, thick]}{\tikzstyle{atmflux}=[]}

\begin{tikzpicture}[>=latex]

% white background
\fill[white] (0,0) rectangle +(8,4);
%\draw [help lines, lightgray] (0,0) grid (8.0,4.0);

% grounding line and inland margin coordinates
\coordinate (cf) at (0.5, 1.75) ;
\path[name path=xgl] (2.5,0) -- +(0,4) ;
\path[name path=xim] (7,0) -- +(0,4) ;
\path[name path=xfl] (4.5,0) -- +(0,4) ;

% bedrock topography
\draw [bedflux, name path=bed]
    (0,0.25) .. controls +(0:3) and +(-5:-2.5) ..  (5,1.75)
	  .. controls +(-5:1) and +(10:-1) .. (8,2);
\fill [bedflux, pattern=north east lines]
    (0,0.25) .. controls +(0:3) and +(-5:-2.5) ..  (5,1.75)
	  .. controls +(-5:1) and +(10:-1) .. (8,2)
	  -- +(0,-0.25)
          .. controls +(10:-1) and +(-5:1) .. (5,1.5)
          .. controls +(-5:-2.5) and +(0:3) .. (0,0);

% locate the grounding line
\path [name intersections={of=bed and xgl, by=gl}] ;
\path [name intersections={of=bed and xim, by=im}] ;

% surface topograpy
\draw [atmflux, name path=surf]
    (cf) -- +(0,0.05)
         .. controls +(0:2.75) and +(0:-2) .. (4.5, 3.5)
         .. controls +(0:1) and +(-60:-1.5) .. (im);
\draw [atmflux, dashed]
    (cf) .. controls +(0:2.75) and +(0:-2) .. (4.5, 3.3)
         .. controls +(0:1) and +(-60:-1.5) .. ($(im)-(0:0.1)$);

% ice shelf base
\draw [name path=base]
    (cf) -- +(0,-0.2)
         .. controls +(0:1.5) and +(135:0.5) .. (gl) ;
\draw [dashed]
    ($(cf)+(0,-0.15)$)
         .. controls +(0:1.5) and +(135:0.5) .. ($(gl)+(45:0.1)$) ;

% sea level
\draw [name path=sl] (cf -| 0,0) -- (cf) ;

% text labels
\node[atmflux] (atm) at (1.5, 3) {atmosphere};
\node (ice) at (4.5, 2.25) {ice sheet};
\node (ocn) at (1, 1) {ocean};
\node[bedflux] (bed) at (4.5, 0.5) {bedrock};
\node (pism) at (6.5, 1) {PISM};
\draw[->] (pism) -- (ice);
\draw[->] (pism) -- (bed);

% fluxes
\coordinate [name intersections={of=bed and xfl, by=bfl}] ;
\coordinate [name intersections={of=surf and xfl, by=afl}] ;
\draw<2>[->, bedflux] (bfl) ++(-0.1,0.25) -- ++(0,-0.5) ;
\draw<2>[->, bedflux] (bfl) ++(0.1,-0.25) -- ++(0,+0.5) ;
\draw<3>[->, atmflux] (afl) ++(-0.1,0.25) -- ++(0,-0.5) ;
\draw<3>[->, atmflux] (afl) ++(0.1,-0.25) -- ++(0,+0.5) ;

\end{tikzpicture}
\\
      \bigskip
      \footlineextra{After: PISM documentation (http://pism-docs.org)}
    \end{frame}

%    \begin{frame}{Boundary conditions}
%      \begin{itemize}
%        \item At the atmosphere interface...
%          \begin{itemize}
%            \item mass is gained and lost (surface mass balance)
%            \item air temperature controls ice temperature
%          \end{itemize}
%        \item At the ocean interface...
%          \begin{itemize}
%            \item mass is gained and lost (basal mass balance)
%            \item water temperature controls ice temperature
%            \item calving of icebergs can occur
%          \end{itemize}
%        \item Lithosphere interface...
%          \begin{itemize}
%            \item mass is gained and lost (basal mass balance)
%            \item geothermal heat flux controls ice temperature
%            \item basal topography controls flow
%            \item sliding can occur
%          \end{itemize}
%      \end{itemize}
%      \pause
%      Boundary models are often crude simplifications of complex processes.
%    \end{frame}

% ----------------------------------------------------------------------
%\subsection{Numerical implementation}
% ----------------------------------------------------------------------

%\begin{frame}{\insertsubsection}
%  \begin{block}{Problem statement}
%    The equations of the physical model describe \alert{continuous} fields, but computers %can deal with a \alert{finite number of points} in time and space
%  \end{block}
%\end{frame}

%\againframe{real-to-num}

%\begin{frame}{Example: the equation of temperature diffusion}
%  \begin{itemize}[<+->]
%    \item We use the previous temperature equation...
%    $$\frac{\partial T}{\partial t}
%      = \mathbf{v} \cdot \vec{\mathrm{grad}}\,T
%      + \frac{k}{\rho c} \Delta T
%      + \frac{4 \mu \dot \epsilon_e^2}{\rho c}$$
%    \item ... considering only diffusion...
%    $$\frac{\partial T}{\partial t}
%      = \frac{k}{\rho c} \Delta T$$
%    \item ... in the vertical dimension only.
%    $$\frac{\partial T}{\partial t}
%      = \kappa\,\frac{\partial^2 T}{\partial z^2}
%      \qquad \mathrm{with} \qquad
%      \kappa = \frac{k}{\rho c}$$
%  \end{itemize}
%\end{frame}

%\begin{frame}{Discretization}
%  \begin{itemize}[<+->]
%    \item In the numerical model, temperature is defined on a grid:
%    $$T_i^n=T(z_i,t^n)$$
%    \item Derivatives are approximated (\alert{finite difference method}):
%    $$\frac{\partial T}{\partial t}\simeq\frac{T_i^{n+1}-T_i^n}{\delta t}
%      \qquad \mathrm{and} \qquad
%      \frac{\partial^2 T}{\partial z^2}\simeq\frac{T_{i+1}^n-2T_i^n+T_{i-1}^n}{\delta z^2}$$
%    \item The temperature equation becomes:
%    $$\frac{T_i^{n+1}-T_i^n}{\delta t}
%    =\kappa\,\frac{T_{i+1}^n-2T_i^n+T_{i-1}^n}{\delta z^2}$$
%    \item This method is said \alert{explicit} as one can write:
%    $$T_i^{n+1} = f(T_{i+1}^n, T_i^n, T_{i-1}^n)$$
%  \end{itemize}
%\end{frame}

%\begin{frame}{Some first-order discretization schemes}
%  \begin{itemize}[<+->]
%    \item Explicit scheme
%    $$\frac{T_i^{n+1}-T_i^n}{\delta t}
%      =\kappa\,\frac{T_{i+1}^n-2T_i^n+T_{i-1}^n}{\delta z^2}$$
%    \item Implicit scheme
%    $$\frac{T_i^{n+1}-T_i^n}{\delta t}
%      =\kappa\,\frac{T_{i+1}^{n+1}-2T_i^{n+1}+T_{i-1}^{n+1}}{\delta z^2}$$
%    \item Semi-implicit (time-centered) scheme
%    $$\frac{T_i^{n+1}-T_i^n}{\delta t}
%      =\frac{1}{2}\,\kappa\,\frac{T_{i+1}^n-2T_i^n+T_{i-1}^n}{\delta z^2}
%      +\frac{1}{2}\,\kappa\,\frac{T_{i+1}^{n+1}-2T_i^{n+1}+T_{i-1}^{n+1}}{\delta z^2}$$
%  \end{itemize}
%\end{frame}

%\begin{frame}{Numerical instability}
%  Different schemes have different properties of \alert{convergence} and \alert{stability}.
%  \begin{columns}
%  \column{55mm}
%    \begin{block}{Stable}
%      \includegraphics[width=\linewidth]{temperature-stable}
%    \end{block}
%  \column{55mm}
%    \begin{block}{Unstable}
%      \includegraphics[width=\linewidth]{temperature-unstable}
%    \end{block}
%  \end{columns}
%\end{frame}

%\begin{frame}{Other discretization methods}
%  There exist other methods than the \alert{finite difference} method:
%  \begin{itemize}[<+->]
%    \item Finite element method
%    \item Discrete element method
%    \item Spectral methods
%    \item ...
%  \end{itemize}
%\end{frame}


% ----------------------------------------------------------------------
\section{Examples}
% ----------------------------------------------------------------------

% ======================================================================
% Glacial geomorphology
% ======================================================================

    \begin{sectionframe}{art_hodel_1927_rigi}{0.75}{Glacial geomorphology}
      \emph{The study of glacial landforms.}
      \footlineextra{Image: Hodel, 1927.}
    \end{sectionframe}


% ----------------------------------------------------------------------
\subsection{Regional examples}
% ----------------------------------------------------------------------

    \begin{sectionframe}{art_hodel_1927_rigi}{0.75}{Regional examples}
      \emph{From the Alps.}
      \footlineextra{Image: Hodel, 1927.}
    \end{sectionframe}

% Glacial erosion

    \begin{backgroundframe}[b]{wikipedia_rhone_aerial}{0.0}{}
      \begin{beamercolorbox}[sep=1em,wd=45mm]{titlelike}
        Rhone Glacier, 2011
      \end{beamercolorbox}
      \footlineextra{Photo: Pegasus2, 2011.}
    \end{backgroundframe}

    \begin{backgroundframe}[b]{photo_rhone_terminus}{0.0}{}
      \begin{beamercolorbox}[sep=1em,wd=45mm]{titlelike}
        Proglacial lake, 2015
      \end{beamercolorbox}
    \end{backgroundframe}

    \begin{backgroundframe}[b]{photo-rhone-01}{0.0}{}
      \begin{beamercolorbox}[sep=1em,wd=60mm]{titlelike}
        Glacial polish and fractures
      \end{beamercolorbox}
    \end{backgroundframe}

    \begin{frame}{Glacial erosion processes}
      \begin{itemize}
        \item Abrasion
          \begin{itemize}
            \item Debris entrained by the ice scratch the bedrock
          \end{itemize}
        \bigskip
        \item Plucking
          \begin{itemize}
            \item Fluctuations of water pressure induce fracturing
            \item Loose blocks are carried away by the ice
          \end{itemize}
      \end{itemize}
    \end{frame}

    \begin{frame}{Landscapes of glacial erosion}
      \begin{columns}
        \column{60mm}
          \includegraphics[width=\linewidth]{photo-glacial-valley}\\
          U-shaped valley
        \column{60mm}
          \includegraphics[width=\linewidth]{nasa-alps-720p}\\
          Glacial overdeepenings
      \end{columns}
      \footlineextra{Source: NASA Visible Earth}
    \end{frame}

% Glacial sedimentation

    \begin{frame}{Glaciers transport eroded materials}
      \includegraphics[width=\linewidth]{artwork-agassiz-1840-unteraar}
      \footlineextra{Source: Agassiz, 1840}
    \end{frame}

    \begin{backgroundframe}[b]{glaciersonline_rhone_2008}{0.0}{}
      \begin{beamercolorbox}[sep=1em,wd=45mm]{titlelike}
        Rhone Glacier, 2008
      \end{beamercolorbox}
      \footlineextra{Source: Glaciers online.}
    \end{backgroundframe}

    \begin{backgroundframe}[b]{glaciersonline_rhone_1900}{0.0}{}
      \begin{beamercolorbox}[sep=1em,wd=45mm]{titlelike}
        Rhone Glacier, 1900
      \end{beamercolorbox}
      \footlineextra{Source: Glaciers online.}
    \end{backgroundframe}

    \begin{backgroundframe}[b]{photo-samuel-rhone-720p}{0.0}{}
      \footlineextra{Photo: Samuel Wiesmann, 2012}
      \begin{beamercolorbox}[sep=1em,wd=45mm]{titlelike}
        Little ice age moraines
      \end{beamercolorbox}
    \end{backgroundframe}

    \begin{frame}{Deglaciation of Zürich}
      \includegraphics[height=80mm]{artwork-heer-1865-zurich}
      \footlineextra{Source: Heer, 1865}
    \end{frame}

    \begin{frame}{Deglaciation of Zürich}
      \includegraphics[height=80mm]{wagner-2002-fig08-720p}
      \footlineextra{Source: Wagner, 2002}
    \end{frame}

    \begin{frame}{The moraine was more visible before}
      \includegraphics[width=\linewidth]{artwork-zb-1884-zurich}
      \footlineextra{Source: Zentralbibliothek Zürich}
    \end{frame}

    \begin{frame}{Drumlins near lake Constance}
      \includegraphics[width=\linewidth]{photo-drumlin-720p}
      \footlineextra{Photo: Martin Groll}
    \end{frame}


% ----------------------------------------------------------------------
\subsection{Ice sheet beds}
% ----------------------------------------------------------------------

    \begin{sectionframe}{art_hodel_1927_rigi}{0.75}{Ice sheet beds}
      \emph{Glacial landforms at a different scale.}
      \footlineextra{Image: Hodel, 1927.}
    \end{sectionframe}

    \begin{frame}{Paleo-ice sheets at the Last Glacial Maximum}
      \begin{columns}
        \column{60mm}
          \includegraphics[width=\linewidth]{ehlers-gibbard-2007-fig07}
        \column{60mm}
          \includegraphics[width=\linewidth]{ehlers-gibbard-2007-fig08}
      \end{columns}
      \bigskip
      Additional 120 to 135 m s.l.e. (Clark and Mix, 2002)
      \footlineextra{Source: Ehlers and Gibbard, 2007}
    \end{frame}

    \begin{frame}{Glacial lineations in the Canadian Arctic}
      \includegraphics[width=\linewidth]{deangelis-phd-fig05}
      \footlineextra{Source: De Angelis, 2007}
    \end{frame}

    \begin{frame}{Transition between slow and fast flow}
      \includegraphics[width=\linewidth]{margold-etal-2015-fig02a}
      \footlineextra{Source: Margold et al, 2015}
    \end{frame}

    \begin{frame}{A more complicated situation}
      \includegraphics[width=\linewidth]{deangelis-phd-fig06}
      \footlineextra{}
    \end{frame}

    \begin{frame}{Comparison of modern and relict bedforms}
      \includegraphics[width=\linewidth]{king-etal-2009}
      \footlineextra{Source: King et al, 2009}
    \end{frame}


% ----------------------------------------------------------------------
\subsection{The role of meltwater}
% ----------------------------------------------------------------------

    \begin{sectionframe}{art_hodel_1927_rigi}{0.75}{The role of meltwater}
      \emph{Glaciofluvial erosion and sedimentation.}
      \footlineextra{Image: Hodel, 1927.}
    \end{sectionframe}

    \begin{backgroundframe}[b]{eo_baffinseaice_crop1280x960_barnesicecap}{0.0}{}
      \begin{beamercolorbox}[sep=1em,wd=45mm]{titlelike}
        Baffin Island
      \end{beamercolorbox}
      \footlineextra{Source: http://earthobservatory.nasa.gov}
    \end{backgroundframe}

    \begin{backgroundframe}[b]{eo_barnesicecap_crop1280x960_northslope}{0.0}{}
      \begin{beamercolorbox}[sep=1em,wd=45mm]{titlelike}
        Baffin ice cap
      \end{beamercolorbox}
      \footlineextra{Source: http://earthobservatory.nasa.gov}
    \end{backgroundframe}

    \begin{backgroundframe}[b]{eo_barnesicecap_crop1280x960_geelake}{0.0}{}
      \begin{beamercolorbox}[sep=1em,wd=45mm]{titlelike}
        Barnes ice cap
      \end{beamercolorbox}
      \footlineextra{Source: http://earthobservatory.nasa.gov}
    \end{backgroundframe}

    \begin{backgroundframe}[b]{photo_bc_channels}{0.0}{}
      \begin{beamercolorbox}[sep=1em,wd=45mm]{titlelike}
        Meltwater channels in British Columbia
      \end{beamercolorbox}
      \footlineextra{Source: http://earthobservatory.nasa.gov}
    \end{backgroundframe}

    \begin{backgroundframe}[b]{photo-sihltahl-720p}{0.0}{}
      \hfill
      \begin{beamercolorbox}[sep=1em,wd=45mm]{titlelike}
        Local analogue
      \end{beamercolorbox}
      \footlineextra{Photo: Roland zh}
    \end{backgroundframe}

    \begin{backgroundframe}[b]{photo-esker-fulufjallet-720p}{0.0}{}
      \begin{beamercolorbox}[sep=1em,wd=45mm]{titlelike}
        Esker
      \end{beamercolorbox}
      \footlineextra{Photo: Hanna Lokrantz, SGU}
    \end{backgroundframe}

    \begin{frame}{Esker}
      \includegraphics[width=\linewidth]{photo-esker-paijanne}
      \footlineextra{Photo: visitpaijanne.fi}
    \end{frame}


% ----------------------------------------------------------------------
\subsection{Ice sheet reconstruction}
% ----------------------------------------------------------------------

    \begin{sectionframe}{art_hodel_1927_rigi}{0.75}{Ice sheet reconstruction}
      \emph{Tying all the evidence together.}
      \footlineextra{Image: Hodel, 1927.}
    \end{sectionframe}

    \begin{frame}{Interpreting landform associations}
      \includegraphics<1>[height=80mm]{kleman-etal-2006-fig10a}
      \includegraphics<2>[height=80mm]{kleman-etal-2006-fig10ab}
         \includegraphics<3>[height=80mm]{kleman-etal-2006-fig10}
      \footlineextra{Source: Kleman et al 2006}
    \end{frame}

    \begin{frame}{Dating techniques}
      \begin{itemize}
        \item Radiocarbon
          \begin{itemize}
            \item Organic material outside the ice margin
          \end{itemize}
        \bigskip
        \item Cosmogenic nuclides
          \begin{itemize}
            \item Erratic boulders in stable locations
            \item Bedrock erosion rate
          \end{itemize}
        \bigskip
        \item Optically stimulated luminescence
          \begin{itemize}
            \item Buried glaciofluvial sediments
          \end{itemize}
      \end{itemize}
    \end{frame}

    \begin{frame}{Deglaciation of the Laurentide ice sheet}
      \begin{columns}
        \column{80mm}
          \includegraphics<1>[width=\linewidth]{dyke-prest-1987-s02a-h720}
          \includegraphics<2>[width=\linewidth]{dyke-prest-1987-s02b-h720}
          \includegraphics<3>[width=\linewidth]{dyke-prest-1987-s02c-h720}
          \includegraphics<4>[width=\linewidth]{dyke-prest-1987-s02d-h720}
          \includegraphics<5>[width=\linewidth]{dyke-prest-1987-s03a-h720}
          \includegraphics<6>[width=\linewidth]{dyke-prest-1987-s03b-h720}
          \includegraphics<7>[width=\linewidth]{dyke-prest-1987-s03c-h720}
          \includegraphics<8>[width=\linewidth]{dyke-prest-1987-s03d-h720}
          \includegraphics<9>[width=\linewidth]{dyke-prest-1987-s04a-h720}
          \includegraphics<10>[width=\linewidth]{dyke-prest-1987-s04b-h720}
          \includegraphics<11>[width=\linewidth]{dyke-prest-1987-s04c-h720}
          \includegraphics<12>[height=80mm]{dyke-prest-1987-s01-h720}
        \column{40mm}
          \only<1>{18\,000}%
          \only<2>{14\,000}%
          \only<3>{13\,000}%
          \only<4>{12\,000}%
          \only<5>{11\,000}%
          \only<6>{10\,000}%
          \only<7>{9\,000}%
          \only<8>{8\,400}%
          \only<9>{8\,000}%
          \only<10>{7\,000}%
          \only<11>{5\,000}%
          \only<1-11>{~years ago}%
      \end{columns}
      \footlineextra{Source: Dyke and Prest, 1987}
    \end{frame}

    \begin{frame}{North american ice sheets - pre-LGM landforms}
      \includegraphics[resolution=254]{kleman-etal-2010-fig06}
      \footlineextra{Source: Kleman et al 2010}
    \end{frame}



% ======================================================================
% Paleoglacier modelling examples
% ======================================================================

% ----------------------------------------------------------------------
\subsection{Alpine ice cap}
% ----------------------------------------------------------------------

    \begin{frame}{Past and present glaciers}
      \centering
      \includegraphics[height=60mm]{plot_worldmap}
    \end{frame}

    \begin{backgroundframe}{map_alps}{0.0}{}
      \footlineextra{Data: Ehlers and Gibbard, 2003; ETOPO1; Natural Earth.}
    \end{backgroundframe}

    \begin{backgroundframe}{photo_mount_logan}{0.0}{}
      \vspace{6cm}\hfill
      \begin{beamercolorbox}[sep=1em,wd=45mm]{titlelike}
        Mount Logan, Canada
      \end{beamercolorbox}
      \footlineextra{Photo: Richard Droker}
    \end{backgroundframe}

    \begin{frame}{Problem statement}
      Given the known Last Glacial Maximum extent...
      \bigskip
      \begin{itemize}
        \item Where are the major \alert{nucleation centres}?
        \bigskip
        \item What are the patterns of \alert{last deglaciation}?
      \end{itemize}
      \bigskip\bigskip\bigskip
      \centering
      Tool: Parallel Ice Sheet Model\\
      \url{http://www.pism-docs.org/wiki/doku.php}
    \end{frame}

    \begin{frame}{}
      \centering
      Alpine ice cap animation\\
      \bigskip
      \url{https://polybox.ethz.ch/index.php/s/XaIAM3NdK9tGNUz}\\
    \end{frame}

% ----------------------------------------------------------------------
\subsection{Cordilleran ice sheet}
% ----------------------------------------------------------------------

    \begin{frame}{Past and present glaciers}
      \centering
      \includegraphics[height=60mm]{plot_worldmap}
    \end{frame}

    \begin{backgroundframe}{map_northamerica}{0.0}{}
      \footlineextra{Data: Dyke et al., 2003; ETOPO1; Natural Earth.}
    \end{backgroundframe}

    \begin{backgroundframe}{map_cordillera}{0.0}{}
      \footlineextra{Data: Dyke et al., 2003; ETOPO1; Natural Earth.}
    \end{backgroundframe}

%    \begin{frame}{Reanalysed temperatures (1981--2010)}
%      \centering
%      \includegraphics<1>{plot-temp-01}
%      \includegraphics<2>{plot-temp-02}\\
%      \uncover<2>{Strong variations in temperature and temperature seasonality}
%      \footlineextra{Data: NARR}
%    \end{frame}

%    \begin{frame}{Reanalysed precipitations (1981--2010)}
%      \centering
%      \includegraphics<1>{plot-prec-01}
%      \includegraphics<2>{plot-prec-02}\\
%      \uncover<2>{Strong variations in precipitation and timing of precipitation}
%      \footlineextra{Data: NARR}
%    \end{frame}

%    \begin{frame}{Simulations of the last glacial cycle (120--0\,kyr)}
%      \begin{columns}
%      \column{80mm}
%      \begin{itemize}
%        \item<+-> Spatial climate patterns from the NARR
%          \begin{itemize}
%            \item monthly temperature
%            \item monthly precipitation
%            \item monthly temperature standard deviation
%          \end{itemize}
%        \item<+-> Time-dependent temperature change from
%          \begin{itemize}
%            \item two Greenland \alert{ice cores}
%              \begin{itemize}
%            \item GRIP
%                \item NGRIP
%              \end{itemize}
%            \item two Antarctic \alert{ice cores}
%              \begin{itemize}
%                \item EPICA
%                  \item Vostok
%              \end{itemize}
%            \item two Pacific ocean \alert{sediment records}
%              \begin{itemize}
%                \item ODP 1012
%                \item ODP 1020
%              \end{itemize}
%          \end{itemize}
%      \end{itemize}
%      \column{40mm}
%      \uncover<2>{\includegraphics{map_records}}
%      \end{columns}
%    \end{frame}

    \begin{frame}{}
      \centering
      Cordilleran ice sheet animation\\
      \bigskip
      \url{https://polybox.ethz.ch/index.php/s/4prO6tvF3nh3G1W}\\
    \end{frame}

    \begin{frame}{Conclusions}
      \begin{itemize}
        \item The maximum stage appears short-lived.
        \bigskip
        \item Most of the glacial cycle features mountain ice caps.
          \begin{itemize}
            \item Skeena Mountains
          \end{itemize}
        \bigskip
        \item Deglaciation towards northern interior ranges.
          \begin{itemize}
            \item Selwyn Mountains, Skeena Mountains
          \end{itemize}
      \end{itemize}
    \end{frame}

% ----------------------------------------------------------------------
\subsection{Haizishan ice cap}
% ----------------------------------------------------------------------

    \begin{frame}{Past and present glaciers}
      \centering
      \includegraphics[height=60mm]{plot_worldmap}
    \end{frame}

    \begin{frame}{Past and present glaciers}
      \centering
      \includegraphics[width=\linewidth]{fu_etal_2013_fig09}
      \footlineextra{Source: Fu et al., 2013.}
    \end{frame}

    \begin{frame}{Past and present glaciers}
      \centering
      \includegraphics[height=60mm]{fu_etal_2013_fig07}
      \footlineextra{Source: Fu et al., 2013.}
    \end{frame}

    \begin{frame}{Modelling Haizishan ice cap}
      \centering
      \includegraphics[height=60mm]{plot_hzs_final}
    \end{frame}



    \begin{backgroundframe}{photo_gries_icefall}{0.0}{}
      \vspace{60mm}\hfill
      \begin{beamercolorbox}[sep=1em,wd=30mm]{titlelike}
        Thank you!
      \end{beamercolorbox}
    \end{backgroundframe}

% ======================================================================
\end{document}
% ======================================================================
