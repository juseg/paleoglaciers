% CORDILLERA
% ==========

% Beamer presentation header
% ======================================================================

\documentclass[aspectratio=1610]{beamer}

% ----------------------------------------------------------------------
% Packages
% ----------------------------------------------------------------------

% encoding
\usepackage[utf8]{inputenc}
\usepackage[english]{babel}
\usepackage{fontawesome}

% math
\usepackage{bm}
\usepackage{physics}
%\usepackage{amsmath}

% graphics
\graphicspath{{figures/}}

% tikz
\usepackage{tikz}
\usetikzlibrary{intersections}
\usetikzlibrary{calc}
\usetikzlibrary{patterns}
%\everymath{\displaystyle}

% ----------------------------------------------------------------------
% Beamer themes
% ----------------------------------------------------------------------

\useinnertheme{rectangles}
\useoutertheme{default}
\usecolortheme{dove}
\setbeamertemplate{navigation symbols}{}
\setbeamertemplate{footline}[frame number]

% ----------------------------------------------------------------------
% Colors
% ----------------------------------------------------------------------

% Color brewer paired CMYK
\definecolor{lightblue}  {rgb}{166, 206, 227}  % qual_Paired_12_01
\definecolor{darkblue}   {rgb}{ 31, 120, 180}  % qual_Paired_12_02
\definecolor{lightgreen} {rgb}{178, 223, 138}  % qual_Paired_12_03
\definecolor{darkgreen}  {rgb}{ 51, 160,  44}  % qual_Paired_12_04
\definecolor{lightred}   {rgb}{251, 154, 153}  % qual_Paired_12_05
\definecolor{darkred}    {rgb}{227,  26,  28}  % qual_Paired_12_06
\definecolor{lightorange}{rgb}{253, 191, 111}  % qual_Paired_12_07
\definecolor{darkorange} {rgb}{255, 127,   0}  % qual_Paired_12_08
\definecolor{lightpurple}{rgb}{202, 178, 214}  % qual_Paired_12_09
\definecolor{darkpurple} {rgb}{106,  61, 154}  % qual_Paired_12_10
\definecolor{lightbrown} {rgb}{255, 255, 153}  % qual_Paired_12_11
\definecolor{darkbrown}  {rgb}{177,  89,  40}  % qual_Paired_12_12

% Color shortcuts 
\definecolor{myblue}     {RGB}{ 31, 120, 180}  % qual_Paired_12_02
\definecolor{mygreen}    {RGB}{ 51, 160,  44}  % qual_Paired_12_04
\definecolor{myred}      {RGB}{227,  26,  28}  % qual_Paired_12_06

% Beamer colors
%\setbeamercolor{alerted text}{fg=myblue} 
\setbeamercolor{titlelike}{parent=palette primary}
\setbeamercolor{frametitle}{parent=palette secondary}
\setbeamercolor{block}{parent=palette secondary}
\newcommand{\textrbf}[1]{\textcolor{myred}{\textbf{#1}}}
\newcommand{\textgbf}[1]{\textcolor{mygreen}{\textbf{#1}}}
\newcommand{\textbbf}[1]{\textcolor{myblue}{\textbf{#1}}}

% Transparent blocks
\addtobeamertemplate{block begin}{\pgfsetfillopacity{0.75}}{\pgfsetfillopacity{1}}
\addtobeamertemplate{block alerted begin}{\pgfsetfillopacity{0.75}}{\pgfsetfillopacity{1}}
\addtobeamertemplate{block example begin}{\pgfsetfillopacity{0.75}}{\pgfsetfillopacity{1}}

% ----------------------------------------------------------------------
% Math shortcuts
% ----------------------------------------------------------------------

% Vectors and tensors
\newcommand{\vect}[1]{\va*{#1}} % bold arrow vectors
\newcommand{\tens}[1]{\vb*{#1}} % bold tensors

% Differential operators
\renewcommand{\div}[1]{\mathrm{div}\,#1}            % divergence
\renewcommand{\grad}[1]{\vect{\mathrm{grad}}\,#1}   % gradient
\newcommand{\tdiv}[1]{\vect{\mathrm{div}}\,#1}      % tensor divergence
\newcommand{\tgrad}[1]{\tens{\mathrm{grad}}\,#1}    % tensor gradient
\newcommand{\matdv}[1]{\pdv{#1}{t}+\vect{v}\cdot\grad{}\,#1}  % material dv.

% Common notations
\newcommand{\doteps}[0]{\dot{\epsilon}} % epsilon dot
\newcommand{\IDT}[0]{\tens{\delta}}     % Identity tensor
\newcommand{\CST}[0]{\tens{\sigma}}     % Cauchy stress tensor
\newcommand{\DST}[0]{\tens{\tau}}       % deviatoric stress tensor
\newcommand{\SRT}[0]{\tens{\doteps}}    % strain-rate tensor
\newcommand{\vv}[0]{\vect{v}}           % velocity vector
\newcommand{\vsia}[0]{\vv_{\mathrm{SIA}}}   % SIA velocity
\newcommand{\vssa}[0]{\vv_{\mathrm{SSA}}}   % SSA velocity
\newcommand{\PDD}[0]{\mathrm{PDD}}
\newcommand{\sPDD}[0]{\sigma_{\mathrm{PDD}}}

% Common units
\newcommand{\e}[1]{\ensuremath{\times 10^{#1}}}
\newcommand{\chem}[1]{\ensuremath{\mathsf{#1}}}
\newcommand{\unit}[1]{\ensuremath{\mathsf{#1}}}
\newcommand{\degree}[0]{\ensuremath{^{\circ}}}
\newcommand{\degC}[0]{\unit{{\degree}C}}

% TikZ invisible math node
\newcommand<>{\mathnode}[2]{%
  \alt#3{\tikz[baseline, remember picture]%
         \node[anchor=base, inner sep=0pt, myred](#1){$#2$};}%
        {#2}}

% TikZ left-aligned semi-transparent text box
\newcommand{\alphabox}[1]{%
  \flushleft\tikz\node[fill=white, fill opacity=0.75, text opacity=1,%
                       align=left, inner sep=1em] {#1};}

% strike-out
\usepackage[normalem]{ulem}
\renewcommand<>{\sout}[1]{\only#2{\beameroriginal{\sout}}{#1}}


% ----------------------------------------------------------------------
% Special frames
% ----------------------------------------------------------------------

\newlength{\imgwidth}

\newenvironment{backgroundframe}[4][c]% options, image, alpha, title
  {% check width of scale image
   \settowidth{\imgwidth}{\includegraphics[height=\paperheight]{#2}}%
   % add image as background canvas
   \setbeamertemplate{background canvas}{%
    \tikz[overlay, remember picture]%
      \node[opacity=1.0] at (current page.center) {%
        \ifdim\imgwidth<\paperwidth
          \includegraphics[width=\paperwidth]{#2}
        \else
          \includegraphics[height=\paperheight]{#2}
        \fi
      };}%
   % add semi-transparent rectangle overlay
   \setbeamertemplate{background}{%
    \tikz\node[fill=white, inner sep=0, opacity=#3,%
               text width=\paperwidth, text height=\paperheight]{};}%
   \begin{frame}[#1]{#4}}
  {\end{frame}}


% section frame with custom image
%\newenvironment{sectionframe}[4][]% options, image, alpha, title
%  {\begin{backgroundframe}{#2}{#3}{}\centering{\huge #4}\\\bigskip}
%  {\end{backgroundframe}}
\newenvironment{sectionframe}[2][]% options, title
  {\begin{backgroundframe}{art_hodel_1927_rigi}{0.75}{}\centering{\huge #2}\\\bigskip}
  {\footlineextra{Background: Hodel, 1927.}\end{backgroundframe}}

% coverfig source: http://tex.stackexchange.com/a/101073
\newcommand<>{\coverfig}[1]{
  \coordinate (zone) at ($(fig.south)!#1!(fig.north)$);
  \fill#2[white, opacity=1.0] (fig.south west) rectangle(fig.east|-zone);
}

%% covergig modified
%\newcommand<>{\coverfig}[2]{\begin{tikzpicture}
%  \node[inner sep=0] (fig) {#1};
%  \coordinate (zone) at ($(fig.south)!#2!(fig.north)$);
%  \fill<1>[black, opacity=0.75] (fig.south west) rectangle (fig.east|-zone);
%\end{tikzpicture}}

% ----------------------------------------------------------------------
% Footline macro
% From http://tex.stackexchange.com/questions/5491/
%      how-do-i-insert-text-into-the-footline-of-a-specific-slide-in-beamer
% ----------------------------------------------------------------------

\makeatletter

% add a macro that saves its argument
\newcommand{\footlineextra}[1]{\gdef\insertfootlineextra{#1}}
%\newbox\footlineextrabox

% add a beamer template that sets the saved argument in a box.
% The * means that the beamer font and color "footline extra" are automatically added. 
\defbeamertemplate*{footline extra}{default}{
    %\begin{beamercolorbox}[ht=2.25ex,dp=1ex,leftskip=\Gm@lmargin]{footline extra}
    \hspace{\Gm@lmargin}\insertfootlineextra
    %\par\vspace{2.5pt}
    %\end{beamercolorbox}
}

\addtobeamertemplate{footline}{%
    % set the box with the extra footline material but make it add no vertical space
    %\setbox\footlineextrabox=\vbox{\usebeamertemplate*{footline extra}}
    %\vskip -\ht\footlineextrabox
    %\vskip -\dp\footlineextrabox
    %\box\footlineextrabox%
    \usebeamertemplate*{footline extra}
}

% alternatively override the footline template
%\setbeamertemplate{footline}{\insertfootlineextra\hfill\insertframenumber\,/\,\inserttotalframenumber}}

% patch \begin{frame} to reset the footline extra material
\let\beamer@original@frame=\frame
\def\frame{\gdef\insertfootlineextra{}\beamer@original@frame}
\footlineextra{}
\makeatother

\setbeamercolor{footline extra}{fg=structure.fg}% for instance


\title{Palaeoglaciology}
\author{Julien Seguinot}
\institute{ETH Zürich and Hokkaido University}
\date{December 17, 2018}

% paper size and margins
%\geometry{papersize={192mm,120mm}}
\setbeamersize{text margin left=4mm}
\setbeamersize{text margin right=4mm}
%\setbeamertemplate{footline}{\vspace{2cm}}
\usepackage{multicol}


% ======================================================================
\begin{document}
% ======================================================================

    % LATER: add links to sources
    % LATER: semi-transparent footline

% Title page

    % LATER background image aspect ratio
    \begin{backgroundframe}[b]{art_hodel_1927_rigi}{0.0}{}
      \alphabox{
        \textbf{\inserttitle}\\
        \insertauthor, \insertinstitute, \insertdate\\
        \href{mailto: seguinot@vaw.baug.ethz.ch}
             {{\faEnvelope} seguinot@vaw.baug.ethz.ch}
        \href{http://people.ee.ethz.ch/~juliens}
             {{\faHome} http://people.ee.ethz.ch/$\sim$juliens}}
      \footlineextra{Artwork: Hodel, 1927.}
    \end{backgroundframe}

    \centering  % center all frames

% Glacier cover past and present

    \begin{frame}{Modern glaciers}
      \includegraphics{plot_worldmap_now}\\
      \pause\bigskip
      \begin{columns}
        \column{60mm}
          \centering
          Antarctica: 58.3 m s.l.e.\\
          (Fretwell et al., 2013)
        \column{60mm}
          \centering
          Greenland: 7.3 m s.l.e.\\
          (Bamber et al., 2013)
      \end{columns}
      \footlineextra{Figure: after Seguinot, \emph{PhD thesis}, 2014.
                     Data: Ehlers and Gibbard, 2007.}
    \end{frame}

    \begin{frame}{Last Glacial Maximum (25--17\,ka)}
      \includegraphics{plot_worldmap_lgm}\\
      \bigskip
      \begin{columns}  % for alignment with previous slide
        \column{120mm}
          \centering
          Additional 120 to 135 m s.l.e. (Clark and Mix, 2002)\pause\\
          \emph{How can one reconstruct past glacier changes in time?}
      \end{columns}
      \footlineextra{Figure: after Seguinot, \emph{PhD thesis}, 2014.
                     Data: Ehlers and Gibbard, 2007.}
    \end{frame}


% ----------------------------------------------------------------------
\section{Glacial geology}
% ----------------------------------------------------------------------

    \begin{sectionframe}{Outline}
      \begin{multicols}{2}
        \tableofcontents
      \end{multicols}
    \end{sectionframe}

    \begin{sectionframe}{Glacial geomorphology}
      \emph{Reconstruct palaeo-glaciers using field evidence.}
    \end{sectionframe}

% -- -- -- -- -- -- -- -- -- -- -- -- -- -- -- -- -- -- -- -- -- -- -- -
\subsection{Historic perspective}
% -- -- -- -- -- -- -- -- -- -- -- -- -- -- -- -- -- -- -- -- -- -- -- -

    \begin{sectionframe}{Historic perspective}
      \emph{Beginning in the 19th century in the Alps.}
    \end{sectionframe}

% Glacial theory

    \begin{backgroundframe}[b]{julien-xt10-170922-184315-170922-184318-5x8}{0.0}{}
      \alphabox{A landscape of current deglaciation\\
                Gorner Glacier, Oct. 2017}
      \footlineextra{Photo: Seguinot, 2017.}
    \end{backgroundframe}

    \begin{frame}{Before the glacial theory...}
      \includegraphics[height=75mm]{art_cole_1829_deluge}\\
      \emph{The Subsiding of the Waters of the Deluge} (Cole, 1829)
      \footlineextra{Artwork: Cole, 1829.}
      % https://americanart.si.edu/artwork/subsiding-waters-deluge-5080
    \end{frame}

    %FIXME photo erratic
    \begin{frame}{}
      J.-P. Perraudin, 1815\\
      \bigskip
      % ``Les glaciers de nos montagnes ont eu jadis une bien plus grande
      %   extension qu'aujourd'hui.'' p.241
      \par\emph{``Toute notre vallée, jusqu'à une grande hauteur au-dessus de la
              Drance, a été occupée par un vaste glacier, qui se prolongeait
              jusques à Martigny, comme le prouvent les blocs de roches qu'on
              trouve dans les environs de cette ville, et qui sont trop gros
              pour que l'eau ait put les y amener.''}\\
      \bigskip
      in J. de Charpentier,\\
      \emph{Essai sur les glaciers et sur le terrain erratique
            du bassin du Rhône},
      p.241--242, 1841.
    \end{frame}

    \begin{frame}{Mapping the erratic deposit (1841)}
      \includegraphics[height=75mm]{map_decharpentier_1841}
      \footlineextra{Source: de Charpentier, 1841.}
    \end{frame}

    \begin{frame}{The glacial theory was first controversial}
      \begin{columns}
        \column{60mm}
          \begin{itemize}
            \item Glacial landforms had been explained by a great flood.
            \pause\bigskip
            \item Single or multiple glaciations?
              \begin{itemize}
                \item catastrophism vs uniformitarianism.
              \end{itemize}
            \pause\bigskip
            \item Greenland and Antarctica had not yet been explored.
          \end{itemize}
        \column{60mm}
          \includegraphics[width=60mm]{photo_nansen_greenland}
          \emph{The first crossing of Greenland} (Nansen, 1988).
      \end{columns}
      \footlineextra{Photo: Nansen, 1890.}
    \end{frame}

% How many glaciations

    \begin{frame}{Two glaciations in North America (1882)}
      \includegraphics[width=\linewidth]{map-chamberlin-1882}
      \footlineextra{Source: Chamberlin, 1882.}
    \end{frame}

    \begin{frame}{Four glaciations in the Alps (1909)}
      \begin{columns}
        \column{45mm}
          \begin{itemize}
            \item<+-> In 1909, at least four glaciations
                      were identified in the Alps.
              \alert{four} glaciations
              \begin{itemize}
                \item Würm
                \item Riss
                \item Mindel
                \item Günz
              \end{itemize}
            \item<+-> In 2011, ``at least \alert{eight}, but probably more
                      lowland glaciations during the Quaternary.''
            \item<+-> \emph{How many glaciations were there in total?}
          \end{itemize}
        \column{75mm}
          \includegraphics<1->[height=80mm]{map_penck_bruckner_1909}
      \end{columns}
      \footlineextra{Source: Penck and Brückner, 1909
                             \uncover<1>{; Preusser et al. (2011).}}
    \end{frame}

% Oxygen isotopes and temperature

    \begin{frame}{Isotopic composition of the ocean}
      \begin{columns}
        \column{60mm}
          \includegraphics<1->[width=\linewidth]{cartoon-umich-d18o-a}
        \column{60mm}
          \includegraphics<2->[width=\linewidth]{cartoon-umich-d18o-b}
      \end{columns}
      \footlineextra{Cartoon: University of Michigan.}
      %https://globalchange.umich.edu/globalchange1/current/labs/Lab10_Vostok/Vostok.htm
    \end{frame}

    % LATER embed photos
    \begin{frame}[t]{Glacial cycles}
      \begin{columns}
        \column{30mm}
          \vspace{2.5mm}
          \includegraphics[width=\linewidth]{photo-morgane-foram-720p}\\
          \vspace{2.5mm}
          \only<-2>{\vspace{30mm}}
          \includegraphics<3>[width=\linewidth]{photo-csiro-bubbles-720p}
          \vspace{10.0mm}
        \column{90mm}
          \includegraphics<1>[width=\linewidth]{plot_timeseries_02}
          \includegraphics<2>[width=\linewidth]{plot_timeseries_02}
          \includegraphics<3>[width=\linewidth]{plot_timeseries_03}
      \end{columns}
      \footlineextra{Data: Lisiecki and Raymo (2005), Jouzel et al., 2007}
    \end{frame}

    % FIXME reframe
    \begin{frame}{}
      \includegraphics[height=90mm]{hodell_2016_fig01}\\
      \footlineextra{Figure: Hodell, 2016.}
      % https://americanart.si.edu/artwork/subsiding-waters-deluge-5080
    \end{frame}
    % https://dx.doi.org/10.1126/science.aal4111 

% -- -- -- -- -- -- -- -- -- -- -- -- -- -- -- -- -- -- -- -- -- -- -- -
\subsection{Glacial landforms}
% -- -- -- -- -- -- -- -- -- -- -- -- -- -- -- -- -- -- -- -- -- -- -- -

    \begin{sectionframe}{Glacial landforms}
      \emph{Glacial erosion and sedimentation.}
    \end{sectionframe}

% Glacial erosion

    \begin{backgroundframe}[b]{julien-xt1-150910-163122}{0.0}{}
      \alphabox{Recently exposed bedrock}
      \vspace{30mm}
      \footlineextra{Photo: Seguinot, 2016.}
    \end{backgroundframe}

    % LATER lake photo

    \begin{frame}{Glacial erosion processes}
      \begin{itemize}
        \item Abrasion
          \begin{itemize}
            \item Debris entrained by the ice scratch the bedrock
          \end{itemize}
        \bigskip
        \item Plucking
          \begin{itemize}
            \item Fluctuations of water pressure induce fracturing
            \item Loose blocks are carried away by the ice
          \end{itemize}
      \end{itemize}
    \end{frame}

    %LATER update photos
    \begin{frame}{Landscapes of glacial erosion}
      \begin{columns}
        \column{60mm}
          \includegraphics[width=\linewidth]{photo-glacial-valley}\\
          U-shaped valley
        \column{60mm}
          \includegraphics[width=\linewidth]{nasa-alps-720p}\\
          Glacial overdeepenings
      \end{columns}
      \footlineextra{Source: NASA Visible Earth}
    \end{frame}

    \begin{frame}{Glaciers transport eroded materials}
      \includegraphics[height=80mm]{art_agassiz_1840_unteraar}
      \footlineextra{Artwork: Agassiz, 1840}
    \end{frame}

% Glacial sedimentation

    \begin{frame}{Rhone Glacier}
      \includegraphics[width=\linewidth]{julien-xt10-161027-152047-161027-152107-1x3}
      The Rhone Glacier and its foreland.
      \footlineextra{Photo: Seguinot, 2016.}
    \end{frame}

    \begin{backgroundframe}[b]{julien-xt10-161027-152047-161027-152107_5x8}{0.0}{}
      \alphabox{Rhone Glacier foreland}
      \footlineextra{Photo: Seguinot, 2016.}
    \end{backgroundframe}

    \begin{backgroundframe}[b]{julien-xt1-161027-132713}{0.0}{}
      \alphabox{Rhone Glacier foreland}
      \footlineextra{Photo: Seguinot, 2016.}
    \end{backgroundframe}

    \begin{backgroundframe}[b]{julien-xt1-161027-154120}{0.0}{}
      \alphabox{Rhone Glacier moraine}
      \footlineextra{Photo: Seguinot, 2016.}
    \end{backgroundframe}

    \begin{frame}{Rhone Glacier}
      \includegraphics[height=80mm]{swisstopo_rhone}
      \footlineextra{Figure: Swisstopo, 2017.}
      % rhone https://map.geo.admin.ch/?bgLayer=ch.swisstopo.pixelkarte-farbe&layers=ch.swisstopo.swissalti3d-reliefschattierung&E=2672104.23&N=1158478.89&zoom=7&lang=en&topic=ech
    \end{frame}

    \begin{frame}{Rhone Glacier}
      \includegraphics[height=80mm]{swisstopo_gletsch}
      \footlineextra{Figure: Swisstopo, 2017.}
      % gletsch https://map.geo.admin.ch/?bgLayer=ch.swisstopo.pixelkarte-farbe&layers=ch.swisstopo.swissalti3d-reliefschattierung&E=2671318.68&N=1157782.77&zoom=8&lang=en&topic=ech
    \end{frame}

    \begin{backgroundframe}[b]{art_hogard_1848_rhone}{0.0}{}
      \alphabox{Rhone Glacier, 1848}
      \footlineextra{Artwork: Hogard, 1848.}
      %http://doi.org/10.3932/ethz-a-000016730
    \end{backgroundframe}

    \begin{frame}{Zürich Lake moraine}
      \includegraphics[height=80mm]{swisstopo_zurich}
      \footlineextra{Figure: Swisstopo, 2017.}
      % zurich https://map.geo.admin.ch/?bgLayer=ch.swisstopo.pixelkarte-farbe&layers=ch.swisstopo.swissalti3d-reliefschattierung&E=2685304.12&N=1245692.48&zoom=5
    \end{frame}

    \begin{frame}{Deglaciation of Zürich}
      \begin{columns}
        \column{40mm}
          \includegraphics[width=40mm]{wagner-2002-fig08-720p}
        \column{80mm}
          \includegraphics[width=80mm]{art_heer_1865_zurich}
      \end{columns}
      \footlineextra{Figure: Wagner, 2002. Artwork: Heer, 1865.}
    \end{frame}

    \begin{backgroundframe}[b]{photo_drumlin_raderach}{0.0}{}
      \alphabox{Drumlins near lake Constance}
      \footlineextra{Photo: Martin Groll}
      %https://commons.wikimedia.org/wiki/File:Drumlin_1789.jpg
    \end{backgroundframe}

    %FIXME photo Scotland
    %FIXME sectionning

%% -- -- -- -- -- -- -- -- -- -- -- -- -- -- -- -- -- -- -- -- -- -- -- -
%\subsection{Ice sheet beds}
%% -- -- -- -- -- -- -- -- -- -- -- -- -- -- -- -- -- -- -- -- -- -- -- -

    \begin{sectionframe}{Ice sheet beds}
      \emph{Glacial landforms at a different scale.}
    \end{sectionframe}

    \begin{frame}{Glacial lineations in the Canadian Arctic}
      \includegraphics[height=80mm]{deangelis-phd-fig05}
      \footlineextra{Source: De Angelis, 2007}
    \end{frame}

    \begin{frame}{Transition between slow and fast flow}
      \includegraphics[height=80mm]{margold-etal-2015-fig02a}
      \footlineextra{Source: Margold et al, 2015}
    \end{frame}

    \begin{frame}{A more complicated situation}
      \includegraphics[height=80mm]{deangelis-phd-fig06}
      \footlineextra{}
    \end{frame}

    \begin{frame}{Comparison of modern and relict bedforms}
      \includegraphics[width=150mm]{king-etal-2009}
      \footlineextra{Source: King et al, 2009}
    \end{frame}

% -- -- -- -- -- -- -- -- -- -- -- -- -- -- -- -- -- -- -- -- -- -- -- -
\subsection{The role of meltwater}
% -- -- -- -- -- -- -- -- -- -- -- -- -- -- -- -- -- -- -- -- -- -- -- -

    \begin{sectionframe}{The role of meltwater}
      \emph{Glaciofluvial erosion and sedimentation.}
    \end{sectionframe}
    % LATER nice photo of Sihl valley

    \begin{frame}{The Sihl Valley}
      \includegraphics<1>[height=80mm]{swisstopo_zurich}
      \includegraphics<2>[height=80mm]{swisstopo_zurichsee}
      \includegraphics<3>[height=80mm]{swisstopo_sihltal}
    % zurich https://map.geo.admin.ch/?bgLayer=ch.swisstopo.pixelkarte-farbe&layers=ch.swisstopo.swissalti3d-reliefschattierung&E=2685304.12&N=1245692.48&zoom=5
    % zurichsee https://map.geo.admin.ch/?bgLayer=ch.swisstopo.pixelkarte-farbe&layers=ch.swisstopo.swissalti3d-reliefschattierung&E=2692918.92&N=1241040.80&zoom=5&lang=en&topic=ech
    % sihltal https://map.geo.admin.ch/?bgLayer=ch.swisstopo.pixelkarte-farbe&layers=ch.swisstopo.swissalti3d-reliefschattierung&E=2697568.92&N=1237210.80&zoom=4&lang=en&topic=ech
      \footlineextra{Figure: Swisstopo, 2017.}
    \end{frame}

    % LATER alphabox
    % LATER background aspect ratio
    \begin{backgroundframe}[b]{eo_baffinseaice_crop1280x960_barnesicecap}{0.0}{}
      \begin{beamercolorbox}[sep=1em,wd=45mm]{titlelike}
        Baffin Island
      \end{beamercolorbox}
      \footlineextra{Source: http://earthobservatory.nasa.gov}
    \end{backgroundframe}

    % LATER background aspect ratio
    \begin{backgroundframe}[b]{eo_barnesicecap_crop1280x960_northslope}{0.0}{}
      \begin{beamercolorbox}[sep=1em,wd=45mm]{titlelike}
        Barnes ice cap
      \end{beamercolorbox}
      \footlineextra{Source: http://earthobservatory.nasa.gov}
    \end{backgroundframe}

    % LATER background aspect ratio
    \begin{backgroundframe}[b]{eo_barnesicecap_crop1280x960_geelake}{0.0}{}
      \begin{beamercolorbox}[sep=1em,wd=45mm]{titlelike}
        Barnes ice cap
      \end{beamercolorbox}
      \footlineextra{Source: http://earthobservatory.nasa.gov}
    \end{backgroundframe}

    % LATER update background photo
    \begin{backgroundframe}[b]{julien-z650-120724-075318}{0.0}{}
      \alphabox{Meltwater channels in British Columbia}
      \footlineextra{Photo: Seguinot, 2012.}
    \end{backgroundframe}

    % FIXME resize
    \begin{backgroundframe}[b]{photo_channel_mendenhall_orig}{0.0}{}
      \alphabox{R-channel at Mendenhall Glacier}
      \footlineextra{Photo: Seguinot, 2016.}
    \end{backgroundframe}
    % https://www.flickr.com/photos/25949441@N02/10856649976

    \begin{backgroundframe}[b]{photo_esker_fulufjallet}{0.0}{}
      \alphabox{Eskers at Fulufjället, Sweden}
      \footlineextra{Photo: Hanna Lokrantz, SGU, 2004}
      % https://www.flickr.com/photos/geologicalsurveyofsweden/6853882122
    \end{backgroundframe}

    \begin{backgroundframe}[b]{photo_esker_punkaharju}{0.0}{}
      \alphabox{Esker at Punkaharju, Finland}
      \footlineextra{Photo: Kosti Keistinen, 2017}
      % https://pixabay.com/en/finnish-punkaharju-landscape-summer-2963810/
    \end{backgroundframe}


% -- -- -- -- -- -- -- -- -- -- -- -- -- -- -- -- -- -- -- -- -- -- -- -
\subsection{Ice sheet reconstruction}
% -- -- -- -- -- -- -- -- -- -- -- -- -- -- -- -- -- -- -- -- -- -- -- -

    \begin{sectionframe}{Ice sheet reconstruction}
      \emph{Combining all the evidence.}
    \end{sectionframe}

    %LATER bigger
    \begin{frame}{Interpreting landform associations}
      \includegraphics<1>[height=80mm]{kleman-etal-2006-fig10a}
      \includegraphics<2>[height=80mm]{kleman-etal-2006-fig10ab}
         \includegraphics<3>[height=80mm]{kleman-etal-2006-fig10}
      \footlineextra{Source: Kleman et al 2006}
    \end{frame}

    \begin{frame}{Dating techniques}
      \begin{itemize}
        \item Radiocarbon
          \begin{itemize}
            \item Organic material outside the ice margin
          \end{itemize}
        \bigskip
        \item Cosmogenic nuclides
          \begin{itemize}
            \item Erratic boulders in stable locations
            \item Bedrock erosion rate
          \end{itemize}
        \bigskip
        \item Optically stimulated luminescence
          \begin{itemize}
            \item Buried glaciofluvial sediments
          \end{itemize}
      \end{itemize}
    \end{frame}

    \begin{frame}{Deglaciation of the Laurentide ice sheet}
      \begin{columns}
        \column{80mm}
          \includegraphics<1>[width=\linewidth]{dyke-prest-1987-s02a-h720}
          \includegraphics<2>[width=\linewidth]{dyke-prest-1987-s02b-h720}
          \includegraphics<3>[width=\linewidth]{dyke-prest-1987-s02c-h720}
          \includegraphics<4>[width=\linewidth]{dyke-prest-1987-s02d-h720}
          \includegraphics<5>[width=\linewidth]{dyke-prest-1987-s03a-h720}
          \includegraphics<6>[width=\linewidth]{dyke-prest-1987-s03b-h720}
          \includegraphics<7>[width=\linewidth]{dyke-prest-1987-s03c-h720}
          \includegraphics<8>[width=\linewidth]{dyke-prest-1987-s03d-h720}
          \includegraphics<9>[width=\linewidth]{dyke-prest-1987-s04a-h720}
          \includegraphics<10>[width=\linewidth]{dyke-prest-1987-s04b-h720}
          \includegraphics<11>[width=\linewidth]{dyke-prest-1987-s04c-h720}
          \includegraphics<12>[height=80mm]{dyke-prest-1987-s01-h720}
        \column{40mm}
          \only<1>{18\,000}%
          \only<2>{14\,000}%
          \only<3>{13\,000}%
          \only<4>{12\,000}%
          \only<5>{11\,000}%
          \only<6>{10\,000}%
          \only<7>{9\,000}%
          \only<8>{8\,400}%
          \only<9>{8\,000}%
          \only<10>{7\,000}%
          \only<11>{5\,000}%
          \only<1-11>{~years ago}%
      \end{columns}
      \footlineextra{Source: Dyke and Prest, 1987}
    \end{frame}

%    \begin{frame}{North american ice sheets - pre-LGM landforms}
%      %\includegraphics[resolution=254]{kleman-etal-2010-fig06}
%      \footlineextra{Source: Kleman et al 2010}
%    \end{frame}



% ----------------------------------------------------------------------
\section{Glacier modelling}
% ----------------------------------------------------------------------

    \begin{sectionframe}{Outline}
      \begin{multicols}{2}
        \tableofcontents
      \end{multicols}
    \end{sectionframe}

    \begin{sectionframe}{Numerical modelling}
      \emph{Reconstructing palaeo-glaciers using ice physics.}
    \end{sectionframe}


% -- -- -- -- -- -- -- -- -- -- -- -- -- -- -- -- -- -- -- -- -- -- -- -
\subsection{Ice thermodynamics}
% -- -- -- -- -- -- -- -- -- -- -- -- -- -- -- -- -- -- -- -- -- -- -- -

    \begin{sectionframe}{Ice thermodynamics}
      \emph{The core of the model.}
    \end{sectionframe}

%    \begin{frame}{Why model past glaciers?}
%      \begin{itemize}[<+->]
%        \item To \alert{complement/interpret} geomorphological evidence
%        \item To \alert{test and constrain} glacier models
%        \item To \alert{spin-up} an ice sheet model for a projection run
%      \end{itemize}
%    \end{frame}

%    \begin{frame}[label=real_to_num]{From the natural world to the model}
%      \includegraphics{graph_real_to_num}
%    \end{frame}

%    \begin{frame}{What is a glacier model?}
%      \begin{tikzpicture}[align=center]
%        \fill[mygreen!75] (-6,-2.5) rectangle (-2.5,2.5);
%        \fill[myblue!75] (-2,-3) rectangle (2,3);
%        \fill[myred!75] (2.5,-2.5) rectangle (6,2.5);
%        \draw[-latex] (-2.5,0) -- (-2,0);
%        \draw[-latex] (2,0) -- (2.5,0);
%        \node at (-4.25,0) {
%          \only<-1>{INPUT}
%          \only<2->{topography\\[1em]
%                    climate\\[1em]
%                    basal heat flux\\[1em]
%                    initial conditions\\[1em]
%                    ...}
%        };
%        \node at (0,0) {
%          \only<-3>{MODEL}
%          \only<4->{thermodynamical core\\[3em]
%                    boundary conditions}
%        };
%        \node at (4.25,0) {
%          \only<-2>{OUTPUT}
%          \only<3->{glacier geometry\\[1em]
%                    velocities\\[1em]
%                    temperature\\[1em]
%                    stresses\\[1em]
%                    ...}
%        };
%      \end{tikzpicture}
%    \end{frame}

    % LATER consistent variables
    \begin{frame}<1-5>{Field equations}
      \begin{itemize}[<+->]
      \item Conservation of volume (incompressibility of flow)
        \begin{equation*}
          \div{
             \mathnode<1>{vv}{\vv}
            } = 0 
        \end{equation*}
      \item Balance of stresses (Stokes equation)
        \begin{equation*}
          \tdiv{
             \mathnode<2>{cst}{\CST}
           } + 
             \mathnode<2>{rhog}{\rho\,\vect{g}}
           = \vect{0}
        \end{equation*}
      \item Constitutive law for ice (Glen's law)
        \begin{equation*}
            \mathnode<3>{srt}{\SRT}
          = A_0\,e^\frac{-Q}{RT_{pa}}\,\tau_e^{n-1}\,
            \mathnode<3>{dst}{\DST}
          \end{equation*}
      \item Conservation of energy (heat equation)
        \begin{equation*}
            \pdv{
              \mathnode<4>{temp}{T}
            }{t}+\vect{v}\cdot\grad{}\,T =
            \mathnode<4>{diff}{\frac{k}{\rho c} \Delta T}
             +
            \mathnode<4>{hsrc}{\frac{\tr(\DST\SRT)}{\rho c}}
        \end{equation*}
      \end{itemize}
      \begin{tikzpicture}[>=latex, overlay, remember picture, myred, font=\scriptsize]
        \path<1> (vv) -- +(3, -0.25) node (vvt) {ice velocity vector};
        \path<2> (rhog) -- +(3, +0.25) node (cstt) {stress tensor};
        \path<2> (rhog) -- +(3, -0.25) node (rhogt) {gravitational force};
        \path<3> (dst) -- +(2, +0.25) node (srtt) {strain-rate tensor};
        \path<3> (dst) -- +(2, -0.25) node (dstt) {deviatoric stress tensor};
        \path<4> (temp) -- +(-2, +0.25) node (tempt) {ice temperature};
        \path<4> (hsrc) -- +(2, +0.25) node (difft) {diffusion term};
        \path<4> (hsrc) -- +(2, -0.25) node (hsrct) {source term};
        \path[draw, ->]<1> (vv) .. controls +(0,-0.5) and +(-1,0) .. (vvt.west);
        \path[draw, ->]<2> (cst) .. controls +(0,+0.5) and +(-1,0.5) .. (cstt.west);
        \path[draw, ->]<2> (rhog) .. controls +(0,-0.5) and +(-1,-0.5) .. (rhogt.west);
        \path[draw, ->]<3> (srt) .. controls +(0,+0.5) and +(-1,0.5) .. (srtt.west);
        \path[draw, ->]<3> (dst) .. controls +(0,-0.5) and +(-0.5,-0.5) .. (dstt.west);
        \path[draw, ->]<4> (temp) .. controls +(0,0.5) and +(1,0) .. (tempt.east);
        \path[draw, ->]<4> (diff) .. controls +(0,0.5) and +(-1,1) .. (difft.west);
        \path[draw, ->]<4> (hsrc) .. controls +(0,-0.5) and +(-1,-1) .. (hsrct.west);
      \end{tikzpicture}
    \end{frame}

%    % LATER fix highlighting
%    \begin{frame}{Shallow approximations of the stress balance}
%      $$\vec{\mathrm{div}} \, \bm\sigma + \rho \, \vec{g} = 0
%        \qquad\Rightarrow\qquad\left\{\begin{array}{l}
%        \alert<3-4>{\frac{\partial\tau_{xx}}{\partial x}}
%        \alert<3-4>{+\frac{\partial\tau_{xy}}{\partial y}}
%        \alert<2-4>{+\frac{\partial\tau_{xz}}{\partial z}}
%        \alert<2-4>{=\frac{\partial p}{\partial x}}\\
%        \alert<3-4>{\frac{\partial\tau_{yx}}{\partial x}}
%        \alert<3-4>{+\frac{\partial\tau_{yy}}{\partial y}}
%        \alert<2-4>{+\frac{\partial\tau_{yz}}{\partial z}}
%        \alert<2-4>{=\frac{\partial p}{\partial y}}\\
%        \alert<4-4>{\frac{\partial\tau_{zx}}{\partial x}}
%        \alert<4-4>{+\frac{\partial\tau_{zy}}{\partial y}}
%        \alert<4-4>{+\frac{\partial\tau_{zz}}{\partial z}}
%        \alert<2-4>{=\frac{\partial p}{\partial z} - \rho g}
%        \end{array}\right.$$
%      \begin{itemize}[<+(1)-| alert@+(1)>]
%        \item Shallow ice approximation
%        \item Shallow shelf approximation
%        \item Full stokes
%        \item<+(1)-> And several others...
%      \end{itemize}
%      \pause
%      Stress balance approximations often define the scope of a glacier model.
%    \end{frame}

    \begin{frame}{Shallow approximations}
      \centering
      % TikZ styles
\tikzstyle{vsia}=[myblue, thick]
\tikzstyle{vssa}=[myred, thick]

\begin{tikzpicture}[>=latex]

% white background
\fill[white] (-1.5,-0.5) rectangle +(11.0,6.5);
%\draw [help lines, lightgray] (0,0) grid (8.0,5.5);

% transition lines and calving front coordinates
\coordinate (tr1) at (2.75, 0);  % transition one
\coordinate (tr2) at (5.25, 0);  % transition two
\coordinate (cf) at (7.75, 2);
\draw[lightgray, name path=tr1] (tr1 |- 0,0) -- +(0,5.5) ;
\draw[lightgray, name path=tr2] (tr2 |- 0,0) -- +(0,5.5) ;

% location of the velocity profiles
\path [name path=x1] (1.0,0) -- +(0,5) ;
\path [name path=x2] (3.0,0) -- +(0,5) ;
\path [name path=x3] (5.5,0) -- +(0,5) ;

% bedrock topography
\draw [name path=bed]
    (0,2) .. controls +(-5:4) and +(180:2) .. (8,1);
\fill [pattern=north east lines]
    (0,2) .. controls +(-5:4) and +(180:2) .. (8,1)
          -- +(0,-0.25)
          .. controls +(180:2) and +(-5:4) .. (0,1.75);

% locate the grounding line
\path [name intersections={of=bed and tr2, by=gl}] ;

% surface topograpy
\draw [name path=surf]
    (0,4.5) .. controls +(-10:4) and +(180:3) .. ($(cf)+(0,0.1)$)
          -- (cf) ;

% ice shelf base
\draw [name path=base]
    (gl) .. controls +(15:0.5) and +(180:1) .. ($(cf)-(0,0.4)$)
         -- (cf);

% sea level5
\draw [name path=sl] (cf) -- (cf -| 8,0) ;

% first velocity profile
\only<2->{
  \path [name intersections={of=surf and x1, by=s1}] ;
  \path [name intersections={of=bed and x1, by=b1}] ;
  \draw[vsia] (b1) -- (s1);
  \draw[vsia, name path=vsia]
    (b1) .. controls ++(0.75,0.5) .. ($(s1)+(0.75,0)$);
  \path[name path=vgrid] (s1) foreach \x in {1,...,5} { -- +(2,0) ++(0,-0.5) };
  \path[name intersections={of=x1 and vgrid, name=a, total=\t}];
  \path[name intersections={of=vsia and vgrid, name=b, total=\t}];
  \draw[vsia, ->] (a-1) -- (b-1);
  \draw[vsia, ->] (a-2) -- (b-2);
  \draw[vsia, ->] (a-3) -- (b-3) node [midway, above] {$\vsia$};
  \draw[vsia, ->] (a-4) -- (b-4);
  \draw[vsia, ->] (a-5) -- (b-5);
}

% second velocity profile
\only<4->{
  \path [name intersections={of=surf and x2, by=s2}] ;
  \path [name intersections={of=bed and x2, by=b2}] ;
  \draw[vssa] (b2) -- (s2);
  \draw[vssa, name path=vssa]
    (b2) -- ($(b2)+(1,0)$) -- ($(s2)+(1,0)$);
  \draw[vsia, name path=vsia]
    ($(b2)+(1,0)$) .. controls ++(1,0.5) .. ($(s2)+(2,0)$);
  \path[name path=vgrid] (s2) foreach \x in {1,...,5} { -- +(2,0) ++(0,-0.5) };
  \path[name intersections={of=x2 and vgrid, name=a, total=\t}];
  \path[name intersections={of=vssa and vgrid, name=b, total=\t}];
  \path[name intersections={of=vsia and vgrid, name=c, total=\t}];
  \draw[vssa, ->] (a-1) -- (b-1);
  \draw[vssa, ->] (a-2) -- (b-2);
  \draw[vssa, ->] (a-3) -- (b-3) node [midway, above] {$\vssa$};
  \draw[vssa, ->] (a-4) -- (b-4);
  \draw[vsia, ->] (b-1) -- (c-1);
  \draw[vsia, ->] (b-2) -- (c-2);
  \draw[vsia, ->] (b-3) -- (c-3) node [midway, above] {$\vsia$};
  \draw[vsia, ->] (b-4) -- (c-4);
}

% third velocity profile
\only<3->{
  \path [name intersections={of=surf and x3, by=s3}] ;
  \path [name intersections={of=base and x3, by=b3}] ;
  \draw[vssa] (b3) -- (s3);
  \draw[vssa, name path=vssa]
    (b3) -- ($(b3)+(2,0)$) -- ($(s3)+(2,0)$);
  \path[name path=vgrid] (s3) foreach \x in {1,...,3} { -- +(2,0) ++(0,-0.5) };
  \path[name intersections={of=x3 and vgrid, name=a, total=\t}];
  \path[name intersections={of=vssa and vgrid, name=b, total=\t}];
  \draw[vssa, ->] (a-1) -- (b-1);
  \draw[vssa, ->] (a-2) -- (b-2);
  \draw[vssa, ->] (a-3) -- (b-3) node [midway, above] {$\vssa$};
}

% add transition lines and annotations
\coordinate (m1) at ($(0,0)!0.5!(tr1)$) ;
\coordinate (m2) at ($(tr1)!0.5!(gl)$) ;
\coordinate (m3) at ($(gl)!0.5!(8,0)$) ;
\coordinate (top1) at ($(0,5.25)$) ;
\coordinate (top2) at ($(0,4.75)$) ;
\coordinate (bot) at ($(0,0.25)$) ;

\node<2-> at (m1|-top1) {ice sheet};
\node<4-> at (m2|-top1) {ice stream};
\node<3-> at (m3|-top1) {ice shelf};
\node<2-> [vsia] at (m1|-top2) {Shallow Ice Approximation};
\node<3-> [vssa] at (m3|-top2) {Shallow Shelf Approximation};
\node<2-> at (m1|-bot) {$\vsia\gg\vssa$};
\node<4-> at (m2|-bot) {$\vsia\sim\vssa$};
\node<3-> at (m3|-bot) {$\vsia\ll\vssa$};

\end{tikzpicture}

      \footlineextra{After: Winkelmann et al., 2011}
    \end{frame}


% -- -- -- -- -- -- -- -- -- -- -- -- -- -- -- -- -- -- -- -- -- -- -- -
\subsection{Boundary conditions}
% -- -- -- -- -- -- -- -- -- -- -- -- -- -- -- -- -- -- -- -- -- -- -- -

    \begin{sectionframe}{Boundary conditions}
      \emph{Atmosphere, bedrock, and ocean.}
    \end{sectionframe}

    \begin{frame}[label=model-interfaces]{Boundary interfaces}
      \centering
      % TikZ styles
\alt<2>{\tikzstyle{bedflux}=[myred, thick]}{\tikzstyle{bedflux}=[]}
\alt<3>{\tikzstyle{atmflux}=[myblue, thick]}{\tikzstyle{atmflux}=[]}

\begin{tikzpicture}[>=latex]

% white background
\fill[white] (0,0) rectangle +(8,4);
%\draw [help lines, lightgray] (0,0) grid (8.0,4.0);

% grounding line and inland margin coordinates
\coordinate (cf) at (0.5, 1.75) ;
\path[name path=xgl] (2.5,0) -- +(0,4) ;
\path[name path=xim] (7,0) -- +(0,4) ;
\path[name path=xfl] (4.5,0) -- +(0,4) ;

% bedrock topography
\draw [bedflux, name path=bed]
    (0,0.25) .. controls +(0:3) and +(-5:-2.5) ..  (5,1.75)
	  .. controls +(-5:1) and +(10:-1) .. (8,2);
\fill [bedflux, pattern=north east lines]
    (0,0.25) .. controls +(0:3) and +(-5:-2.5) ..  (5,1.75)
	  .. controls +(-5:1) and +(10:-1) .. (8,2)
	  -- +(0,-0.25)
          .. controls +(10:-1) and +(-5:1) .. (5,1.5)
          .. controls +(-5:-2.5) and +(0:3) .. (0,0);

% locate the grounding line
\path [name intersections={of=bed and xgl, by=gl}] ;
\path [name intersections={of=bed and xim, by=im}] ;

% surface topograpy
\draw [atmflux, name path=surf]
    (cf) -- +(0,0.05)
         .. controls +(0:2.75) and +(0:-2) .. (4.5, 3.5)
         .. controls +(0:1) and +(-60:-1.5) .. (im);
\draw [atmflux, dashed]
    (cf) .. controls +(0:2.75) and +(0:-2) .. (4.5, 3.3)
         .. controls +(0:1) and +(-60:-1.5) .. ($(im)-(0:0.1)$);

% ice shelf base
\draw [name path=base]
    (cf) -- +(0,-0.2)
         .. controls +(0:1.5) and +(135:0.5) .. (gl) ;
\draw [dashed]
    ($(cf)+(0,-0.15)$)
         .. controls +(0:1.5) and +(135:0.5) .. ($(gl)+(45:0.1)$) ;

% sea level
\draw [name path=sl] (cf -| 0,0) -- (cf) ;

% text labels
\node[atmflux] (atm) at (1.5, 3) {atmosphere};
\node (ice) at (4.5, 2.25) {ice sheet};
\node (ocn) at (1, 1) {ocean};
\node[bedflux] (bed) at (4.5, 0.5) {bedrock};
\node (pism) at (6.5, 1) {PISM};
\draw[->] (pism) -- (ice);
\draw[->] (pism) -- (bed);

% fluxes
\coordinate [name intersections={of=bed and xfl, by=bfl}] ;
\coordinate [name intersections={of=surf and xfl, by=afl}] ;
\draw<2>[->, bedflux] (bfl) ++(-0.1,0.25) -- ++(0,-0.5) ;
\draw<2>[->, bedflux] (bfl) ++(0.1,-0.25) -- ++(0,+0.5) ;
\draw<3>[->, atmflux] (afl) ++(-0.1,0.25) -- ++(0,-0.5) ;
\draw<3>[->, atmflux] (afl) ++(0.1,-0.25) -- ++(0,+0.5) ;

\end{tikzpicture}
\\
      \bigskip
      \footlineextra{After: PISM documentation (http://pism-docs.org)}
    \end{frame}

    \begin{frame}{Boundary conditions}
      \begin{itemize}
        \item At the atmosphere interface...
          \begin{itemize}
            \item mass is gained and lost (surface mass balance)
            \item air temperature controls ice temperature
          \end{itemize}
        \item At the ocean interface...
          \begin{itemize}
            \item mass is gained and lost (basal mass balance)
            \item water temperature controls ice temperature
            \item calving of icebergs can occur
          \end{itemize}
        \item Lithosphere interface...
          \begin{itemize}
            \item mass is gained and lost (basal mass balance)
            \item geothermal heat flux controls ice temperature
            \item basal topography controls flow
            \item sliding can occur
          \end{itemize}
      \end{itemize}
      \pause
      Boundary models are often crude simplifications of complex processes.
    \end{frame}

    \begin{frame}{Surface accumulation (snowfall)}
      \centering
      \begin{tikzpicture}[>=latex]

% white background
\fill[white] (0,0) rectangle +(8,4);
%\draw[help lines, lightgray] (0,0) grid +(8,4);

\coordinate (o) at (3,1);
\coordinate (tsnow) at ($(o)+(0,2)$);
\coordinate (train) at ($(o)+(2,0)$);

\draw[->] (0.25,|-o) -- +(7.0,0) node[above] {$T$ (\degC)};
\draw[->] (o|-,0.25) -- +(0,3.5) node[right] {snow fraction};
\draw[myblue, thick] (0.25,|-tsnow) -- (tsnow) -- (train) -- (7,|-train) ;

\node [below left] at (o) {0};
\node [below left] at (tsnow) {1};
\node [below] at (train) {2};

\end{tikzpicture}
\\
      \begin{itemize}
        \item equal to precipitation when temperature is below 0\degC
        \item decreases to zero linearly with temperature between 0 and 2\degC
      \end{itemize}
    \end{frame}

    \begin{frame}{Surface ablation (melt)}
      \begin{columns}
        \column{60mm}
         \begin{itemize}
           \item<1-> proportional to the number of positive degree-days
             $$ \mathrm{PDD} = \int_{t_1}^{t_2} T^{+} \dd{t} $$
           \item<3-> includes daily temperature variability
             $$ T^{+} = \int_{0}^{\infty} T_{ac}
                         \, e^{-\frac{(T-T_{ac})^2}{2\sigma^2}} \dd{T} $$
          \end{itemize}
        \column{60mm}
          \centering    
          \only<1-3>{\alt<3>{\tikzstyle{sd}=[draw, thick, myred]}{\tikzstyle{sd}=[]}

\begin{tikzpicture}[>=latex]

% white background
\fill[white] (0,0) rectangle +(6,6.5);
%\draw[help lines, lightgray] (0,0) grid +(8,4);

% melt plot
\only<1->{
  \coordinate (o) at (3,4.75);
  \coordinate (p1) at ($(o)-(2.75,0)$);
  \coordinate (p2) at ($(o)+(2.75,1.25)$);
  \draw[->] (p1) -- +(5.5,0) node[above left] {$T$ (\degC)};
  \draw[->] (o) -- +(0,1.5) node[right] {$M$};
  \node[below] at (o) {0};
  \draw (p1) -- (o) -- (p2) ;
  \path[sd, dashed] (p1)
    .. controls ($(p1)!0.75!(o)$) and ($(p2)!0.75!(o)$) .. (p2) ;
}

% seasonal cycle plot
\only<2->{
  \coordinate (o) at (0.25,0.75);
  \coordinate (p1) at ($(o)+(0,1.5)$);
  \coordinate (p2) at ($(o)+(2.5,3)$);
  \coordinate (p3) at ($(o)+(5,1.5)$);
  \draw[->] (o) -- +(5.5,0) node[above left] {$t$ (yr)};
  \draw[->] (o) -- +(0,3.5) node[right] {$T$};
  \node [below] at (o) {0};
  \node [below] at (p3|-o) {1};
  \draw (p1)  .. controls +(1,0) and +(-1,0) .. (p2)
              .. controls +(1,0) and +(-1,0) .. (p3) ;
  \path[sd, dashed] ($(p1)+(0,-1)$)
              .. controls +(1,0) and +(-1,0) .. ($(p2)+(0,-0.25)$)
              .. controls +(1,0) and +(-1,0) .. ($(p3)+(0,-1)$) ;
  \path[sd, dashed] ($(p1)+(0,1)$)
              .. controls +(1,0) and +(-1,0) .. ($(p2)+(0,0.25)$)
              .. controls +(1,0) and +(-1,0) .. ($(p3)+(0,1)$) ;
  \path<3>[<->, sd] (p3) -- ($(p3)+(0,1)$) node [midway, right] {$\sigma$};
}

\end{tikzpicture}

}
      \end{columns}
    \end{frame}

    \begin{frame}{Climate forcing}
      In palaeoglacier modelling, input climate is the main unknown.\\
      Sources of information include:\\\bigskip
      \begin{itemize}
        \item Palaeo-climate simulations
          \begin{itemize}
            \item Coarse resolution
            \item Based on ice sheet reconstruction
          \end{itemize}
        \item Palaeo-climate proxies
          \begin{itemize}
            \item Often far from glaciers
          \end{itemize}
      \end{itemize}
    \end{frame}


% -- -- -- -- -- -- -- -- -- -- -- -- -- -- -- -- -- -- -- -- -- -- -- -
\subsection{Numerical implementation}
% -- -- -- -- -- -- -- -- -- -- -- -- -- -- -- -- -- -- -- -- -- -- -- -

    \begin{sectionframe}{Numerical implementation}
      \emph{The equations of the physical model describe \alert{continuous}
        fields, \\ but computers only deal with a \alert{finite number of
        points} in time and space.}
    \end{sectionframe}

    \begin{frame}{Example: the equation of temperature diffusion}
      \begin{itemize}[<+->]
        \item We use the previous temperature equation...
        $$\frac{\partial T}{\partial t}
          = \mathbf{v} \cdot \vec{\mathrm{grad}}\,T
          + \frac{k}{\rho c} \Delta T
          + \frac{4 \mu \dot \epsilon_e^2}{\rho c}$$
        \item ... considering only diffusion...
        $$\frac{\partial T}{\partial t}
         = \frac{k}{\rho c} \Delta T$$
        \item ... in the vertical dimension only.
        $$\frac{\partial T}{\partial t}
          = \kappa\,\frac{\partial^2 T}{\partial z^2}
          \qquad \mathrm{with} \qquad
          \kappa = \frac{k}{\rho c}$$
      \end{itemize}
    \end{frame}

    \begin{frame}{Discretization}
      \begin{itemize}[<+->]
        \item In the numerical model, temperature is defined on a grid:
        $$T_i^n=T(z_i,t^n)$$
        \item Derivatives are approximated (\alert{finite difference method}):
        $$\frac{\partial T}{\partial t}\simeq\frac{T_i^{n+1}-T_i^n}{\delta t}
          \qquad \mathrm{and} \qquad
          \frac{\partial^2 T}{\partial z^2}\simeq
            \frac{T_{i+1}^n-2T_i^n+T_{i-1}^n}{\delta z^2}$$
        \item The temperature equation becomes:
        $$\frac{T_i^{n+1}-T_i^n}{\delta t}
        =\kappa\,\frac{T_{i+1}^n-2T_i^n+T_{i-1}^n}{\delta z^2}$$
        \item This method is said \alert{explicit} as one can write:
        $$T_i^{n+1} = f(T_{i+1}^n, T_i^n, T_{i-1}^n)$$
      \end{itemize}
    \end{frame}

    \begin{frame}{Some first-order discretization schemes}
      \begin{itemize}[<+->]
        \item Explicit scheme
        $$\frac{T_i^{n+1}-T_i^n}{\delta t}
         =\kappa\,\frac{T_{i+1}^n-2T_i^n+T_{i-1}^n}{\delta z^2}$$
        \item Implicit scheme
        $$\frac{T_i^{n+1}-T_i^n}{\delta t}
          =\kappa\,\frac{T_{i+1}^{n+1}-2T_i^{n+1}+T_{i-1}^{n+1}}{\delta z^2}$$
        \item Semi-implicit (time-centered) scheme
        $$\frac{T_i^{n+1}-T_i^n}{\delta t}
          =\frac{1}{2}\,\kappa\,\frac{T_{i+1}^n-2T_i^n+T_{i-1}^n}{\delta z^2}
          +\frac{1}{2}\,\kappa\,\frac{T_{i+1}^{n+1}-2T_i^{n+1}+T_{i-1}^{n+1}}{\delta z^2}$$
      \end{itemize}
    \end{frame}

    \begin{frame}{Numerical instability}
      Different schemes have different properties of
      \alert{convergence} and \alert{stability}.
      \begin{columns}
      \column{60mm}
        \begin{block}{Stable}
          \includegraphics[width=\linewidth]{seguinot_2009_stable}
        \end{block}
      \pause
      \column{60mm}
        \begin{block}{Unstable}
          \includegraphics[width=\linewidth]{seguinot_2009_unstable}
        \end{block}
      \end{columns}
    \end{frame}

    \begin{frame}{Other discretization methods}
      There exist other methods than the \alert{finite difference} method:
      \begin{itemize}
        \item Finite element method
        \item Discrete element method
        \item Spectral methods
        \item ...
      \end{itemize}
    \end{frame}


% ----------------------------------------------------------------------
\subsection{Example application}
% ----------------------------------------------------------------------

    \begin{sectionframe}{Example}
      \emph{Modelling Alpine paleo-glaciers.}
    \end{sectionframe}

%    \begin{frame}{Last Glacial Maximum ice extent}
%      \includegraphics[height=80mm]{alpcyc_hr_locmap}
%      \footlineextra{Data: SRTM; Natural Earth; Ehlers et al., 2011.}
%    \end{frame}

    \begin{frame}{The Alps during the Last Glacial Maximum}
      \includegraphics[height=80mm]{map_coutterand_2017}
      \footlineextra{Figure: Coutterand, 2017.}
    \end{frame}

    \begin{backgroundframe}[b]{photo_mount_logan}{0.0}{}
      \alphabox{Kluane ice field}
      \footlineextra{Photo: Richard Droker.}
    \end{backgroundframe}

    \begin{sectionframe}{Problem statement}
      \begin{itemize}
        \item Can an ice flow model reproduce geologic reconstructions?
          \begin{itemize}
            \item Last Glacial Maximum ice \alert{extent} from moraines
            \item Last Glacial Maximum ice \alert{thickness} from trimlines
          \end{itemize}
      \end{itemize}
      \bigskip\bigskip\bigskip\pause
      Tool: Parallel Ice Sheet Model (PISM)\\
      \bigskip
      Method: High-resolution simulation of the last glacial cycle (120--0 ka)
    \end{sectionframe}

    \begin{frame}{Modelled Last Glacial Maximum}
      \includegraphics[height=80mm]{alpcyc_hr_maxextvel}
    \end{frame}

    \begin{frame}{Multiple advance and retreat}
      \includegraphics[height=80mm]{alpcyc_hr_profiles_four}
    \end{frame}

    \begin{frame}{Timing of Last Glacial Maximum}
      \includegraphics[height=80mm]{alpcyc_hr_maxthkage}
    \end{frame}

    \begin{sectionframe}{Conclusions}
      \begin{itemize}
        \item Glacial geology allows reconstruction of palaeo-glaciers\\
          \emph{\small but the record is sparse in time and space.}
        \pause\bigskip
        \item Numerical modelling allows to fill the gaps\\
          \emph{\small but uncertainties are large.}
      \end{itemize}
    \end{sectionframe}

    \begin{sectionframe}{Thank you!}
    \end{sectionframe}


% ======================================================================
\end{document}
% ======================================================================
