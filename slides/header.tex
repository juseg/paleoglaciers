% Beamer presentation header
% ======================================================================

\documentclass[aspectratio=1610]{beamer}

% ----------------------------------------------------------------------
% Packages
% ----------------------------------------------------------------------

% encoding
\usepackage[utf8]{inputenc}
\usepackage[english]{babel}
\usepackage{fontawesome}

% math
\usepackage{bm}
\usepackage{physics}
%\usepackage{amsmath}

% graphics
\graphicspath{{figures/}}

% tikz
\usepackage{tikz}
\usetikzlibrary{intersections}
\usetikzlibrary{calc}
\usetikzlibrary{patterns}
%\everymath{\displaystyle}

% ----------------------------------------------------------------------
% Beamer themes
% ----------------------------------------------------------------------

\useinnertheme{rectangles}
\useoutertheme{default}
\usecolortheme{dove}
\setbeamertemplate{navigation symbols}{}
\setbeamertemplate{footline}[frame number]

% ----------------------------------------------------------------------
% Colors
% ----------------------------------------------------------------------

% Color brewer paired CMYK
\definecolor{lightblue}  {rgb}{166, 206, 227}  % qual_Paired_12_01
\definecolor{darkblue}   {rgb}{ 31, 120, 180}  % qual_Paired_12_02
\definecolor{lightgreen} {rgb}{178, 223, 138}  % qual_Paired_12_03
\definecolor{darkgreen}  {rgb}{ 51, 160,  44}  % qual_Paired_12_04
\definecolor{lightred}   {rgb}{251, 154, 153}  % qual_Paired_12_05
\definecolor{darkred}    {rgb}{227,  26,  28}  % qual_Paired_12_06
\definecolor{lightorange}{rgb}{253, 191, 111}  % qual_Paired_12_07
\definecolor{darkorange} {rgb}{255, 127,   0}  % qual_Paired_12_08
\definecolor{lightpurple}{rgb}{202, 178, 214}  % qual_Paired_12_09
\definecolor{darkpurple} {rgb}{106,  61, 154}  % qual_Paired_12_10
\definecolor{lightbrown} {rgb}{255, 255, 153}  % qual_Paired_12_11
\definecolor{darkbrown}  {rgb}{177,  89,  40}  % qual_Paired_12_12

% Color shortcuts 
\definecolor{myblue}     {RGB}{ 31, 120, 180}  % qual_Paired_12_02
\definecolor{mygreen}    {RGB}{ 51, 160,  44}  % qual_Paired_12_04
\definecolor{myred}      {RGB}{227,  26,  28}  % qual_Paired_12_06

% Beamer colors
%\setbeamercolor{alerted text}{fg=myblue} 
\setbeamercolor{titlelike}{parent=palette primary}
\setbeamercolor{frametitle}{parent=palette secondary}
\setbeamercolor{block}{parent=palette secondary}
\newcommand{\textrbf}[1]{\textcolor{myred}{\textbf{#1}}}
\newcommand{\textgbf}[1]{\textcolor{mygreen}{\textbf{#1}}}
\newcommand{\textbbf}[1]{\textcolor{myblue}{\textbf{#1}}}

% Transparent blocks
\addtobeamertemplate{block begin}{\pgfsetfillopacity{0.75}}{\pgfsetfillopacity{1}}
\addtobeamertemplate{block alerted begin}{\pgfsetfillopacity{0.75}}{\pgfsetfillopacity{1}}
\addtobeamertemplate{block example begin}{\pgfsetfillopacity{0.75}}{\pgfsetfillopacity{1}}

% ----------------------------------------------------------------------
% Math shortcuts
% ----------------------------------------------------------------------

% Vectors and tensors
\newcommand{\vect}[1]{\va*{#1}} % bold arrow vectors
\newcommand{\tens}[1]{\vb*{#1}} % bold tensors

% Differential operators
\renewcommand{\div}[1]{\mathrm{div}\,#1}            % divergence
\renewcommand{\grad}[1]{\vect{\mathrm{grad}}\,#1}   % gradient
\newcommand{\tdiv}[1]{\vect{\mathrm{div}}\,#1}      % tensor divergence
\newcommand{\tgrad}[1]{\tens{\mathrm{grad}}\,#1}    % tensor gradient
\newcommand{\matdv}[1]{\pdv{#1}{t}+\vect{v}\cdot\grad{}\,#1}  % material dv.

% Common notations
\newcommand{\doteps}[0]{\dot{\epsilon}} % epsilon dot
\newcommand{\IDT}[0]{\tens{\delta}}     % Identity tensor
\newcommand{\CST}[0]{\tens{\sigma}}     % Cauchy stress tensor
\newcommand{\DST}[0]{\tens{\tau}}       % deviatoric stress tensor
\newcommand{\SRT}[0]{\tens{\doteps}}    % strain-rate tensor
\newcommand{\vv}[0]{\vect{v}}           % velocity vector
\newcommand{\vsia}[0]{\vv_{\mathrm{SIA}}}   % SIA velocity
\newcommand{\vssa}[0]{\vv_{\mathrm{SSA}}}   % SSA velocity
\newcommand{\PDD}[0]{\mathrm{PDD}}
\newcommand{\sPDD}[0]{\sigma_{\mathrm{PDD}}}

% Common units
\newcommand{\e}[1]{\ensuremath{\times 10^{#1}}}
\newcommand{\chem}[1]{\ensuremath{\mathsf{#1}}}
\newcommand{\unit}[1]{\ensuremath{\mathsf{#1}}}
\newcommand{\degree}[0]{\ensuremath{^{\circ}}}
\newcommand{\degC}[0]{\unit{{\degree}C}}

% TikZ invisible math node
\newcommand<>{\mathnode}[2]{%
  \alt#3{\tikz[baseline, remember picture]%
         \node[anchor=base, inner sep=0pt, myred](#1){$#2$};}%
        {#2}}

% TikZ left-aligned semi-transparent text box
\newcommand{\alphabox}[1]{%
  \flushleft\tikz\node[fill=white, fill opacity=0.75, text opacity=1,%
                       align=left, inner sep=1em] {#1};}

% strike-out
\usepackage[normalem]{ulem}
\renewcommand<>{\sout}[1]{\only#2{\beameroriginal{\sout}}{#1}}


% ----------------------------------------------------------------------
% Special frames
% ----------------------------------------------------------------------

\newlength{\imgwidth}

\newenvironment{backgroundframe}[4][c]% options, image, alpha, title
  {% check width of scale image
   \settowidth{\imgwidth}{\includegraphics[height=\paperheight]{#2}}%
   % add image as background canvas
   \setbeamertemplate{background canvas}{%
    \tikz[overlay, remember picture]%
      \node[opacity=1.0] at (current page.center) {%
        \ifdim\imgwidth<\paperwidth
          \includegraphics[width=\paperwidth]{#2}
        \else
          \includegraphics[height=\paperheight]{#2}
        \fi
      };}%
   % add semi-transparent rectangle overlay
   \setbeamertemplate{background}{%
    \tikz\node[fill=white, inner sep=0, opacity=#3,%
               text width=\paperwidth, text height=\paperheight]{};}%
   \begin{frame}[#1]{#4}}
  {\end{frame}}


% section frame with custom image
%\newenvironment{sectionframe}[4][]% options, image, alpha, title
%  {\begin{backgroundframe}{#2}{#3}{}\centering{\huge #4}\\\bigskip}
%  {\end{backgroundframe}}
\newenvironment{sectionframe}[2][]% options, title
  {\begin{backgroundframe}{art_hodel_1927_rigi}{0.75}{}\centering{\huge #2}\\\bigskip}
  {\footlineextra{Background: Hodel, 1927.}\end{backgroundframe}}

% coverfig source: http://tex.stackexchange.com/a/101073
\newcommand<>{\coverfig}[1]{
  \coordinate (zone) at ($(fig.south)!#1!(fig.north)$);
  \fill#2[white, opacity=1.0] (fig.south west) rectangle(fig.east|-zone);
}

%% covergig modified
%\newcommand<>{\coverfig}[2]{\begin{tikzpicture}
%  \node[inner sep=0] (fig) {#1};
%  \coordinate (zone) at ($(fig.south)!#2!(fig.north)$);
%  \fill<1>[black, opacity=0.75] (fig.south west) rectangle (fig.east|-zone);
%\end{tikzpicture}}

% ----------------------------------------------------------------------
% Footline macro
% From http://tex.stackexchange.com/questions/5491/
%      how-do-i-insert-text-into-the-footline-of-a-specific-slide-in-beamer
% ----------------------------------------------------------------------

\makeatletter

% add a macro that saves its argument
\newcommand{\footlineextra}[1]{\gdef\insertfootlineextra{#1}}
%\newbox\footlineextrabox

% add a beamer template that sets the saved argument in a box.
% The * means that the beamer font and color "footline extra" are automatically added. 
\defbeamertemplate*{footline extra}{default}{
    %\begin{beamercolorbox}[ht=2.25ex,dp=1ex,leftskip=\Gm@lmargin]{footline extra}
    \hspace{\Gm@lmargin}\insertfootlineextra
    %\par\vspace{2.5pt}
    %\end{beamercolorbox}
}

\addtobeamertemplate{footline}{%
    % set the box with the extra footline material but make it add no vertical space
    %\setbox\footlineextrabox=\vbox{\usebeamertemplate*{footline extra}}
    %\vskip -\ht\footlineextrabox
    %\vskip -\dp\footlineextrabox
    %\box\footlineextrabox%
    \usebeamertemplate*{footline extra}
}

% alternatively override the footline template
%\setbeamertemplate{footline}{\insertfootlineextra\hfill\insertframenumber\,/\,\inserttotalframenumber}}

% patch \begin{frame} to reset the footline extra material
\let\beamer@original@frame=\frame
\def\frame{\gdef\insertfootlineextra{}\beamer@original@frame}
\footlineextra{}
\makeatother

\setbeamercolor{footline extra}{fg=structure.fg}% for instance

