% ======================================================================
% Paleoglacier modelling example on the Alps
% ======================================================================

    \begin{sectionframe}{art_hodel_1927_rigi}{0.75}{Example}
      \emph{Modelling paleoglaciers in the Alps.}
      \footlineextra{Image: Hodel, 1927.}
    \end{sectionframe}

% ----------------------------------------------------------------------
\subsection{Context}
% ----------------------------------------------------------------------

    \begin{sectionframe}{art_hodel_1927_rigi}{0.75}{Context}
      \emph{The Alps at the Last Glacial Maximum.}
      \footlineextra{Image: Hodel, 1927.}
    \end{sectionframe}

    \begin{frame}{Last Glacial Maximum ice extent}
      \includegraphics[width=\linewidth]{alpcyc_locmap}
      \footlineextra{Data: SRTM; Natural Earth; Ehlers et al., 2011.}
    \end{frame}

    \begin{backgroundframe}{photo_mount_logan}{0.0}
          {Last Glacial Maximum ice thickness}
      \footlineextra{Photo: Kluane ice cap, Richard Droker.}
    \end{backgroundframe}

    \begin{sectionframe}{art_hodel_1927_rigi}{0.75}{Problem statement}
      \begin{itemize}
        \item Can an ice flow model reproduce geologic reconstructions?
          \begin{itemize}
            \item Last Glacial Maximum ice \alert{extent} from moraines
            \item Last Glacial Maximum ice \alert{thickness} from trimlines
          \end{itemize}
      \end{itemize}
      \bigskip\bigskip\bigskip\pause
      Tool: Parallel Ice Sheet Model (PISM)\\
      \bigskip
      Method: High-resolution simulation of the last glacial cycle (120--0 ka)
    \end{sectionframe}


% ----------------------------------------------------------------------
\subsection{Climate forcing}
% ----------------------------------------------------------------------

    \begin{sectionframe}{art_hodel_1927_rigi}{0.75}{Climate forcing}
      \emph{Sensitivity study}
      \footlineextra{Image: Hodel, 1927.}
    \end{sectionframe}

    \begin{frame}{Present-day climate patterns}
      \includegraphics[width=\linewidth]{alpcyc_inputs}
      \footlineextra{Data: WorldClim; ERA-Interim; Goutorbe et al., 2011.}
    \end{frame}

    \begin{frame}{Paleoclimatic evolution}
      \begin{columns}
      \column{80mm}
      \begin{itemize}
        \item Simple approach
          \begin{itemize}
            \item Temperature offset time-series
            \item Constant precipitation
          \end{itemize}
        \bigskip\pause
        \item Choice criteria
          \begin{itemize}
            \item Continuous 120--0\,ka
            \item High resolution $\approx$1\,ka
            \item Proxy for temperature
            \item \sout<3->{Not far from the Alps}
          \end{itemize}
        \bigskip\pause
        \item Time-dependent temperature change from
          \begin{itemize}
            \item \textbbf{GRIP} Greenland ice \chem{\delta^{18}O}
            \item \textrbf{EPICA} Antarctic ice \chem{\delta^{18}O}
            \item \textgbf{MD01-2444} sediment \chem{U^{K'}_{37}}
          \end{itemize}
      \end{itemize}
      \column{40mm}
      \includegraphics[height=80mm]{alpcyc_records}
      \end{columns}
    \end{frame}

    \begin{frame}{Ice volume evolution}
      \begin{tikzpicture}
        \node[inner sep=0] (fig)
          {\includegraphics[width=\linewidth]{alpcyc_timeseries}};
        \coverfig<1>{0.53}
        \coverfig<2>{0.0}
      \end{tikzpicture}
      \uncover<2>{Ice volume fluctuations are \alert{rapid},
                  but smaller with \textrbf{EPICA} forcing.}
    \end{frame}

    \begin{frame}{Glaciated areas MIS 2 and 4}
      \begin{tikzpicture}
        \node[inner sep=0] (fig)
          {\includegraphics[width=\linewidth]{alpcyc_footprints}};
        \coverfig<1>{0.48}
        \coverfig<2>{0.0}
      \end{tikzpicture}
      \uncover<2>{Only \textrbf{EPICA} forcing yield
                  realistic \alert{MIS 4} conditions}
    \end{frame}

    \begin{frame}{Increasing spatial resolution}
      \includegraphics<1>[width=\linewidth]{alpcyc_boottopo_5km}
      \includegraphics<2>[width=\linewidth]{alpcyc_boottopo_1km}
      \footlineextra{Data: SRTM.}
    \end{frame}

% ----------------------------------------------------------------------
\subsection{Results}
% ----------------------------------------------------------------------

    \begin{sectionframe}{art_hodel_1927_rigi}{0.75}{Results}
      \emph{Last Glacial Maximum (29--17 ka)}
      \footlineextra{Image: Hodel, 1927.}
    \end{sectionframe}

    \begin{frame}{}
      \centering
      Alps glacial cycle animation\\
      \bigskip
      \url{https://polybox.ethz.ch/index.php/s/XaIAM3NdK9tGNUz}
    \end{frame}

    \begin{frame}{Last Glacial Maximum snapshot}
      \includegraphics<1>[width=\linewidth]{alpcyc_lgmvel_26ka}
      \includegraphics<2>[width=\linewidth]{alpcyc_lgmvel_25ka}
      \includegraphics<3>[width=\linewidth]{alpcyc_lgmvel_24ka}
      \includegraphics<4>[width=\linewidth]{alpcyc_lgmvel_21ka}
      \includegraphics<5>[width=\linewidth]{alpcyc_lgmvel_18ka}
    \end{frame}

    \begin{frame}{Timing of Last Glacial Maximum}
      \includegraphics[width=\linewidth]{alpcyc_timing}
    \end{frame}

    \begin{sectionframe}{art_hodel_1927_rigi}{0.75}{Conclusions}
      \begin{itemize}
        \item The modelled \alert{Last Glacial Maximum} is a transient stage.
          \begin{itemize}
            \item Glaciers are not in equilibrium with climate.
            \item Timing potentially varies across the range.
          \end{itemize}
        \pause
        \item Its extent can be modelled using \alert{modern precipitation}.
          \begin{itemize}
            \item Model could be right for wrong reasons, or
            \item Southerly wind regime was short-lived.
          \end{itemize}
        \pause
        \item Its thickness is \alert{much higher} than in reconstructions.
          \begin{itemize}
            \item Modelled surface elevation is 800 m above trimlines.
            \item Rhone valley trimlines match thermal boundary.
          \end{itemize}
      \end{itemize}
      \footlineextra{Image: Hodel, 1927.}
    \end{sectionframe}

