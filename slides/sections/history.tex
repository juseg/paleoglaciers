% ======================================================================
% History on Alpine glaciation studies
% ======================================================================

    \begin{sectionframe}{art_hodel_1927_rigi}{0.75}{History}
      \emph{The study of past glaciers\\\bigskip
            began in the 19th century in the Alps.}
      \footlineextra{Image: Hodel, 1927.}
    \end{sectionframe}


% ----------------------------------------------------------------------
\subsection{Glacial theory in the Alps}
% ----------------------------------------------------------------------

    \begin{sectionframe}{art_hodel_1927_rigi}{0.75}{Glacial theory}
      \emph{A new explanation for the origin of erratic blocs.}
      \footlineextra{Image: Hodel, 1927.}
    \end{sectionframe}

% Glacial theory

    \begin{backgroundframe}{art_hodel_1927_rigi}{0.75}{}
      J.-P. Perraudin, 1815\\
      \bigskip
      % ``Les glaciers de nos montagnes ont eu jadis une bien plus grande
      %   extension qu'aujourd'hui.'' p.241
      \emph{``Toute notre vallée, jusqu'à une grande hauteur au-dessus de la
              Drance, a été occupée par un vaste glacier, qui se prolongeait
              jusques à Martigny, comme le prouvent les blocs de roches qu'on
              trouve dans les environs de cette ville, et qui sont trop gros
              pour que l'eau ait put les y amener.''}\\
      \bigskip
      in J. de Charpentier,\\
      \emph{Essai sur les glaciers et sur le terrain erratique
            du bassin du Rhône},
      p.241--242, 1841.
      \footlineextra{Image: Hodel, 1927.}
    \end{backgroundframe}

    \begin{frame}{Mapping the erratic deposit (1841)}
      \includegraphics[width=\linewidth]{map_decharpentier_1841}
      \footlineextra{Source: de Charpentier, 1841.}
    \end{frame}


    \begin{frame}{The glacial theory was first controversial}
      \begin{columns}
        \column{60mm}
          \begin{itemize}
            \item<+-> Glacial landforms had been explained by a great flood.
            \bigskip
            \item<+-> Single or multiple glaciations?
              \begin{itemize}
                \item catastrophism vs uniformitarianism.
              \end{itemize}
            \bigskip
            \item<+-> Greenland and Antarctica had not yet been explored.
          \end{itemize}
        \column{60mm}
          \includegraphics<1-3>[width=60mm]{artwork-cole-1829-deluge-720p.jpg}
      \end{columns}
      \footlineextra{Source: Cole, 1829.}
    \end{frame}


% How many glaciations

    \begin{frame}{Two glaciations in North America (1882)}
      \includegraphics[width=\linewidth]{map-chamberlin-1882}
      \footlineextra{Source: Chamberlin, 1882.}
    \end{frame}

    \begin{frame}{Four glaciations in the Alps (1909)}
      \begin{columns}
        \column{45mm}
          \begin{itemize}
            \item<+-> In 1909, at least four glaciations
                      were identified in the Alps.
              \alert{four} glaciations
              \begin{itemize}
                \item Würm
                \item Riss
                \item Mindel
                \item Günz
              \end{itemize}
            \item<+-> In 2011, ``at least \alert{eight}, but probably more
                      lowland glaciations during the Quaternary.''
          \end{itemize}
        \column{75mm}
          \includegraphics<1->[height=80mm]{map_penck_bruckner_1909}
      \end{columns}
      \footlineextra{Source: Penck and Brückner, 1909
                             \uncover<1>{; Preusser et al. (2011).}}
    \end{frame}


% ----------------------------------------------------------------------
\subsection{Discovery of glacial cycles}
% ----------------------------------------------------------------------

    \begin{sectionframe}{art_hodel_1927_rigi}{0.75}{Glacial cycles}
      \emph{How many glaciations were there in total?}
      \footlineextra{Image: Hodel, 1927.}
    \end{sectionframe}

% Oxygen isotopes and temperature

    \begin{frame}{Benthic foraminifera}
      \includegraphics[height=80mm]{photo-morgane-foram-720p}
      \footlineextra{Photo: Morgane Brosse}
    \end{frame}

    \begin{frame}{Isotopic composition of the ocean}
      \begin{columns}
        \column{60mm}
          \includegraphics<1->[width=\linewidth]{cartoon-umich-d18o-a}
        \column{60mm}
          \includegraphics<2->[width=\linewidth]{cartoon-umich-d18o-b}
      \end{columns}
      %\footlineextra{Source: \url{http://www.globalchange.umich.edu/gctext/Inquiries/Inquiries_by_Unit/Unit_8a.htm}{University of Michigan}.
    \end{frame}

    \begin{frame}[t]{Glacial cycles}
      \begin{columns}
        \column{30mm}
          \includegraphics<1->[width=\linewidth]{photo-morgane-foram-720p}\\
          \includegraphics<3->[width=\linewidth]{photo-csiro-bubbles-720p}
        \column{90mm}
          \includegraphics<1>[width=\linewidth]{paleo-timeseries-lr04-5ka}
          \includegraphics<2->[width=\linewidth]{paleo-timeseries-lr04-2ka}\\
          \includegraphics<3->[width=\linewidth]{paleo-timeseries-epica-2ka}
      \end{columns}
      \footlineextra{Data: Lisiecki and Raymo (2005), Jouzel et al., 2007}
    \end{frame}

% Ice cover past and present

    \begin{frame}{Land ice cover at present}
      \begin{columns}
        \column{60mm}
          \includegraphics[width=\linewidth]{ehlers-gibbard-2007-fig09}\\
          \bigskip
          Antarctica: 58.3 m s.l.e.\\
          (Fretwell et al., 2013)
        \column{60mm}
          \includegraphics[width=\linewidth]{ehlers-gibbard-2007-fig10}\\
          \bigskip
          Greenland: 7.3 m s.l.e.\\
          (Bamber et al., 2013)
      \end{columns}
      \footlineextra{Source: Ehlers and Gibbard, 2007}
    \end{frame}

    \begin{frame}{Last Glacial Maximum}
      \begin{columns}
        \column{60mm}
          \includegraphics[width=\linewidth]{ehlers-gibbard-2007-fig07}
        \column{60mm}
          \includegraphics[width=\linewidth]{ehlers-gibbard-2007-fig08}
      \end{columns}
      \bigskip
      Additional 120 to 135 m s.l.e. (Clark and Mix, 2002)
      \footlineextra{Source: Ehlers and Gibbard, 2007}
    \end{frame}

    \begin{frame}{Pleistocene maximum glaciation}
      \begin{columns}
        \column{60mm}
          \includegraphics[width=\linewidth]{ehlers-gibbard-2007-fig05}
        \column{60mm}
          \includegraphics[width=\linewidth]{ehlers-gibbard-2007-fig06}
      \end{columns}
      \bigskip
      Intermediate stages are still poorly known...
      \footlineextra{Source: Ehlers and Gibbard, 2007}
    \end{frame}

