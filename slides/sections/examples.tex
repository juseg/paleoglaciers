% ======================================================================
% Paleoglacier modelling examples
% ======================================================================

% ----------------------------------------------------------------------
\subsection{Alpine ice cap}
% ----------------------------------------------------------------------

    \begin{frame}{Past and present glaciers}
      \centering
      \includegraphics[height=60mm]{plot_worldmap}
    \end{frame}

    \begin{backgroundframe}{map_alps}{0.0}{}
      \footlineextra{Data: Ehlers and Gibbard, 2003; ETOPO1; Natural Earth.}
    \end{backgroundframe}

    \begin{backgroundframe}{photo_mount_logan}{0.0}{}
      \vspace{6cm}\hfill
      \begin{beamercolorbox}[sep=1em,wd=45mm]{titlelike}
        Mount Logan, Canada
      \end{beamercolorbox}
      \footlineextra{Photo: Richard Droker}
    \end{backgroundframe}

    \begin{frame}{Problem statement}
      Given the known Last Glacial Maximum extent...
      \bigskip
      \begin{itemize}
        \item Where are the major \alert{nucleation centres}?
        \bigskip
        \item What are the patterns of \alert{last deglaciation}?
      \end{itemize}
      \bigskip\bigskip\bigskip
      \centering
      Tool: Parallel Ice Sheet Model\\
      \url{http://www.pism-docs.org/wiki/doku.php}
    \end{frame}

    \begin{frame}{}
      \centering
      Alpine ice cap animation\\
      \bigskip
      \url{https://polybox.ethz.ch/index.php/s/XaIAM3NdK9tGNUz}\\
    \end{frame}

% ----------------------------------------------------------------------
\subsection{Cordilleran ice sheet}
% ----------------------------------------------------------------------

    \begin{frame}{Past and present glaciers}
      \centering
      \includegraphics[height=60mm]{plot_worldmap}
    \end{frame}

    \begin{backgroundframe}{map_northamerica}{0.0}{}
      \footlineextra{Data: Dyke et al., 2003; ETOPO1; Natural Earth.}
    \end{backgroundframe}

    \begin{backgroundframe}{map_cordillera}{0.0}{}
      \footlineextra{Data: Dyke et al., 2003; ETOPO1; Natural Earth.}
    \end{backgroundframe}

%    \begin{frame}{Reanalysed temperatures (1981--2010)}
%      \centering
%      \includegraphics<1>{plot-temp-01}
%      \includegraphics<2>{plot-temp-02}\\
%      \uncover<2>{Strong variations in temperature and temperature seasonality}
%      \footlineextra{Data: NARR}
%    \end{frame}

%    \begin{frame}{Reanalysed precipitations (1981--2010)}
%      \centering
%      \includegraphics<1>{plot-prec-01}
%      \includegraphics<2>{plot-prec-02}\\
%      \uncover<2>{Strong variations in precipitation and timing of precipitation}
%      \footlineextra{Data: NARR}
%    \end{frame}

%    \begin{frame}{Simulations of the last glacial cycle (120--0\,kyr)}
%      \begin{columns}
%      \column{80mm}
%      \begin{itemize}
%        \item<+-> Spatial climate patterns from the NARR
%          \begin{itemize}
%            \item monthly temperature
%            \item monthly precipitation
%            \item monthly temperature standard deviation
%          \end{itemize}
%        \item<+-> Time-dependent temperature change from
%          \begin{itemize}
%            \item two Greenland \alert{ice cores}
%              \begin{itemize}
%            \item GRIP
%                \item NGRIP
%              \end{itemize}
%            \item two Antarctic \alert{ice cores}
%              \begin{itemize}
%                \item EPICA
%                  \item Vostok
%              \end{itemize}
%            \item two Pacific ocean \alert{sediment records}
%              \begin{itemize}
%                \item ODP 1012
%                \item ODP 1020
%              \end{itemize}
%          \end{itemize}
%      \end{itemize}
%      \column{40mm}
%      \uncover<2>{\includegraphics{map_records}}
%      \end{columns}
%    \end{frame}

    \begin{frame}{}
      \centering
      Cordilleran ice sheet animation\\
      \bigskip
      \url{https://polybox.ethz.ch/index.php/s/4prO6tvF3nh3G1W}\\
    \end{frame}

    \begin{frame}{Conclusions}
      \begin{itemize}
        \item The maximum stage appears short-lived.
        \bigskip
        \item Most of the glacial cycle features mountain ice caps.
          \begin{itemize}
            \item Skeena Mountains
          \end{itemize}
        \bigskip
        \item Deglaciation towards northern interior ranges.
          \begin{itemize}
            \item Selwyn Mountains, Skeena Mountains
          \end{itemize}
      \end{itemize}
    \end{frame}

% ----------------------------------------------------------------------
\subsection{Haizishan ice cap}
% ----------------------------------------------------------------------

    \begin{frame}{Past and present glaciers}
      \centering
      \includegraphics[height=60mm]{plot_worldmap}
    \end{frame}

    \begin{frame}{Past and present glaciers}
      \centering
      \includegraphics[width=\linewidth]{fu_etal_2013_fig09}
      \footlineextra{Source: Fu et al., 2013.}
    \end{frame}

    \begin{frame}{Past and present glaciers}
      \centering
      \includegraphics[height=60mm]{fu_etal_2013_fig07}
      \footlineextra{Source: Fu et al., 2013.}
    \end{frame}

    \begin{frame}{Modelling Haizishan ice cap}
      \centering
      \includegraphics[height=60mm]{plot_hzs_final}
    \end{frame}

