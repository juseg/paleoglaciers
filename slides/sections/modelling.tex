% ======================================================================
% Glacier modelling principles
% ======================================================================

    \begin{sectionframe}{photo_gries_icefall}{0.75}{Glacier modelling}
      \emph{Using approximated ice flow physics and numerical methods}
    \end{sectionframe}

% ----------------------------------------------------------------------
\subsection{Overview}
% ----------------------------------------------------------------------

    \begin{frame}{}
      \centering
      ``The future of glaciers''\\
      \bigskip
      \url{https://www.youtube.com/watch?v=eJNIr_0zOyk}
      \footlineextra{Author: Jouvet et al., 2013.}
    \end{frame}

    \begin{frame}{From the natural world to the model}
      \includegraphics{graph_real_to_num}
    \end{frame}

    \begin{frame}{What is a glacier model?}
      \begin{tikzpicture}[align=center]
        \fill[mygreen!75] (-6,-2.5) rectangle (-2.5,2.5);
        \fill[myblue!75] (-2,-3) rectangle (2,3);
        \fill[myred!75] (2.5,-2.5) rectangle (6,2.5);
        \draw[-latex] (-2.5,0) -- (-2,0);
        \draw[-latex] (2,0) -- (2.5,0);
        \node at (-4.25,0) {
          \only<-1>{INPUT}
          \only<2->{topography\\[1em]
                    climate\\[1em]
                    basal heat flux\\[1em]
                    initial conditions\\[1em]
                    ...}
        };
        \node at (0,0) {
          \only<-3>{MODEL}
          \only<4->{thermodynamical core\\[3em]
                    boundary conditions}
        };
        \node at (4.25,0) {
          \only<-2>{OUTPUT}
          \only<3->{glacier geometry\\[1em]
                    velocities\\[1em]
                    temperature\\[1em]
                    stresses\\[1em]
                    ...}
        };
      \end{tikzpicture}
    \end{frame}

% ----------------------------------------------------------------------
\subsection{Ice thermodynamics}
% ----------------------------------------------------------------------

\begin{frame}{Field equations}
  \begin{itemize}[<+->]
  \item Conservation of volume (incompressibility of flow)
    \begin{equation*}
      \div{
         \mathnode<1>{vv}{\vv}
        } = 0
    \end{equation*}
  \item Balance of stresses (Stokes equation)
    \begin{equation*}
      \tdiv{
         \mathnode<2>{cst}{\CST}
       } + 
         \mathnode<2>{rhog}{\rho\,\vect{g}}
       = \vect{0}
    \end{equation*}
  \item Constitutive law for ice (Glen's law)
    \begin{equation*}
        \mathnode<3>{srt}{\SRT}
      = A_0\,e^\frac{-Q}{RT_{pa}}\,\tau_e^{n-1}\,
        \mathnode<3>{dst}{\DST}
      \end{equation*}
  \item Conservation of energy (heat equation)
    \begin{equation*}
        \pdv{
          \mathnode<4>{temp}{T}
        }{t}+\vect{v}\cdot\grad{}\,T =
        \mathnode<4>{diff}{\frac{k}{\rho c} \Delta T}
         +
        \mathnode<4>{hsrc}{\frac{\tr(\DST\SRT)}{\rho c}}
    \end{equation*}
  \end{itemize}
  \begin{tikzpicture}[>=latex, overlay, remember picture, myred, font=\scriptsize]
    \path<1> (vv) -- +(3, -0.25) node (vvt) {ice velocity vector};
    \path<2> (rhog) -- +(3, +0.25) node (cstt) {stress tensor};
    \path<2> (rhog) -- +(3, -0.25) node (rhogt) {gravitational force};
    \path<3> (dst) -- +(2, +0.25) node (srtt) {strain-rate tensor};
    \path<3> (dst) -- +(2, -0.25) node (dstt) {deviatoric stress tensor};
    \path<4> (temp) -- +(-2, +0.25) node (tempt) {ice temperature};
    \path<4> (hsrc) -- +(2, +0.25) node (difft) {diffusion term};
    \path<4> (hsrc) -- +(2, -0.25) node (hsrct) {source term};
    \path[draw, ->]<1> (vv) .. controls +(0,-0.5) and +(-1,0) .. (vvt.west);
    \path[draw, ->]<2> (cst) .. controls +(0,+0.5) and +(-1,0.5) .. (cstt.west);
    \path[draw, ->]<2> (rhog) .. controls +(0,-0.5) and +(-1,-0.5) .. (rhogt.west);
    \path[draw, ->]<3> (srt) .. controls +(0,+0.5) and +(-1,0.5) .. (srtt.west);
    \path[draw, ->]<3> (dst) .. controls +(0,-0.5) and +(-0.5,-0.5) .. (dstt.west);
    \path[draw, ->]<4> (temp) .. controls +(0,0.5) and +(1,0) .. (tempt.east);
    \path[draw, ->]<4> (diff) .. controls +(0,0.5) and +(-1,1) .. (difft.west);
    \path[draw, ->]<4> (hsrc) .. controls +(0,-0.5) and +(-1,-1) .. (hsrct.west);
  \end{tikzpicture}
  \only<5>{}
\end{frame}

%    \begin{frame}{Shallow approximations of the stress balance}
%      $$\vec{\mathrm{div}} \, \bm\sigma + \rho \, \vec{g} = 0
%        \qquad\Rightarrow\qquad\left\{\begin{array}{l}
%        \alert<3-4>{\frac{\partial\tau_{xx}}{\partial x}}
%        \alert<3-4>{+\frac{\partial\tau_{xy}}{\partial y}}
%        \alert<2-4>{+\frac{\partial\tau_{xz}}{\partial z}}
%        \alert<2-4>{=\frac{\partial p}{\partial x}}\\
%        \alert<3-4>{\frac{\partial\tau_{yx}}{\partial x}}
%        \alert<3-4>{+\frac{\partial\tau_{yy}}{\partial y}}
%        \alert<2-4>{+\frac{\partial\tau_{yz}}{\partial z}}
%        \alert<2-4>{=\frac{\partial p}{\partial y}}\\
%        \alert<4-4>{\frac{\partial\tau_{zx}}{\partial x}}
%        \alert<4-4>{+\frac{\partial\tau_{zy}}{\partial y}}
%        \alert<4-4>{+\frac{\partial\tau_{zz}}{\partial z}}
%        \alert<2-4>{=\frac{\partial p}{\partial z} - \rho g}
%        \end{array}\right.$$
%      \begin{itemize}[<+(1)-| alert@+(1)>]
%        \item Shallow ice approximation
%        \item Shallow shelf approximation
%        \item Full stokes
%        \item<+(1)-> And several others...
%      \end{itemize}
%      \pause
%      Stress balance approximations often define the scope of a glacier model.
%    \end{frame}

    \begin{frame}{Shallow approximations}
      \centering
      % TikZ styles
\tikzstyle{vsia}=[myblue, thick]
\tikzstyle{vssa}=[myred, thick]

\begin{tikzpicture}[>=latex]

% white background
\fill[white] (-1.5,-0.5) rectangle +(11.0,6.5);
%\draw [help lines, lightgray] (0,0) grid (8.0,5.5);

% transition lines and calving front coordinates
\coordinate (tr1) at (2.75, 0);  % transition one
\coordinate (tr2) at (5.25, 0);  % transition two
\coordinate (cf) at (7.75, 2);
\draw[lightgray, name path=tr1] (tr1 |- 0,0) -- +(0,5.5) ;
\draw[lightgray, name path=tr2] (tr2 |- 0,0) -- +(0,5.5) ;

% location of the velocity profiles
\path [name path=x1] (1.0,0) -- +(0,5) ;
\path [name path=x2] (3.0,0) -- +(0,5) ;
\path [name path=x3] (5.5,0) -- +(0,5) ;

% bedrock topography
\draw [name path=bed]
    (0,2) .. controls +(-5:4) and +(180:2) .. (8,1);
\fill [pattern=north east lines]
    (0,2) .. controls +(-5:4) and +(180:2) .. (8,1)
          -- +(0,-0.25)
          .. controls +(180:2) and +(-5:4) .. (0,1.75);

% locate the grounding line
\path [name intersections={of=bed and tr2, by=gl}] ;

% surface topograpy
\draw [name path=surf]
    (0,4.5) .. controls +(-10:4) and +(180:3) .. ($(cf)+(0,0.1)$)
          -- (cf) ;

% ice shelf base
\draw [name path=base]
    (gl) .. controls +(15:0.5) and +(180:1) .. ($(cf)-(0,0.4)$)
         -- (cf);

% sea level5
\draw [name path=sl] (cf) -- (cf -| 8,0) ;

% first velocity profile
\only<2->{
  \path [name intersections={of=surf and x1, by=s1}] ;
  \path [name intersections={of=bed and x1, by=b1}] ;
  \draw[vsia] (b1) -- (s1);
  \draw[vsia, name path=vsia]
    (b1) .. controls ++(0.75,0.5) .. ($(s1)+(0.75,0)$);
  \path[name path=vgrid] (s1) foreach \x in {1,...,5} { -- +(2,0) ++(0,-0.5) };
  \path[name intersections={of=x1 and vgrid, name=a, total=\t}];
  \path[name intersections={of=vsia and vgrid, name=b, total=\t}];
  \draw[vsia, ->] (a-1) -- (b-1);
  \draw[vsia, ->] (a-2) -- (b-2);
  \draw[vsia, ->] (a-3) -- (b-3) node [midway, above] {$\vsia$};
  \draw[vsia, ->] (a-4) -- (b-4);
  \draw[vsia, ->] (a-5) -- (b-5);
}

% second velocity profile
\only<4->{
  \path [name intersections={of=surf and x2, by=s2}] ;
  \path [name intersections={of=bed and x2, by=b2}] ;
  \draw[vssa] (b2) -- (s2);
  \draw[vssa, name path=vssa]
    (b2) -- ($(b2)+(1,0)$) -- ($(s2)+(1,0)$);
  \draw[vsia, name path=vsia]
    ($(b2)+(1,0)$) .. controls ++(1,0.5) .. ($(s2)+(2,0)$);
  \path[name path=vgrid] (s2) foreach \x in {1,...,5} { -- +(2,0) ++(0,-0.5) };
  \path[name intersections={of=x2 and vgrid, name=a, total=\t}];
  \path[name intersections={of=vssa and vgrid, name=b, total=\t}];
  \path[name intersections={of=vsia and vgrid, name=c, total=\t}];
  \draw[vssa, ->] (a-1) -- (b-1);
  \draw[vssa, ->] (a-2) -- (b-2);
  \draw[vssa, ->] (a-3) -- (b-3) node [midway, above] {$\vssa$};
  \draw[vssa, ->] (a-4) -- (b-4);
  \draw[vsia, ->] (b-1) -- (c-1);
  \draw[vsia, ->] (b-2) -- (c-2);
  \draw[vsia, ->] (b-3) -- (c-3) node [midway, above] {$\vsia$};
  \draw[vsia, ->] (b-4) -- (c-4);
}

% third velocity profile
\only<3->{
  \path [name intersections={of=surf and x3, by=s3}] ;
  \path [name intersections={of=base and x3, by=b3}] ;
  \draw[vssa] (b3) -- (s3);
  \draw[vssa, name path=vssa]
    (b3) -- ($(b3)+(2,0)$) -- ($(s3)+(2,0)$);
  \path[name path=vgrid] (s3) foreach \x in {1,...,3} { -- +(2,0) ++(0,-0.5) };
  \path[name intersections={of=x3 and vgrid, name=a, total=\t}];
  \path[name intersections={of=vssa and vgrid, name=b, total=\t}];
  \draw[vssa, ->] (a-1) -- (b-1);
  \draw[vssa, ->] (a-2) -- (b-2);
  \draw[vssa, ->] (a-3) -- (b-3) node [midway, above] {$\vssa$};
}

% add transition lines and annotations
\coordinate (m1) at ($(0,0)!0.5!(tr1)$) ;
\coordinate (m2) at ($(tr1)!0.5!(gl)$) ;
\coordinate (m3) at ($(gl)!0.5!(8,0)$) ;
\coordinate (top1) at ($(0,5.25)$) ;
\coordinate (top2) at ($(0,4.75)$) ;
\coordinate (bot) at ($(0,0.25)$) ;

\node<2-> at (m1|-top1) {ice sheet};
\node<4-> at (m2|-top1) {ice stream};
\node<3-> at (m3|-top1) {ice shelf};
\node<2-> [vsia] at (m1|-top2) {Shallow Ice Approximation};
\node<3-> [vssa] at (m3|-top2) {Shallow Shelf Approximation};
\node<2-> at (m1|-bot) {$\vsia\gg\vssa$};
\node<4-> at (m2|-bot) {$\vsia\sim\vssa$};
\node<3-> at (m3|-bot) {$\vsia\ll\vssa$};

\end{tikzpicture}

      \footlineextra{After: Winkelmann et al., 2011}
    \end{frame}

% ----------------------------------------------------------------------
\subsection{Boundary conditions}
% ----------------------------------------------------------------------

    \begin{frame}[label=model-interfaces]{Boundary interfaces}
      \centering
      \input{../figures/pism_interfaces}\\
      \bigskip
      \footlineextra{After: PISM documentation (http://pism-docs.org)}
    \end{frame}

%    \begin{frame}{Boundary conditions}
%      \begin{itemize}
%        \item At the atmosphere interface...
%          \begin{itemize}
%            \item mass is gained and lost (surface mass balance)
%            \item air temperature controls ice temperature
%          \end{itemize}
%        \item At the ocean interface...
%          \begin{itemize}
%            \item mass is gained and lost (basal mass balance)
%            \item water temperature controls ice temperature
%            \item calving of icebergs can occur
%          \end{itemize}
%        \item Lithosphere interface...
%          \begin{itemize}
%            \item mass is gained and lost (basal mass balance)
%            \item geothermal heat flux controls ice temperature
%            \item basal topography controls flow
%            \item sliding can occur
%          \end{itemize}
%      \end{itemize}
%      \pause
%      Boundary models are often crude simplifications of complex processes.
%    \end{frame}

% ----------------------------------------------------------------------
%\subsection{Numerical implementation}
% ----------------------------------------------------------------------

%\begin{frame}{\insertsubsection}
%  \begin{block}{Problem statement}
%    The equations of the physical model describe \alert{continuous} fields, but computers %can deal with a \alert{finite number of points} in time and space
%  \end{block}
%\end{frame}

%\againframe{real-to-num}

%\begin{frame}{Example: the equation of temperature diffusion}
%  \begin{itemize}[<+->]
%    \item We use the previous temperature equation...
%    $$\frac{\partial T}{\partial t}
%      = \mathbf{v} \cdot \vec{\mathrm{grad}}\,T
%      + \frac{k}{\rho c} \Delta T
%      + \frac{4 \mu \dot \epsilon_e^2}{\rho c}$$
%    \item ... considering only diffusion...
%    $$\frac{\partial T}{\partial t}
%      = \frac{k}{\rho c} \Delta T$$
%    \item ... in the vertical dimension only.
%    $$\frac{\partial T}{\partial t}
%      = \kappa\,\frac{\partial^2 T}{\partial z^2}
%      \qquad \mathrm{with} \qquad
%      \kappa = \frac{k}{\rho c}$$
%  \end{itemize}
%\end{frame}

%\begin{frame}{Discretization}
%  \begin{itemize}[<+->]
%    \item In the numerical model, temperature is defined on a grid:
%    $$T_i^n=T(z_i,t^n)$$
%    \item Derivatives are approximated (\alert{finite difference method}):
%    $$\frac{\partial T}{\partial t}\simeq\frac{T_i^{n+1}-T_i^n}{\delta t}
%      \qquad \mathrm{and} \qquad
%      \frac{\partial^2 T}{\partial z^2}\simeq\frac{T_{i+1}^n-2T_i^n+T_{i-1}^n}{\delta z^2}$$
%    \item The temperature equation becomes:
%    $$\frac{T_i^{n+1}-T_i^n}{\delta t}
%    =\kappa\,\frac{T_{i+1}^n-2T_i^n+T_{i-1}^n}{\delta z^2}$$
%    \item This method is said \alert{explicit} as one can write:
%    $$T_i^{n+1} = f(T_{i+1}^n, T_i^n, T_{i-1}^n)$$
%  \end{itemize}
%\end{frame}

%\begin{frame}{Some first-order discretization schemes}
%  \begin{itemize}[<+->]
%    \item Explicit scheme
%    $$\frac{T_i^{n+1}-T_i^n}{\delta t}
%      =\kappa\,\frac{T_{i+1}^n-2T_i^n+T_{i-1}^n}{\delta z^2}$$
%    \item Implicit scheme
%    $$\frac{T_i^{n+1}-T_i^n}{\delta t}
%      =\kappa\,\frac{T_{i+1}^{n+1}-2T_i^{n+1}+T_{i-1}^{n+1}}{\delta z^2}$$
%    \item Semi-implicit (time-centered) scheme
%    $$\frac{T_i^{n+1}-T_i^n}{\delta t}
%      =\frac{1}{2}\,\kappa\,\frac{T_{i+1}^n-2T_i^n+T_{i-1}^n}{\delta z^2}
%      +\frac{1}{2}\,\kappa\,\frac{T_{i+1}^{n+1}-2T_i^{n+1}+T_{i-1}^{n+1}}{\delta z^2}$$
%  \end{itemize}
%\end{frame}

%\begin{frame}{Numerical instability}
%  Different schemes have different properties of \alert{convergence} and \alert{stability}.
%  \begin{columns}
%  \column{55mm}
%    \begin{block}{Stable}
%      \includegraphics[width=\linewidth]{temperature-stable}
%    \end{block}
%  \column{55mm}
%    \begin{block}{Unstable}
%      \includegraphics[width=\linewidth]{temperature-unstable}
%    \end{block}
%  \end{columns}
%\end{frame}

%\begin{frame}{Other discretization methods}
%  There exist other methods than the \alert{finite difference} method:
%  \begin{itemize}[<+->]
%    \item Finite element method
%    \item Discrete element method
%    \item Spectral methods
%    \item ...
%  \end{itemize}
%\end{frame}
