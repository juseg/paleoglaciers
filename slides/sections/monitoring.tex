% ======================================================================
% Glacier monitoring in Greenland
% ======================================================================

    \begin{sectionframe}{photo_gries_icefall}{0.75}{Glacier monitoring}
      \emph{Bowdoin calving glacier, Northwest Greenland}
    \end{sectionframe}

    \begin{frame}{Past and present glaciers}
      \centering
      \includegraphics[height=60mm]{plot_worldmap}
      \footlineextra{Data: Ehlers and Gibbard, 2004.}
    \end{frame}

% ----------------------------------------------------------------------
\subsection{Context}
% ----------------------------------------------------------------------

    \begin{frame}{The Greenland ice sheet is loosing mass}
      \begin{itemize}
        \item Surface mass balance
        \begin{itemize}
          \item Melt
          \item Sublimation
        \end{itemize}
        \pause\bigskip
        \item Dynamic thinning\\
        \small{(acceleration, thinning and retreat of marine-terminating glaciers)}
        \begin{itemize}
          \item Meltwater-induced acceleration
          \item Iceberg calving
        \end{itemize}
      \end{itemize}
    \end{frame}

    \begin{frame}{Greenland outlet glacier hydrology}
      \centering
      \includegraphics[width=\textwidth]{chu_etal_2014_fig01}
      \footlineextra{Source: Chu et al., 2014.}
    \end{frame}

    \begin{frame}{Bowdoin glacier and borehole locations}
      \centering
      \includegraphics[height=70mm]{map_grl}
      \footlineextra{Data: GIMP, MapQuest Open Aerial.}
    \end{frame}

    \begin{frame}{Greenland mass lost from gravity anomalies}
      \centering
      \includegraphics[width=100mm]<1>{grace_01}
      \includegraphics[width=100mm]<2>{grace_02}
      \includegraphics[width=100mm]<3>{grace_03}\\
      \uncover<2->{Mass loss spread to the north-west.}
      \footlineextra{Source: Sutterley et al. (2014)\only<3->{, Sugiyama et al. (2015)}.}
    \end{frame}

% ----------------------------------------------------------------------
\subsection{Setting}
% ----------------------------------------------------------------------

    \begin{backgroundframe}{photo_bowdoin_aerial}{0.0}{}
    \end{backgroundframe}

    \begin{backgroundframe}{photo_bowdoin_camp}{0.0}{}
    \end{backgroundframe}

    \begin{backgroundframe}{photo_bowdoin_surface}{0.0}{}
    \end{backgroundframe}

    \begin{backgroundframe}{photo_bowdoin_night}{0.0}{}
      \vspace{6cm}\hfill
      \begin{beamercolorbox}[sep=1em,wd=30mm]{titlelike}
        27 Dec. 2015
      \end{beamercolorbox}
    \end{backgroundframe}

% ----------------------------------------------------------------------
\subsection{Results}
% ----------------------------------------------------------------------

    \begin{frame}{Temperature profiles from bed to surface}
      \centering
      \includegraphics[height=70mm]{pf_temp}
    \end{frame}

    \begin{frame}{Ice deformation over 8 months.}
      \centering
      \includegraphics[height=70mm]{pf_tilt}
    \end{frame}

    \begin{frame}{}
      \centering
      Bowdoin tilt animation\\
      \bigskip
      \url{https://polybox.ethz.ch/index.php/s/OwVk9gOpYmtu44a/}
    \end{frame}

    \begin{frame}{Surface velocity from remote sensing}
      \begin{columns}
        \column{60mm}
          \centering
          \includegraphics[width=\textwidth]{satvel_landsat}\\
          Landsat
        \column{60mm}
          \centering
          \includegraphics[width=\textwidth]{satvel_sentinel}\\
          Sentinel-1
      \end{columns}
      \footlineextra{Data: T. Abe, D. Sakakibara\only<3->{, S. Leinss}.}
    \end{frame}

    \begin{frame}{Ice deformation vs. surface velocity}
      \centering
      \includegraphics[height=70mm]{ts_satvel}
      \footlineextra{Data: S. Sugiyama, T. Abe, S. Leinss.}
    \end{frame}

    \begin{frame}{}
      \centering
      Bowdoin icequakes timelapse\\
      \bigskip
      \url{https://www.youtube.com/watch?v=U3F6kv3To3Y}\\
      \footlineextra{Author: Podolskiy et al., 2016.}
    \end{frame}

    \begin{frame}{Conclusions}
      \begin{itemize}
        \item Strong spatial variations in temperature
        \bigskip
        \item Basal sliding accounts for 90 \% of surface motion
        \bigskip
        \item Early summer speed-up at the onset of surface melt
        \bigskip
        \item High seismicity near the calving front
      \end{itemize}
      \pause\bigskip
      $\implies$ Field data is important to calibrate glacier models.
    \end{frame}
